\section{Fixtures}
\label{fixtures}
Fixtures\index{fixtures} are data structures and values which may be
referred to by name during testing.  NST provides the ability to use
fixtures across multiple tests and test groups, and to inject fixtures
into the runtime namespace for debugging.

\fbox{Outdated --- see the docstring} 
A set of fixtures is defined using the \texttt{def-fixtures}
macro:\index{def-fixtures@\texttt{def-fixtures}}
\begin{center}
\fbox{FILL IN}
%  \textbf{def-fixtures} \textit{name} \texttt{\&key}
%  \textit{bindings} \textit{uses} \textit{outer}
%  \textit{inner} \textit{documentation}
\end{center}
where
\begin{itemize}
\item \textit{name} is the name to be associated with this set of
  fixtures.
\item \textit{bindings} is a list of name/value lists, as is used for
  Lisp \texttt{let} blocks.
\item \textit{uses} is a list of the names of other fixture sets which
  this declaration assumes to be available.  This declaration is
  optional, but will supress some warnings.
\item \textit{outer} and \textit{inner} are lists of declarations to
  be included in a \texttt{declare} statement respectively outside and
  inside of the \texttt{let}-binding in which the \textit{bindings}
  are applied.
\item \textit{documentation} describes the fixture set.
\end{itemize}
A fixture in one set may refer back to other fixtures in the same set
(\emph{\`a la} \texttt{let*}) but forward references are not allowed.

Examples of fixture definitions:
\begin{verbatim}
\fbox{FILL IN}
%  (def-fixtures f1 :bindings ((c 3) (d 'asdfg)))
%  (def-fixtures f2 :uses (f1) :bindings ((d 4) (e 'asdfg) (f c)))
\end{verbatim}
There are examples of using these fixture definitions in tests (i.e.\
with \texttt{def-test}) in Section~\ref{first-test-use-of-fixtures},
p.~\pageref{first-test-use-of-fixtures}.

%%% Local Variables: 
%%% mode: latex
%%% TeX-master: "manual"
%%% End: 
