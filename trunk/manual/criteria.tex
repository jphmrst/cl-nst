\section{Defining check criteria}
\fbox{THIS WHOLE SECTION NEEDS UPDATING} Do this last, or maybe John
can; it's all macrology.
\subsection{Aliases over checks}
{\ttfamily\begin{tabbing}
\textrm{Syntax: 
}(de\=f-check-alias (name \&rest args)
\\ \> [ documentation ]
\\ \> expansion)
\end{tabbing}}
Defines how a criterion should be rewritten as another criteria.

Examples:
\\ \texttt{(def-check-alias (:forms-eq) `(:predicate eq))}
\\ \texttt{(def-check-alias (:symbol name) `(:eq ',name))}
\par\fbox{Fill in details}

\subsection{Reporting forms}
\texttt{make-check-report}, \texttt{emit-failure}, \texttt{emit-warning}.
\par\fbox{Fill in details}

\subsection{Basic checks on values}
The \texttt{check-forms} return a quoted Lisp expression calculating a
checking report for the given forms.  
{\ttfamily\begin{tabbing}
\textrm{Syntax: 
}(de\=f-value-check (na\=me \=subcriteria-lambda-list
\\ \> \> \> test-forms-lambda-list
\\ \> \> \&key \= blurb-format
\\ \> \>        \> full-format
\\ \> \>        \> stack-transformer)
\\ \> expansion)
\end{tabbing}}
Comma-prefix names from the \texttt{subcriteria-lambda-list}, but do not
comma-prefix names from the \texttt{test-forms-lambda-list}.
Declarations are OK, but quote them.  Examples:
\begin{verbatim}
(def-value-check (:pass () (&rest chk))
  `(declare (ignorable chk))
  `(make-check-result))

(def-value-check (:eq (eq-form) (check-form))
  `(if (eq ,eq-form check-form)
     (make-check-result)
     (emit-failure :format "Not eq to ~s" :args '(,eq-form))))
\end{verbatim}
\par\fbox{Fill in details}

\subsection{General checks}
{\ttfamily\begin{tabbing}
\textrm{Syntax: 
}(de\=f-control-check (na\=me \=subcriteria-lambda-list
\\ \>\> \> test-forms-formal-parameter
\\ \>\> \&key \=stack-transformer
\\ \>\> \> blurb-format
\\ \>\> \> full-format)
\\ \> check-forms)
\end{tabbing}}
Comma-prefix all the names.  \texttt{test-forms-formal-parameter} must
be just a symbol.  Calls out to \texttt{continue-check} are useful;
its second argument must be a Lisp expression which will evaluate to a
list of values.
\begin{verbatim}
(def-control-check (:apply (transform criterion) forms)
  (continue-check criterion
		  `(multiple-value-list (apply #',transform ,forms))))
\end{verbatim}
\par\fbox{Fill in details}

%%% Local Variables: 
%%% mode: latex
%%% TeX-master: "manual"
%%% End: 
