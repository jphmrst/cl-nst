
\section{Test groups}
\label{sec:groups}
Groups of tests\index{group}\index{test group|see{group}} can be
associated with fixture sets, stateful initiatization, and stateful
cleanup.  The syntax of a test group declaration
is:\index{def-test-group@\texttt{def-test-group}}
\begin{verbatim}
  (def-test-group NAME (FIXTURE FIXTURE ... FIXTURE)
    [ (:setup FORM FORM ... FORM) ]
    [ (:cleanup FORM FORM ... FORM) ]
    [ (:startup FORM FORM ... FORM) ]
    [ (:finish FORM FORM ... FORM) ]
    [ (:each-setup FORM FORM ... FORM) ]
    [ (:each-cleanup FORM FORM ... FORM) ]
    [ (:documentation STRING) ]
    TEST
    TEST
    ...
    TEST)
\end{verbatim}

%%% Local Variables: 
%%% mode: latex
%%% TeX-master: "../manual/manual"
%%% TeX-PDF-mode: t
%%% End: 

%
\texttt{NAME} is the name of this test group.  The \texttt{FIXTURE}s
are to be applied to the tests in this group.

The \texttt{:setup} forms\index{setup@\texttt{:setup}} are run after
inclusion of names from fixture sets, but before any tests from the
group.  Individual tests should make no assumptions as to whether the
setup is unique to that test, or whether it is shared among several
tests of that group.  

The \texttt{:cleanup} forms\index{cleanup@\texttt{:cleanup}} are
normally run after the setup completes; however the cleanup form will
not be run if the setup form raises an error.  When the user asks the
runtime system to enter the debugger on an error, the cleanup form
will not run unless the user explicitly enabled a resumption of the
test routine from the debugger.  The cleanup form will be run in other
circumstances, including at a user-requested break in testing at
either failure or error.

The
\texttt{:fixtures-setup}\index{fixtures-setup@\texttt{:fixtures-setup}}
(respectively
\texttt{:fixtures-cleanup})\index{fixtures-cleanup@\texttt{:fixtures-cleanup}}
form is run before fixtures are bound (after their bindings are
released).  These forms are useful, for example, to initialize a
database connection from which the fixture values are drawn.

The \texttt{:each-setup}\index{each-cleanup@\texttt{:each-cleanup}}
and \texttt{:each-cleanup}\index{each-cleanup@\texttt{:each-cleanup}}
forms are run before each test, rather than once for the group.

The \texttt{:documentation}\index{documentation@\texttt{:documentation}}
form sets the docstring for the class.

%%% Local Variables: 
%%% mode: latex
%%% TeX-master: "manual"
%%% TeX-PDF-mode: t
%%% End: 
