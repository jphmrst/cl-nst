
\section{Test groups}
Groups of tests\index{group}\index{test group|see{group}} can be
associated with fixture sets, stateful initiatization, and stateful
cleanup.  The syntax of a test group declaration
is:\index{def-test-group@\texttt{def-test-group}}
\begin{verbatim}
  (def-test-group NAME (FIXTURE FIXTURE ... FIXTURE)
    [ (:setup FORM FORM ... FORM) ]
    [ (:cleanup FORM FORM ... FORM) ]
    [ (:startup FORM FORM ... FORM) ]
    [ (:finish FORM FORM ... FORM) ]
    [ (:each-setup FORM FORM ... FORM) ]
    [ (:each-cleanup FORM FORM ... FORM) ]
    [ (:documentation STRING) ]
    TEST
    TEST
    ...
    TEST)
\end{verbatim}

%%% Local Variables: 
%%% mode: latex
%%% TeX-master: "../manual/manual"
%%% TeX-PDF-mode: t
%%% End: 

where \fbox{UPDATE}
\begin{itemize}
\item\textit{name} is the name of this test group.
\item\textit{fixture-sets} is a list of fixture sets --- given by
  name, or as anonymous fixtures --- to be applied to the tests in
  this group.
\item\textit{forms} are one of the following:
\begin{itemize}
\item A test definition, described below.
\item Test group documentation, \texttt{(:documentation
    \textit{string})}~\index{documentation@\texttt{:documentation}}.
\item A setup form, \texttt{(:setup
    \textit{setup})}~\index{setup@\texttt{:setup}}.  The setup form is
  run after inclusion of names from fixture sets, but before any tests
  from the group.  Individual tests should make no assumptions as to
  whether the setup is unique to that test, or whether it is shared
  among several tests of that group.
\item A cleanup form, \texttt{(:cleanup
    \textit{cleanup})}~\index{cleanup@\texttt{:cleanup}}.  The
  cleanup form is normally run after the setup completes; however the
  cleanup form will not be run if the setup form raises an error.
  When the user asks the runtime system to enter the debugger on an
  error, the cleanup form will not run unless the user explicitly
  enabled a resumption of the test routine from the debugger.  The
  cleanup form will be run in other circumstances, including at a
  user-requested break in testing at either failure or error.
\item A compile deferral supression, \texttt{(:defer-compile
    nil)}~\index{defer-compile@\texttt{:defer-compile}}.  Normally the
  form of a test is evaluated at run time, so that \emph{e.g.}\ macros
  will be expanded according to the most current declaration.  This
  flag will cause test forms to be compiled when the test is first
  loaded.
\end{itemize}
\end{itemize}

%%% Local Variables: 
%%% mode: latex
%%% TeX-master: "manual"
%%% End: 
