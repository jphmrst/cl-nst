\section{Checks}
\fbox{Reword} The
\texttt{def-check}\index{check}\index{def-check@\texttt{def-check}}
form allows a higher-level description of tests.  The basic syntax of
the form is
\begin{tabbing}\hspace*{1cm}
\fbox{UPDATE}
%  (\textbf{de}\=\textbf{f-}\=\textbf{check} (\textit{name}
%  	\texttt{\&key} \= \textit{setup}
%  	     \textit{cleanup} \textit{fixtures})
%  \\ \>\> \textit{criterion}
%  \\ \>\texttt{\&rest} \textit{forms})
\end{tabbing}
The \textit{setup}, \textit{cleanup} and \textit{fixtures} forms are
passed as-is to the underlying \texttt{def-test}, and work the same
way.  The \textit{criterion} is a list or symbol specifying the
properties which should hold for the \textit{forms}.  When there are
no \textit{setup}, \textit{cleanup} or \textit{fixtures} arguments,
the \textit{name} may be given without parentheses.  Likewise, any
criterion consisting of a single symbol, e.g.\ \texttt{(:pass)}, may
be abbreviated as just the symbol without the parentheses, e.g.\
\texttt{:pass}.

\fbox{UPDATE BELOW} Where this list and the list in getting-started
differ, the latter is almost certainly correct.  But this order is
better, re-ored getting-started.

\def\criteriaGroup#1{\subsection{#1}}
\def\endcriteriaGroup{}
\def\criteriaDoc#1#2#3#4#5#6#7{%
\subsubsection{The \texttt{#2} check} 
#4\index{#1@\texttt{#2}}
\\ Syntax: \texttt{#3}
#6{#7}}
\def\noEx#1{}
\def\singleEx#1{\\Example: \texttt{#1}}
\def\multiEx#1{}
\def\tabbingEx#1{\\Example: \begin{tabbing}#1\end{tabbing}}

\section{Defining test criteria}
The criteria used in test forms decide whether, when and how to use
the forms under test and the forms and subcriteria provided to each
test criterion.  Criteria receive their arguments as forms, and may
examine them as forms with or without evaluation, as the particular
criterion requires.  NST provides three mechanisms for defining new
criteria.
\begin{itemize}
\item Defining a criterion by specifying how it should be rewritten to
  another criterion.  This mechanism is both the simplest and the most
  limited in the manipulations it can define.  The
  \texttt{def-criterion-alias} macro provides this mechanism, which we
  discuss in Section~\ref{sec:def-criterion-alias}.
\item Defining a criterion with call-by-value semantics for the values
  under test, specifying how it assesses the results of evaluating the
  forms under test.  The \texttt{def-criterion} macro provides this
  mechanism, which we discuss in Section~\ref{sec:def-criterion}.
\item Defining a criterion receiving the original, unmanipulated forms
  provided as criterion arguments and forms under test.  The
  \texttt{def-criterion-unevaluated} macro provides this mechanism,
  which we discuss in Section~\ref{sec:def-criterion-unevaluated}.
\end{itemize}
The first mechanism is essentially a variation of \texttt{defmacro}.
Under both of the latter two mechanisms, the criteria definition is
made as Lisp code calculating a \emph{test report}.

The functions and macros for defining new criteria are exported from
package \texttt{nst-criteria-api}.

\subsection{Aliases over criteria}
\label{sec:def-criterion-alias}
The simplest mechanism for defining a new criterion involves simply
defining one criterion to rewrite as another using
\texttt{def-criterion-alias}:\index{def-criterion-alias@\texttt{def-criterion-alias}}
%
{\ttfamily\begin{tabbing}
\textrm{Syntax: 
}(de\=f-criterion-alias (name \&rest args)
\\ \> [ documentation ]
\\ \> expansion)
\end{tabbing}}
The body of the expansion should be a Lisp form which, when evaluated,
returns an S-expression quoting the new criterion which the rewrite
should produce.  The \texttt{args} are passed as far Lisp macros: they
are not evaluated and are most typically comma-inserted into a
backquoted result.  For example:
\\ \texttt{(def-criterion-alias (:forms-eq) `(:predicate eq))}
\\ \texttt{(def-criterion-alias (:symbol name) `(:eq ',name))}

\subsection{Reporting forms}
\label{sec:criteria-forms-report}
The other two criteria-defining mechanisms define the expansion of a
criterion into Lisp.  For both of these mechanisms, this Lisp code is
expected to return a test report.  NST provides three functions for
building test reports:
\begin{itemize}
\item
  \input{../gen/make-success-report_function_latex-style}
  \index{make-success-report@\texttt{make-success-report}}
\item
  \input{../gen/make-failure-report_function_latex-style}
  \index{make-failure-report@\texttt{make-failure-report}}
  \index{failure@\texttt{failure}}
\item
  \input{../gen/make-warning-report_function_latex-style}
  \index{make-warning-report@\texttt{make-warning-report}}
  \index{warning@\texttt{warning}}
\end{itemize}

\subsection{Defining criteria over evaluated values}
\label{sec:def-criterion}
\input{../gen/def-criterion_compiler-macro_latex-style}%

\subsection{Processing subcriteria on values}
\label{sec:subcriteria-values}
Since the arguments to the criterion itself (as opposed to the tested
forms) are passed unevaluated as for macro arguments, they can contain
\emph{subcriteria} which can be incorporated into the main criterion's
assessment.%
  \index{check-subcriterion-on-value@\texttt{check-subcriterion-on-value}}

{\ttfamily\begin{tabbing}
\textrm{Syntax: }(check-subcriterion-on-value CRITERION EXPR)
\end{tabbing}}
% The \texttt{continue-check} function converts subcriteria into quoted
% Lisp:
% \begin{verbatim}
%   (continue-check criterion-form values-forms)
% \end{verbatim}
% The \texttt{values-forms} argument should be a quoted Lisp expression
% which, when evaluated, returns a list of values to be assessed by the
% subcriterion.
% 
% In fact, \texttt{def-form-criterion} and \texttt{def-values-criterion}
% expand to method definitions which are used in \texttt{continue-check}.

\subsection{General criteria definitions}
\label{sec:def-criterion-unevaluated}
\index{def-criterion-unevaluated@\texttt{def-criterion-unevaluated}}
{\ttfamily\begin{tabbing}
\textrm{Syntax: }(de\=f-criterion-unevaluated (\=name
\\ \>            \>criterion-args-lambda-list)
\\ \>            \>form-argument)
\\ \> FORM
\\ \> FORM
\\ \> ~$\vdots$
\\ \> FORM)
\end{tabbing}}%
As under \texttt{def-criterion}, the body of these criteria
definitions receive the forms provided as the actual parameters of the
criterion itself, and should return a test result report.  However,
these criteria receive the unevaluated forms under test, deciding when
and whether to evaluate them.
% 
% Examples:
% \begin{verbatim}
% (def-form-criterion (:apply (transform criterion) forms)
%   (continue-check criterion
% 		  `(multiple-value-list (apply #',transform ,forms))))
% 
% (def-form-criterion (:not (subcriterion) exprs-form)
%   (let ((subcheck (gensym)))
%     `(let ((,subcheck ,(continue-check subcriterion exprs-form)))
%        (cond
%         ((check-result-errors ,subcheck)
%          ,subcheck)
%         ((check-result-failures ,subcheck)
%          (check-result :info (check-result-info ,subcheck)))
%         (t
%          (make-failure-report :format "Expected failure from ~s"
%                        :args '(,subcriterion)))))))
% \end{verbatim}

\subsection{Processing subcriteria on the unevaluated form}
\label{sec:subcriteria-form}
{\ttfamily\begin{tabbing}
\textrm{Syntax: }(check-subcriterion-on-form CRITERION FORM)
\end{tabbing}}%
  \index{check-subcriterion-on-form@\texttt{check-subcriterion-on-form}}

\subsection{Older criteria-defining macros}
\label{sec:def-values-criterion}
\label{sec:def-form-criterion}
The \texttt{def-values-criterion} and \texttt{def-form-criterion}
macros are deprecated as of NST 1.3.0, and will be make-emoved-report at some
point.  Code using \texttt{def-values- criterion} should continue to
work as before.  \emph{However, code using \texttt{def-form-
    criterion} in any but the simplest ways is very likely to fail.}
%
In NST 1.3 criteria are translated into method definitions, whereas in
earlier versions criteria guided the macro expansion of tests.
Unfortunately, the nature of \texttt{def-form-criterion} declarations
eludes translation into the new scheme.%
  \index{def-values-criterion@\texttt{def-values-criterion}}%
  \index{def-form-criterion@\texttt{def-form-criterion}}

%%% Local Variables: 
%%% mode: latex
%%% TeX-master: "manual"
%%% TeX-PDF-mode: t
%%% End: 


\subsection{Compound forms}

\subsubsection{The \texttt{:apply} check}
This check applies some transforming function to a form, and applies
another check to the result.\index{apply@\texttt{:apply}}
\\ Syntax: \texttt{(:apply \textit{function} \textit{subcriterion})}
\\ Example: \texttt{(def-check applycheck (:apply cadr (:eql 10)) '(0 10 20))}

\subsubsection{The \texttt{:check-err} check}
This test verifies that evaluating a check throws an
error.\index{err@\texttt{:check-err}} The difference between
\texttt{:check-err} and \texttt{:err} is that the latter deals only
with evaluation of a form, whereas \texttt{:check-err} is more about
the unit testing process.  This form is mostly useful for temporarily
disregarding certain checks until some later fix, when they
\emph{won't} throw an error.
%
\\ Syntax: \texttt{(:check-err \textit{criterion})}
\\ Example: \texttt{(def-check check-err1 (:check-err :forms-eq) 'asdfgh (error "this should be caught"))}

\subsubsection{The \texttt{:progn} check}
This check evaluate a list of Lisp forms before evaluating the
remaining check criteria.
\\ Syntax: (\texttt{:progn} \textit{form} \ldots\ \textit{form} \textit{criterion})
\\ Example: Assuming that \texttt{zz} is defined as some variable,
\\ \hspace*{2em}\texttt{(def-check form1 (:progn (setf zz 3) (:eql 3)) zz)}

\subsubsection{The \texttt{:proj} check}
This check projects certain values from the current stack of forms
(numbering from 0) for another criterion.  This check is useful for
composing new checks as aliases.
\\ Syntax: \texttt{(:proj (\textit{num} \ldots\  \textit{num}) \textit{criterion})}
\\ Example: \texttt{(def-check proj-1 (:proj (0 2) :forms-eq) 'a 3 (car '(a b)))}

\subsubsection{The \texttt{:with} check}
\fbox{UNIMPLEMENTED --- is this still needed?}
\par
This check assumes that the next item is a list, whose contents are
expanded into the methods-and-forms.\index{with@\texttt{:with}} This
check is especially useful with the \texttt{def-check-cri\-ter\-ion}
command below.
\\ Syntax: \texttt{(:with \emph{criterion})}
\\ Example: The following two \texttt{def-check}s are equivalent.
\begin{verbatim}
  (def-check seq1
      (:seq (:predicate symbolp) (:eql 1) (:symbol d))
      '(a 1 d))
  (def-check with-seq1
      (:with (:seq (:predicate symbolp) (:eql 1) (:symbol d)))
      '(a 1 d))
\end{verbatim}

\subsection{Checks on lists}

\subsubsection{The \texttt{:each} check}
This check verifies that every member of the list to which a form
evaluates satisfies some other check.\index{each@\texttt{:each}}
\\ Syntax: \texttt{(:each \emph{criterion})}
\\ Example: \texttt{(def-check each1 (:each (:symbol a)) '(a a a a a))}

\subsubsection{The \texttt{:seq} check}
This check verifies, for each element $n$ of the $n$-element list to
which the form evaluates, that that element satisfies the respective
check.\index{seq@\texttt{:seq}}
\\ Syntax: \texttt{(:seq \emph{criteria} \ldots\ \emph{criteria})}
{\ttfamily\begin{tabbing}
\textrm{Example:}\ (de\=f-\=check seqcheck
\\ \>\>  (:seq (:predicate symbolp) (:eql 1) (:symbol d))
\\ \>  '(a 1 d))
\end{tabbing}}

\subsubsection{The \texttt{:permute} check}
The permute check verifies that some permutation of the list to which
the form evaluates satisfies some other check.
\index{permute@\texttt{:permute}}
\\ Syntax: \texttt{(:permute \textit{criterion}}
{\ttfamily\begin{tabbing}
\textrm{Examples:}\ \=(def-check permute1 (:permute (:each (:eq 'a))) '(a a))
\\ \> (de\=f-\=check permute2
\\ \> \> \> (:permute (:seq \=(:symbol b)
\\ \> \> \>             \>(:predicate symbolp)
\\ \> \> \>             \>(:predicate numberp)))
\\ \> \>'(1 a b))
\end{tabbing}}

\subsection{Checks on vectors}

\subsubsection{The \texttt{:across} check}
Like the \texttt{seq} check, but for vectors rather than
lists.\index{across@\texttt{:across}}
\\ Syntax: \texttt{(:across \emph{criteria} \ldots\ \emph{criteria})}
{\ttfamily\begin{tabbing}
\textrm{Example:}\ (de\=f-\=check across1
\\ \>\>  (:across (:predicate symbolp) (:eql 1))
\\ \>  (vector 'a 1))
\end{tabbing}}

\subsection{Checks on class instances}

\subsubsection{The \texttt{:slots} check} 
This check specifies checks for some or all of the slots of a class
member.\index{slots@\texttt{:slots}}
\\ Syntax: \texttt{(:slots  (\textit{slot} \textit{criterion})  \ldots\ (\textit{slot} \textit{criterion}))}
{\ttfamily\begin{tabbing}
\textrm{Example:}\ \= (defc\=lass classcheck ()
\\ \> \> (\=(s1 :initarg :s1 :reader get-s1)
\\ \> \> \> (s2 :initarg :s2)
\\ \> \> \> (s3 :initarg :s3)))
\\ \> (def-check \=slot1
\\ \> \>  (:slots \=(s1 (:eql 10))
\\ \> \>  \>(s2 (:symbol zz))
\\ \> \>  \>(s3 (:seq \=(:symbol q) (:symbol w)
\\ \> \>  \> \> (:symbol e) (:symbol r))))
\\ \> \>  (m\=ake-instance 'classcheck
\\ \> \> \>  :s1 10 :s2 'zz :s3 '(q w e r)))
\end{tabbing}}

%%% Local Variables: 
%%% mode: latex
%%% TeX-master: "manual"
%%% End: 
