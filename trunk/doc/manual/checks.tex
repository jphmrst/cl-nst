\section{Checks}
\fbox{Reword} The
\texttt{def-check}\index{check}\index{def-check@\texttt{def-check}}
form allows a higher-level description of tests.  The basic syntactic
forms are:
\begin{verbatim}
  (def-test (NAME [ :group GROUP-NAME ]
                  [ :setup FORM ]
                  [ :cleanup FORM ]
                  [ :fixtures (FIXTURE FIXTURE ... FIXTURE) ]
                  [ :documentation STRING ] )
      criterion
    FORM FORM ... FORM)

  (def-test NAME
      criterion
    FORM FORM ... FORM)
\end{verbatim}

The \textit{setup}, \textit{cleanup} and \textit{fixtures} forms are
passed as-is to the underlying \texttt{def-test}, and work the same
way.  The \textit{criterion} is a list or symbol specifying the
properties which should hold for the \textit{forms}.  When there are
no \textit{setup}, \textit{cleanup} or \textit{fixtures} arguments,
the \textit{name} may be given without parentheses.  Likewise, any
criterion consisting of a single symbol, e.g.\ \texttt{(:pass)}, may
be abbreviated as just the symbol without the parentheses, e.g.\
\texttt{:pass}.

\fbox{UPDATE BELOW} Where this list and the list in getting-started
differ, the latter is almost certainly correct.  But this order is
better, re-ored getting-started.

\def\criteriaGroup#1{\subsection{#1}}
\def\endcriteriaGroup{}
\def\criteriaDoc#1#2#3#4#5#6#7#8{%
\subsubsection{The \texttt{#2} check} 
#4\index{#1@\texttt{#2}}
\\ Syntax: \texttt{#3}
#7{#8}#6}
\def\noEx#1{}
\def\singleEx#1{\\Example: \texttt{#1}}
\def\multiEx#1{}
\def\tabbingEx#1{\\Example: \begin{tabbing}#1\end{tabbing}}
\def\noExpl{}
\def\hasExpl#1{\par#1}

%% \criteriaDoc
%%    1 - criteria name without punctuation
%%    2 - criteria name with punctuation
%%    3 - syntax
%%    4 - explanation
%%    5 - applicable to
%%    6 - example type --- \noEx, \singleEx, \multiEx, \tabbingEx
%%    7 - example(s)

%% \kwdCriteriaDoc
%%    1 - criteria name
%%    2 - syntax
%%    3 - explanation
%%    4 - applicable to
%%    5 - example type --- \noEx, \singleEx, \multiEx, \tabbingEx
%%    6 - example(s)

\def\kwdCriteriaDoc#1#2#3#4#5#6{\criteriaDoc{#1}{:#1}{#2}{#3}{#4}{#5}{#6}}

\begin{criteriaGroup}{Basic checks}

\kwdCriteriaDoc{pass}{:pass}
{A trivial test, which always passes.}
{Any.}
{\singleEx}{(def-check passing-test :pass 3 4 "sd")}

\kwdCriteriaDoc{fail}{(:fail FORMAT ARG ...\ ARG)}
{A trivial test, which always fails.  The format string and arguments
should be suitable for the Lisp \texttt{format} function.}
{Any.}
{\singleEx}{(def-check fails (:fail "Expected a \~{}a" "string") 312)}

\end{criteriaGroup}


%%% Local Variables: 
%%% mode: latex
%%% TeX-master: "manual"
%%% End: 
