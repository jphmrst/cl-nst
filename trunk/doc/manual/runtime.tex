\section{The runtime system}
\fbox{The system is now much smaller; update to match the REPL but move any neat idea to Section~\ref{temp-removed}.}

The runtime system provides several operations for scheduling and
running tests, and debugging failing and erring tests.  The operations
are accessible from the
\texttt{run-nst-commands}\index{run-nst-commands@\texttt{run-nst-commands}}
function.  Under Allegro, the top-level alias
\texttt{:nst}\index{nst@\texttt{:nst}} provides a shorthand to this
function; for the sake of brevity we use this shorthand below.

\subsection{Output control}
\subsubsection{The \texttt{:help} operation}
\index{help@\texttt{:help}}%
Shows a brief summary of the runtime system operations.
\\ Syntax: \texttt{:nst :help}

\subsubsection{The \texttt{:verbose} operation}
\index{verbose@\texttt{:verbose}}%
Sets whether the runtime system should show verbose output.
\\ Syntax: \texttt{:nst :verbose FORM}

\subsubsection{The \texttt{:debug} operation}
\index{debug@\texttt{:debug}}%
Sets whether NST debugging messages should be generated.
\\ Syntax: \texttt{:nst :debug BOOL}

\subsubsection{The \texttt{:summarize-scheduled} operation}
\index{summarize-scheduled@\texttt{:summarize-scheduled}}%
If \texttt{BOOL} evaluates to non-null, then the runtime system will
print a summary after running scheduled tests with \texttt{:run},
\texttt{:continue}, etc.
\\ Syntax: \texttt{:nst :summarize-scheduled BOOL}

\subsubsection{The \texttt{:summarize-single} operation}
\index{summarize-single@\texttt{:summarize-single}}%
If \texttt{BOOL} evaluates to non-null, then the runtime system will
print a summary after one-time test runs initiated by
\texttt{:run-test}, \texttt{:run-group}, etc.
\\ Syntax: \texttt{:nst :summarize-single BOOL}

\subsubsection{The \texttt{:dump} operation}
\index{dump@\texttt{:dump}}%
Prints the state of the test system.
\\ Syntax: \texttt{:nst :dump}

\subsection{Marking tests of interest for execution}
\label{nominating-tests}
NST allows tests to be marked for execution by the \texttt{:run}
command.  Tests can be marked by package or group, or as an individual
test.

\subsubsection{The \texttt{:p} operation}
\index{p@\texttt{:p}}%
Indicates that all tests in a particular package should be run.
\\ Syntax: \texttt{:nst :p PACKAGE}

\subsubsection{The \texttt{:g} operation}
\index{g@\texttt{:g}}%
Indicates that all tests in a particular group should be run.
\\ Syntax: \texttt{:nst :g GROUP}

\subsubsection{The \texttt{:t} operation}
\index{t@\texttt{:t}}%
Indicates that the named test should be run.
\\ Syntax: \texttt{:nst :t TEST}

\subsection{Test suite execution}
These commands control execution of tests selected by the commands of
the previous section.

\subsubsection{The \texttt{:run} operation}
\index{run@\texttt{:run}}%
Runs all marked tests.
\\ Syntax: \texttt{:nst :run}

\subsubsection{The \texttt{:continue} operation}
\index{continue@\texttt{:continue}}%
Resumes running tests after an interruption arising from one of the
three switch setting above.
\\ Syntax: \texttt{:nst :continue}

\subsubsection{The \texttt{:retry} operation}
\index{retry@\texttt{:retry}}%
Retries failed or error-raising tests from the last run.
\\ Syntax: \texttt{:nst :retry}

\subsubsection{The \texttt{:blurb} operation}
\index{blurb@\texttt{:blurb}}%
Describes the outcome of the last run of a test.
\\ Syntax: \texttt{:nst :blurb GROUPNAME TESTNAME}

\subsection{Controlling test suite execution behavior}
The following commands control options governing how the runtime
system should behave when a test fails or raises an error.

\subsubsection{The \texttt{:break-on-wrong} operation}
\index{break-on-wrong@\texttt{:break-on-wrong}}%
If \texttt{BOOL} evaluates to non-null, then a failing test will cause
test execution to pause.
\\ Syntax: \texttt{:nst :break-on-wrong BOOL}

\subsubsection{The \texttt{:break-on-error} operation}
\index{break-on-error@\texttt{:break-on-error}}%
If \texttt{BOOL} evaluates to non-null, then any error in a test run
will cause test execution to pause.
\\ Syntax: \texttt{:nst :break-on-error BOOL}

\subsubsection{The \texttt{:debug-on-error} operation}
\index{debug-on-error@\texttt{:debug-on-error}}%
If \texttt{BOOL} evaluates to non-null, then an error in a test run
will pause test execution and enter debug mode.
\\ Syntax: \texttt{:nst :debug-on-error BOOL}

\subsection{Test definition}
\subsubsection{The \texttt{:defer-test-compile} operation}
\index{defer-test-compile@\texttt{:defer-test-compile}}%
Sets whether tests defined subsequently should, by default, defer
compilation of their forms until actually running the test.  This
feature is useful when debugging code involving macros, but changing
this feature in the runtime system can lead to confusion.  It is
surely almost always the right thing to set this flag locally via
def-test-group and def-test.  In fact, this operation may be removed
in a future version of the runtime system.
\\ Syntax: \texttt{:nst :defer-test-compile BOOL}

\subsection{One-off execution}
The following commands allow one-time execution of tests without
altering the settings made by the commands of
Section~\ref{nominating-tests}.

\subsubsection{The \texttt{:run-package} operation}
\index{run-package@\texttt{:run-package}}%
Runs all of the tests in a package.
\\ Syntax: \texttt{:nst :run-package PACKAGE}

\subsubsection{The \texttt{:run-group} operation}
\index{run-group@\texttt{:run-group}}%
Runs all of the tests in a single group.
\\ Syntax: \texttt{:nst :run-group GROUP}

\subsubsection{The \texttt{:run-test} operation}
\index{run-test@\texttt{:run-test}}%
Runs a single test.
\\ Syntax: \texttt{:nst :run-test GROUP TEST}

\subsection{Opening fixtures}
The use of fixtures would make debugging tedious without access to
these named values.  Fortunately, it is straightforward to inject
these names into the runtime system.

\subsubsection{The \texttt{:open} operation}
\index{open@\texttt{:open}}%
Fixtures\index{fixtures!debugging} can be \emph{opened} into the
interactive namespace for debugging with the
\texttt{:nst~:open}\index{open@\texttt{:open}} command:
\\ Syntax: \texttt{:nst :open FIXTURE-NAME}
\\ Example:
\begin{verbatim}
  CL-USER(75): (nst:def-fixtures small-fixture
                  :bindings ((fix-var1 3) (fix-var2 'asdfg)))
  NIL
  CL-USER(76): (boundp 'fix-var1)
  NIL
  CL-USER(77): :nst :open small-fixture
  Opened fixture SMALL-FIXTURE.
  CL-USER(78): fix-var1
  3
  CL-USER(79): 
\end{verbatim}

\subsubsection{The \texttt{:open{\textasteriskcentered}} operation}
\index{open{\textasteriskcentered}@\texttt{:open{\textasteriskcentered}}}%
Multiple fixtures\index{fixtures!debugging} can be \emph{opened} into the
interactive namespace with a single command using the
\texttt{:nst~:open} command:
\\ Syntax: \texttt{:nst :open\* } {\textit{FIXTURE-NAME}{\textasteriskcentered}}


\subsubsection{The \texttt{:open-used} operation}
\index{open-used@\texttt{:open-used}}%
If \texttt{BOOL} evaluates to non-null, then opening a fixture will
always also open the fixtures it uses.  Default is t.
\\ Syntax: \texttt{:nst :open-used BOOL}

\subsubsection{The \texttt{:reopen} operation}
\index{reopen@\texttt{:reopen}}%
If \texttt{BOOL} evaluates to non-null, then fixtures will be
re-opened \emph{e.g.}\ when required multiple times by opening
different fixtures that use them.
\\ Syntax: \texttt{:nst :reopen BOOL}

%%% Local Variables: 
%%% mode: latex
%%% TeX-master: "manual"
%%% End: 
