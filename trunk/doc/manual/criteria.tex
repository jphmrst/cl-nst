
\section{Defining test criteria}
The criteria used in test forms decide whether, when and how to use
the forms under test and the forms and subcriteria provided to each
test criterion.  Criteria receive their arguments as forms, and may
examine them with or without evaluation, as the particular criterion
requires.  NST provides two mechanisms for defining new criteria, and
a number of support functions for use within these definitions.  The
simpler, but more limited, way to define a new criterion is by
specifying how it should be rewritten to another criterion.  The
\texttt{def-criterion-alias} macro provides this mechanism, which we
discuss in Section~\ref{sec:def-criterion-alias}.  The
\texttt{def-criterion} macro provides the more general mechanism for
criteria definition, where Lisp code produces a result report from the
forms under test and criterion's forms and subcriteria.  We discuss
\texttt{def-criterion} in Section~\ref{sec:def-criterion}.  We discuss
the NST API for creating these result reports in
Section~\ref{sec:criteria-forms-report}, and for recursive processing
of subcriteria in Section~\ref{sec:subcriteria}.

The functions and macros for defining new criteria are exported from
package \texttt{nst-criteria-api}.

\subsection{Aliases over criteria}
\label{sec:def-criterion-alias}
\input{../gen/def-criterion-alias_compiler-macro_nst-item-style}

\subsection{Reporting forms}
\label{sec:criteria-forms-report}
NST provides functions both for building test reports, and for adding
information to a report:
\begin{itemize}
\item
  \input{../gen/make-success-report_function_nst-item-style}
  \index{make-success-report@\texttt{make-success-report}}
\item
  \input{../gen/make-failure-report_function_nst-item-style}
  \index{make-failure-report@\texttt{make-failure-report}}
  \index{failure@\texttt{failure}}
\item
  \input{../gen/make-warning-report_function_nst-item-style}
  \index{make-warning-report@\texttt{make-warning-report}}
  \index{warning@\texttt{warning}}
\item
  \input{../gen/make-error-report_function_nst-item-style}
  \index{make-error-report@\texttt{make-error-report}}
  \index{error@\texttt{error}}
\item
  \input{../gen/add-error_function_nst-item-style}
  \index{add-error@\texttt{add-error}}
\item
  \input{../gen/add-failure_function_nst-item-style}
  \index{add-failure@\texttt{add-failure}}
\item
  \input{../gen/add-info_function_nst-item-style}
  \index{add-info@\texttt{add-info}}
\end{itemize}

\subsection{Processing subcriteria}
\label{sec:subcriteria}
The criterion itself can contain \emph{subcriteria} which can be
incorporated into the main criterion's assessment.  NST provides two
functions which trigger testing by a subcriterion, each returning the
check's result report.
\par
\input{../gen/check-criterion-on-value_function_nst-item-style}%
\par
\input{../gen/check-criterion-on-form_function_nst-item-style}%

\subsection{General criteria definitions}
\label{sec:def-criterion}
\input{../gen/def-criterion_compiler-macro_nst-item-style}%

%%% Local Variables: 
%%% mode: latex
%%% TeX-master: "manual"
%%% TeX-PDF-mode: t
%%% End: 
