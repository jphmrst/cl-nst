\section{Defining test criteria}
NST provides three mechanisms for defining new criteria.
\begin{itemize}
\item Defining a criterion by specifying how it should be rewritten to
  another criterion.  This mechanism is the simplest, and is the most
  limited in the manipulations it can define.
\item Defining a criterion by specifying how it assesses the values
  provided to it.
\item Defining a criterion by specifying how it manipulates the
  (possibly unevaluated) forms of a \texttt{def-check}.
\end{itemize}
The first mechanism is essentially a variation of \texttt{defmacro}.
Under both of the latter two mechanisms, the criteria definition is
made as Lisp code calculating a \emph{test report}.

\subsection{Aliases over criteria}
The simplest mechanism for defining a new criterion involves simply
defining one criterion to rewrite as another using
\texttt{def-criterion-alias}:
%
{\ttfamily\begin{tabbing}
\textrm{Syntax: 
}(de\=f-criterion-alias (name \&rest args)
\\ \> [ documentation ]
\\ \> expansion)
\end{tabbing}}
The body of the expansion should be a Lisp form which, when evaluated,
returns an S-expression quoting the new criterion which the rewrite
should produce.  The \texttt{args} are passed as far Lisp macros: they
are not evaluated and are most typically comma-inserted into a
backquoted result.  For example:
\\ \texttt{(def-test-alias (:forms-eq) `(:predicate eq))}
\\ \texttt{(def-test-alias (:symbol name) `(:eq ',name))}

\subsection{Reporting forms}
The other two criteria-defining mechanisms define the expansion of a
criterion into Lisp.  For both of these mechanisms, this Lisp code is
expected to return a test report.  NST provides three functions for
building test reports:
\begin{itemize}
\item\texttt{check-report}
\par\fbox{Fill in details}
\item\texttt{emit-failure}
\par\fbox{Fill in details}
\item\texttt{emit-warning}
\par\fbox{Fill in details}
\end{itemize}

\subsection{Defining criteria for values}
The Lisp body of a \texttt{def-form-criterion} should evaluate to a
quoted Lisp expression calculating a test report.  These criteria
definitions are like functions with two sets of formal parameters:
\begin{itemize}
\item One set correspond to the actual parameters of the criterion
  itself.  The values of these parameters are available to the
  criterion expansion to guide the construction of the resulting Lisp
  expression.
\item The other set corresponds to the evaluation of the form under
  examination by the \texttt{def-test}.  These values are available
  only at run-time, and may appear unquoted within the result
  expression, but are not bound during the evaluation which produces
  the result expression.
\end{itemize}
% 
{\ttfamily\begin{tabbing}
\textrm{Syntax: 
}(de\=f-values-criterion (na\=me \=subcriteria-lambda-list
\\ \> \> \> test-values-lambda-list
\\ \> \> \&key \= blurb-format
\\ \> \>        \> full-format
\\ \> \>        \> stack-transformer)
\\ \> expansion)
\end{tabbing}}
%
Every time the new criterion is used in a test, the \texttt{expansion}
will be evaluated in a context which provides values to the
\texttt{subcriteria-lambda-list}.  The resulting expression will be
used as part of the body of a method to a NST-internal generic
function which corresponds to the \texttt{def-test} itself.  This
method, in code enclosing the resulting expression, will provide
values to the \texttt{test-values-lambda-list}.

The \texttt{test-values-lambda-list} will most often contain only a
single symbol.  Declarations regarding the names in the
\texttt{test-values-lambda-list} are OK, but should be quoted.

Examples:
\begin{verbatim}
(def-value-check (:pass () (&rest chk))
  `(declare (ignorable chk))
  `(make-check-result))

(def-value-check (:eq (eq-form) (check-form))
  `(if (eq ,eq-form check-form)
     (make-check-result)
     (emit-failure :format "Not eq to ~s" :args '(,eq-form))))
\end{verbatim}

\subsection{Defining criteria for a values-returning form}
As for \texttt{def-values-criterion}, the body of a
\texttt{def-form-criterion} should specify a quoted Lisp expression
calculating a test report.
{\ttfamily\begin{tabbing}
\textrm{Syntax: 
}(de\=f-form-criterion (na\=me \=subcriteria-lambda-list
\\ \>\> \> test-forms-formal-parameter
\\ \>\> \&key \=stack-transformer
\\ \>\> \> blurb-format
\\ \>\> \> full-format)
\\ \> check-forms)
\end{tabbing}}
Comma-prefix all the names.  \texttt{test-forms-formal-parameter} must
be just a symbol.  Calls out to \texttt{continue-check} are useful;
its second argument must be a Lisp expression which will evaluate to a
list of values.
\begin{verbatim}
(def-control-check (:apply (transform criterion) forms)
  (continue-check criterion
		  `(multiple-value-list (apply #',transform ,forms))))
\end{verbatim}
\par\fbox{Fill in details}

%%% Local Variables: 
%%% mode: latex
%%% TeX-master: "manual"
%%% End: 
