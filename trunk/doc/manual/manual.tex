%% File manual.tex
%%
%% NST by John Maraist, based on RRT by Robert Goldman.
%%
%% NST is Copyright (c) 2006-2009 Smart Information Flow Technologies.
%% RRT is Copyright (c) 2005 Robert Goldman, released under the LGPL,
%% and the Lisp-specific preamble to that license.

%%% Local Variables: 
%%% mode: latex
%%% TeX-master: t
%%% End: 

\documentclass[draft]{article}
\usepackage{makeidx}
\usepackage[pdftex]{hyperref}
\title{NST 1.1 users' manual}
\author{John Maraist\thanks{Smart Information Flow Technologies, 211 N.\ First St.\ Suite 300, Minneapolis, MN 55401; \textsl{jmaraist} at \textsl{sift.info}.}}
\setcounter{tocdepth}{2}
\makeindex

\begin{document}
\maketitle
\thispagestyle{empty}
\tableofcontents
\parskip 0.6em
\parindent 0pt

\section*{Overview}
This document is the manual and users' guide to the NST test
framework.  NST is a unit test system which provides support for test
fixture data, stateful setup and cleanup of tests, grouping of tests,
and (we think!) a useful runtime interface.  Suggestions and comments
are welcome; a list of known bugs and infelicities concludes this
document.  The file \texttt{nst-nst.lisp} in this distribution holds
the NST tests for NST, and contains many examples (some of which we
have adapted to this document).

\section{Fixtures}
\label{fixtures}
\input{gen/def-fixtures_compiler-macro_latex-style}

\input{gen/with-fixtures_compiler-macro_latex-style}


%%% Local Variables: 
%%% mode: latex
%%% TeX-master: "manual"
%%% TeX-PDF-mode: t
%%% End: 


\section{Test groups}
\label{sec:groups}
\input{gen/def-test-group_compiler-macro_latex-style}

%%% Local Variables: 
%%% mode: latex
%%% TeX-master: "manual"
%%% TeX-PDF-mode: t
%%% End: 

\section{Tests}
\fbox{Reword} The
\texttt{def-check}\index{check}\index{def-check@\texttt{def-check}}
form allows a higher-level description of tests.  The basic syntax of
the form is
\begin{tabbing}\hspace*{1cm}
\fbox{UPDATE}
%  (\textbf{de}\=\textbf{f-}\=\textbf{check} (\textit{name}
%  	\texttt{\&key} \= \textit{setup}
%  	     \textit{cleanup} \textit{fixtures})
%  \\ \>\> \textit{criterion}
%  \\ \>\texttt{\&rest} \textit{forms})
\end{tabbing}
The \textit{setup}, \textit{cleanup} and \textit{fixtures} forms are
passed as-is to the underlying \texttt{def-test}, and work the same
way.  The \textit{criterion} is a list or symbol specifying the
properties which should hold for the \textit{forms}.  When there are
no \textit{setup}, \textit{cleanup} or \textit{fixtures} arguments,
the \textit{name} may be given without parentheses.  Likewise, any
criterion consisting of a single symbol, e.g.\ \texttt{(:pass)}, may
be abbreviated as just the symbol without the parentheses, e.g.\
\texttt{:pass}.

\fbox{UPDATE BELOW} Where this list and the list in getting-started
differ, the latter is almost certainly correct.  But this order is
better, re-ored getting-started.
\subsection{Basic checks}
\subsubsection{The \texttt{:pass} check} 
This check always passes.\index{pass@\texttt{:pass}}
\\ Syntax: \texttt{(:pass)}
\\ Example: \texttt{(def-check passing-test () (:pass) 3 4 "sd")}

\subsubsection{The \texttt{:fail} check} 
This check always fails.\index{fail@\texttt{:fail}}
\\ Syntax: \texttt{(:fail)}
\\ Example: \texttt{(def-check failing-test () (:fail) 312)}

\subsubsection{The \texttt{:warn} check} 
This check is considered not to fail, but associates a warning message
with the test.\index{warn@\texttt{:warn}}
\\ Syntax: \texttt{(:warn \emph{format} \&rest \emph{args})}
\\ Example:
\begin{tabbing}
\texttt{(de}\=\texttt{f-}\=\texttt{check warn-me ()}
\\ \>\>\texttt{(:warn "\~{}d is not a perfect square" 5)}
\\ \>\texttt{"asdf")}
\end{tabbing}

\subsubsection{The \texttt{:eq} check} 
This test verifies that the form to be tested evaluates to a value
\texttt{eq}-identified with the value of some designated
form.\index{eq@\texttt{:eq}}
\\ Syntax: (\texttt{:eq} \textit{correct-form})
\\ Example: \texttt{(def-check eq1 () (:eq 'b) (cadr '(a b c)))}

\subsubsection{The \texttt{:symbol} check} 
This test verifies that a form evaluates to some named
symbol.\index{symbol@\texttt{:symbol}}
\\ Syntax: \texttt{(:symbol \emph{symbol-name})}
\\ Passing example: \texttt{(def-check sym1  () (:symbol a) (car '(a b c)))}
\\ Failing example: \texttt{(def-check sym1x () (:symbol a) (cadr '(a b c)))}

\subsubsection{The \texttt{:eql}, \texttt{:equal} and \texttt{:equalp} check} 
These checks are similar to the \texttt{:eq} check, but using
respectively the \texttt{eql}, \texttt{equal} and \texttt{equalp}
predicates.\index{eql@\texttt{:eql}}\index{equal@\texttt{:equal}}\index{equalp@\texttt{:equalp}}
\\ Syntax: (\texttt{:eql} \textit{correct-form})
\\ Example: \texttt{(def-check eql1 (:eql 2) (cadr '(1 2 3)))}

\subsubsection{The \texttt{:forms-eq} check} 
This test verifies that two forms evaluate to \texttt{eq}-identified
values.\index{forms-eq@\texttt{:forms-eq}}
\\ Syntax: (\texttt{:forms-eq})
{\ttfamily\begin{tabbing}
\textrm{Example:}\ (de\=f-check eqforms1 :forms-eq
\\ \>  (cadr '(a b c)) (caddr '(a c b)))
\end{tabbing}}

\subsubsection{The \texttt{:forms-eql} check} 
This test verifies that two forms evaluate to \texttt{eql}-identified
values.\index{forms-eql@\texttt{:forms-eql}}
\\ Syntax: \texttt{:forms-eql}
{\ttfamily\begin{tabbing}
\textrm{Example:}\ (de\=f-check eqlforms1 :forms-eq
\\ \>  (cadr '(a 3 c)) (caddr '(a c 3)))
\end{tabbing}}

\subsubsection{The \texttt{:predicate} check} 
This test applies some predicate function to a form, and verifies that
the result is non-null.\index{predicate@\texttt{:predicate}}
\\ Syntax: \texttt{(:predicate \emph{function})}
\\ Examples:
\\ $~~\bullet$ \texttt{(def-check pred1 (:predicate numberp) 3)}
\\ $~~\bullet$ \texttt{(def-check pred2 (:predicate eql) (+ 1 2) 3)}
\\
Note that the predicate is applied to \emph{all} of the forms
following the criterion.

\subsubsection{The \texttt{:err} check} 
This test verifies that a form throws an error.  Note that this
criterion should only be used as the top-level criterion in a
\texttt{def-check}; use within one of the compound forms below may
have unexpected results.\index{err@\texttt{:err}}
\\ Syntax: (\texttt{:err})
\\ Example: \texttt{(def-check err1 (:err) (error "this should be caught"))}

\subsubsection{The \texttt{:perf} check}
This test sets minimum performance requirements for evaluation of a
form  Note that this form should only be
used as the top-level criterion in a \texttt{def-check};
use within one of the compound forms below may have
unexpected results.\index{err@\texttt{:perf}}
\\ Syntax: (\texttt{:perf} [ \texttt{:min} $|$ \texttt{:sec} $|$ \texttt{:ms} ] \textit{num})
\\Examples: \texttt{(def-check perf1 (:perf :min 2) (ack 3 5))}

% \subsection{Use of fixtures}
% 
% Fixtures can be used on a per-test basis with \texttt{def-check} just
% as with \texttt{def-test}.  The syntax is the same, for example,
% \begin{verbatim}
% \end{verbatim}

\subsection{Compound forms}

\subsubsection{The \texttt{:not} check} 
This check verifies that some other check fails (but does not throw an
error).\index{not@\texttt{:not}}
\\ Syntax: \texttt{(:not \emph{criterion})}
\\ Example: \texttt{(def-check not1 (:not (:symbol b)) 'a)}

\subsubsection{The \texttt{:all} check} 
This criterion brings several other criteria under one check, and
verifies that they all pass.\index{all@\texttt{:all}}
\\ Syntax: \texttt{(:all \&rest \emph{criteria})}
\\ Example:
\begin{tabbing}
\texttt{(de}\=\texttt{f-}\=\texttt{check not1 ()}
\\ \>\>\texttt{(:all }\=\texttt{(:predicate even-p)}
\\ \>\>\>\texttt{(:predicate prime-p))}
\\ \>\texttt{2)}
\end{tabbing}

\subsubsection{The \texttt{:any} check} 
This criterion brings several other criteria under one check, and
verifies that at least one passes.\index{all@\texttt{:all}}
\\ Syntax: \texttt{(:any \&rest \emph{criteria})}
\\ Example:
\begin{tabbing}
\texttt{(de}\=\texttt{f-}\=\texttt{check not1 ()}
\\ \>\>\texttt{(:anl }\=\texttt{(:predicate even-p)}
\\ \>\>\>\texttt{(:predicate prime-p))}
\\ \>\texttt{5)}
\end{tabbing}

\subsubsection{The \texttt{:apply} check}
This check applies some transforming function to a form, and applies
another check to the result.\index{apply@\texttt{:apply}}
\\ Syntax: \texttt{(:apply \textit{function} \textit{subcriterion})}
\\ Example: \texttt{(def-check applycheck (:apply cadr (:eql 10)) '(0 10 20))}

\subsubsection{The \texttt{:check-err} check}
This test verifies that evaluating a check throws an
error.\index{err@\texttt{:check-err}} The difference between
\texttt{:check-err} and \texttt{:err} is that the latter deals only
with evaluation of a form, whereas \texttt{:check-err} is more about
the unit testing process.  This form is mostly useful for temporarily
disregarding certain checks until some later fix, when they
\emph{won't} throw an error.
%
\\ Syntax: \texttt{(:check-err \textit{criterion})}
\\ Example: \texttt{(def-check check-err1 (:check-err :forms-eq) 'asdfgh (error "this should be caught"))}

\subsubsection{The \texttt{:progn} check}
This check evaluate a list of Lisp forms before evaluating the
remaining check criteria.
\\ Syntax: (\texttt{:progn} \textit{form} \ldots\ \textit{form} \textit{criterion})
\\ Example: Assuming that \texttt{zz} is defined as some variable,
\\ \hspace*{2em}\texttt{(def-check form1 (:progn (setf zz 3) (:eql 3)) zz)}

\subsubsection{The \texttt{:proj} check}
This check projects certain values from the current stack of forms
(numbering from 0) for another criterion.  This check is useful for
composing new checks as aliases.
\\ Syntax: \texttt{(:proj (\textit{num} \ldots\  \textit{num}) \textit{criterion})}
\\ Example: \texttt{(def-check proj-1 (:proj (0 2) :forms-eq) 'a 3 (car '(a b)))}

\subsubsection{The \texttt{:with} check}
\fbox{UNIMPLEMENTED --- is this still needed?}
\par
This check assumes that the next item is a list, whose contents are
expanded into the methods-and-forms.\index{with@\texttt{:with}} This
check is especially useful with the \texttt{def-check-cri\-ter\-ion}
command below.
\\ Syntax: \texttt{(:with \emph{criterion})}
\\ Example: The following two \texttt{def-check}s are equivalent.
\begin{verbatim}
  (def-check seq1
      (:seq (:predicate symbolp) (:eql 1) (:symbol d))
      '(a 1 d))
  (def-check with-seq1
      (:with (:seq (:predicate symbolp) (:eql 1) (:symbol d)))
      '(a 1 d))
\end{verbatim}

\subsection{Checks on lists}

\subsubsection{The \texttt{:each} check}
This check verifies that every member of the list to which a form
evaluates satisfies some other check.\index{each@\texttt{:each}}
\\ Syntax: \texttt{(:each \emph{criterion})}
\\ Example: \texttt{(def-check each1 (:each (:symbol a)) '(a a a a a))}

\subsubsection{The \texttt{:seq} check}
This check verifies, for each element $n$ of the $n$-element list to
which the form evaluates, that that element satisfies the respective
check.\index{seq@\texttt{:seq}}
\\ Syntax: \texttt{(:seq \emph{criteria} \ldots\ \emph{criteria})}
{\ttfamily\begin{tabbing}
\textrm{Example:}\ (de\=f-\=check seqcheck
\\ \>\>  (:seq (:predicate symbolp) (:eql 1) (:symbol d))
\\ \>  '(a 1 d))
\end{tabbing}}

\subsubsection{The \texttt{:permute} check}
The permute check verifies that some permutation of the list to which
the form evaluates satisfies some other check.
\index{permute@\texttt{:permute}}
\\ Syntax: \texttt{(:permute \textit{criterion}}
{\ttfamily\begin{tabbing}
\textrm{Examples:}\ \=(def-check permute1 (:permute (:each (:eq 'a))) '(a a))
\\ \> (de\=f-\=check permute2
\\ \> \> \> (:permute (:seq \=(:symbol b)
\\ \> \> \>             \>(:predicate symbolp)
\\ \> \> \>             \>(:predicate numberp)))
\\ \> \>'(1 a b))
\end{tabbing}}

\subsection{Checks on vectors}

\subsubsection{The \texttt{:across} check}
Like the \texttt{seq} check, but for vectors rather than
lists.\index{across@\texttt{:across}}
\\ Syntax: \texttt{(:across \emph{criteria} \ldots\ \emph{criteria})}
{\ttfamily\begin{tabbing}
\textrm{Example:}\ (de\=f-\=check across1
\\ \>\>  (:across (:predicate symbolp) (:eql 1))
\\ \>  (vector 'a 1))
\end{tabbing}}

\subsection{Checks on class instances}

\subsubsection{The \texttt{:slots} check} 
This check specifies checks for some or all of the slots of a class
member.\index{slots@\texttt{:slots}}
\\ Syntax: \texttt{(:slots  (\textit{slot} \textit{criterion})  \ldots\ (\textit{slot} \textit{criterion}))}
{\ttfamily\begin{tabbing}
\textrm{Example:}\ \= (defc\=lass classcheck ()
\\ \> \> (\=(s1 :initarg :s1 :reader get-s1)
\\ \> \> \> (s2 :initarg :s2)
\\ \> \> \> (s3 :initarg :s3)))
\\ \> (def-check \=slot1
\\ \> \>  (:slots \=(s1 (:eql 10))
\\ \> \>  \>(s2 (:symbol zz))
\\ \> \>  \>(s3 (:seq \=(:symbol q) (:symbol w)
\\ \> \>  \> \> (:symbol e) (:symbol r))))
\\ \> \>  (m\=ake-instance 'classcheck
\\ \> \> \>  :s1 10 :s2 'zz :s3 '(q w e r)))
\end{tabbing}}

%%% Local Variables: 
%%% mode: latex
%%% TeX-master: "manual"
%%% End: 


\section{Defining test criteria}
The criteria used in test forms decide whether, when and how to use
the forms under test and the forms and subcriteria provided to each
test criterion.  Criteria receive their arguments as forms, and may
examine them as forms with or without evaluation, as the particular
criterion requires.  NST provides three mechanisms for defining new
criteria.
\begin{itemize}
\item Defining a criterion by specifying how it should be rewritten to
  another criterion.  This mechanism is both the simplest and the most
  limited in the manipulations it can define.  The
  \texttt{def-criterion-alias} macro provides this mechanism, which we
  discuss in Section~\ref{sec:def-criterion-alias}.
\item Defining a criterion with call-by-value semantics for the values
  under test, specifying how it assesses the results of evaluating the
  forms under test.  The \texttt{def-criterion} macro provides this
  mechanism, which we discuss in Section~\ref{sec:def-criterion}.
\item Defining a criterion receiving the original, unmanipulated forms
  provided as criterion arguments and forms under test.  The
  \texttt{def-criterion-unevaluated} macro provides this mechanism,
  which we discuss in Section~\ref{sec:def-criterion-unevaluated}.
\end{itemize}
The first mechanism is essentially a variation of \texttt{defmacro}.
Under both of the latter two mechanisms, the criteria definition is
made as Lisp code calculating a \emph{test report}.

The functions and macros for defining new criteria are exported from
package \texttt{nst-criteria-api}.

\subsection{Aliases over criteria}
\label{sec:def-criterion-alias}
The simplest mechanism for defining a new criterion involves simply
defining one criterion to rewrite as another using
\texttt{def-criterion-alias}:\index{def-criterion-alias@\texttt{def-criterion-alias}}
%
{\ttfamily\begin{tabbing}
\textrm{Syntax: 
}(de\=f-criterion-alias (name \&rest args)
\\ \> [ documentation ]
\\ \> expansion)
\end{tabbing}}
The body of the expansion should be a Lisp form which, when evaluated,
returns an S-expression quoting the new criterion which the rewrite
should produce.  The \texttt{args} are passed as far Lisp macros: they
are not evaluated and are most typically comma-inserted into a
backquoted result.  For example:
\\ \texttt{(def-criterion-alias (:forms-eq) `(:predicate eq))}
\\ \texttt{(def-criterion-alias (:symbol name) `(:eq ',name))}

\subsection{Reporting forms}
\label{sec:criteria-forms-report}
The other two criteria-defining mechanisms define the expansion of a
criterion into Lisp.  For both of these mechanisms, this Lisp code is
expected to return a test report.  NST provides three functions for
building test reports:
\begin{itemize}
\item
  \input{../gen/make-success-report_function_latex-style}
  \index{make-success-report@\texttt{make-success-report}}
\item
  \input{../gen/make-failure-report_function_latex-style}
  \index{make-failure-report@\texttt{make-failure-report}}
  \index{failure@\texttt{failure}}
\item
  \input{../gen/make-warning-report_function_latex-style}
  \index{make-warning-report@\texttt{make-warning-report}}
  \index{warning@\texttt{warning}}
\end{itemize}

\subsection{Defining criteria over evaluated values}
\label{sec:def-criterion}
\input{../gen/def-criterion_compiler-macro_latex-style}%

\subsection{Processing subcriteria on values}
\label{sec:subcriteria-values}
Since the arguments to the criterion itself (as opposed to the tested
forms) are passed unevaluated as for macro arguments, they can contain
\emph{subcriteria} which can be incorporated into the main criterion's
assessment.%
  \index{check-subcriterion-on-value@\texttt{check-subcriterion-on-value}}

{\ttfamily\begin{tabbing}
\textrm{Syntax: }(check-subcriterion-on-value CRITERION EXPR)
\end{tabbing}}
% The \texttt{continue-check} function converts subcriteria into quoted
% Lisp:
% \begin{verbatim}
%   (continue-check criterion-form values-forms)
% \end{verbatim}
% The \texttt{values-forms} argument should be a quoted Lisp expression
% which, when evaluated, returns a list of values to be assessed by the
% subcriterion.
% 
% In fact, \texttt{def-form-criterion} and \texttt{def-values-criterion}
% expand to method definitions which are used in \texttt{continue-check}.

\subsection{General criteria definitions}
\label{sec:def-criterion-unevaluated}
\index{def-criterion-unevaluated@\texttt{def-criterion-unevaluated}}
{\ttfamily\begin{tabbing}
\textrm{Syntax: }(de\=f-criterion-unevaluated (\=name
\\ \>            \>criterion-args-lambda-list)
\\ \>            \>form-argument)
\\ \> FORM
\\ \> FORM
\\ \> ~$\vdots$
\\ \> FORM)
\end{tabbing}}%
As under \texttt{def-criterion}, the body of these criteria
definitions receive the forms provided as the actual parameters of the
criterion itself, and should return a test result report.  However,
these criteria receive the unevaluated forms under test, deciding when
and whether to evaluate them.
% 
% Examples:
% \begin{verbatim}
% (def-form-criterion (:apply (transform criterion) forms)
%   (continue-check criterion
% 		  `(multiple-value-list (apply #',transform ,forms))))
% 
% (def-form-criterion (:not (subcriterion) exprs-form)
%   (let ((subcheck (gensym)))
%     `(let ((,subcheck ,(continue-check subcriterion exprs-form)))
%        (cond
%         ((check-result-errors ,subcheck)
%          ,subcheck)
%         ((check-result-failures ,subcheck)
%          (check-result :info (check-result-info ,subcheck)))
%         (t
%          (make-failure-report :format "Expected failure from ~s"
%                        :args '(,subcriterion)))))))
% \end{verbatim}

\subsection{Processing subcriteria on the unevaluated form}
\label{sec:subcriteria-form}
{\ttfamily\begin{tabbing}
\textrm{Syntax: }(check-subcriterion-on-form CRITERION FORM)
\end{tabbing}}%
  \index{check-subcriterion-on-form@\texttt{check-subcriterion-on-form}}

\subsection{Older criteria-defining macros}
\label{sec:def-values-criterion}
\label{sec:def-form-criterion}
The \texttt{def-values-criterion} and \texttt{def-form-criterion}
macros are deprecated as of NST 1.3.0, and will be make-emoved-report at some
point.  Code using \texttt{def-values- criterion} should continue to
work as before.  \emph{However, code using \texttt{def-form-
    criterion} in any but the simplest ways is very likely to fail.}
%
In NST 1.3 criteria are translated into method definitions, whereas in
earlier versions criteria guided the macro expansion of tests.
Unfortunately, the nature of \texttt{def-form-criterion} declarations
eludes translation into the new scheme.%
  \index{def-values-criterion@\texttt{def-values-criterion}}%
  \index{def-form-criterion@\texttt{def-form-criterion}}

%%% Local Variables: 
%%% mode: latex
%%% TeX-master: "manual"
%%% TeX-PDF-mode: t
%%% End: 

\section{Integration with ASDF}
NST's integration with ASDF is a work in progress.  This section
described the current integration, the ways we expect it to change,
and a less-flexible and lower-level, but likely more stable,
alternative integration technique.

\subsection{NST's ASDF systems}
From version~1.2.2, the system \texttt{:asdf-nst} provides two classes
for ASDF system definitions, \texttt{asdf:nst-test-runner} and
\texttt{asdf:nst-test-holder}.

Up to NST~1.2.1 \texttt{:asdf-nst} provided a single class
\texttt{asdf:nst-testable}, and in the future we plan to reunify the
current two classes into a single class again.  However our first
implementation required NST to be loaded even when a system was
\emph{not} being tested, because we had no way to distinguish the
source code associated with testing from production code.  We plan to
solve this problem with a new file type \texttt{nst-file} in a future
version of NST.  This file type would \emph{not} be compiled or loaded
for the \texttt{compile-op} or \texttt{load-op} of the system, only
for its \texttt{test-op}\,.

\subsubsection{Test-running systems}
ASDF systems of the \texttt{asdf:nst-test-runner} class do not
themselves contain NST declarations in their source code, but may
identify other systems which do, and which should be tested as a part
of testing the given system.  These systems also allow local
definitions of NST's configuration for the execution of their tests.

Specify that a system runs NST tests by providing \texttt{:class
  asdf:nst-test-runner} argument to \texttt{asdf:defsystem}.  Use the
\texttt{:nst-systems} argument to name the systems which house the
actual unit tests:
\begin{itemize}
\item\texttt{:nst-systems ({\slshape system} {\slshape system}
    $\cdots$ {\slshape system})}\par Specifies a list of other systems
  which should be tested when testing this system.  These other
  systems do \emph{not} otherwise need to be identified as a
  dependency of this system (nor, for that matter, does \texttt{:nst}
  itself); they will be loaded upon \texttt{test-op} if they are not
  yet present.
\end{itemize}
Another optional argument to an \texttt{nst-test-runner} system
definition is:
\begin{itemize}
\item\texttt{:nst-init ({\slshape arg-list ...\ arg-list})}\par
  Initializing arguments to NST, to be executed after this system is
  loaded.  Each \texttt{arg-list} is passed as the arguments as if to
  a call to the \texttt{nst-cmd} macro.
\item\texttt{:nst-debug-config {\slshape form}}\par NST debugging
  customization for this system.  The \texttt{FORM} Should be an
  expression which, when evaluated, returns a list of keyword
  arguments; note that to give the list itself, it must be explicitly
  quoted, \emph{which is a change of behavior from pre-1.2.2
    versions}.
\item\texttt{:nst-debug-protect ({\slshape symbol ...\ symbol})} \par
  Gives a list of variables whose values should be saved before
  applying any configuration changes from \texttt{:nst-debug-config},
  and restored after testing.
\item\texttt{:nst-push-debug-config {\slshape t-or-nil}}\par If
  non-nil, then when this system is loaded its \texttt{:nst-debug} and
  \texttt{:nst-debug-protect} settings will be used as NST's defaults.
\end{itemize}

\subsubsection{Test-containing systems}
The \texttt{asdf:nst-test-holder} class is a subclass of
\texttt{nst-test-runner} for systems which are not only tested via
NST, but also contains NST tests in their source code.

Specify that a system defines NST tests by providing \texttt{:class
  asdf:nst-test-holder} to \texttt{asdf:defsystem}.  The arguments for
\texttt{asdf:nst-test-runner} may be used for
\texttt{asdf:nst-test-holder}, as well as the following:
\begin{itemize}
\item\texttt{:nst-packages ({\slshape package} {\slshape package}
    $\cdots$ {\slshape package})}\par When the system is tested, all
  groups and tests in the named packages should be run.
\item\texttt{:nst-groups (({\slshape package} {\slshape group})
    $\cdots$ ({\slshape package} {\slshape group}))}\par When the
  system is tested, tests in the named groups should be run.  Naming
  the package separately from the group and test in this argument (and
  in the similar arguments below) allows the group to be named before
  its package is necessarily defined.
\item\texttt{:nst-tests (({\slshape package} {\slshape group}
    {\slshape test}) $\cdots$ ({\slshape package} {\slshape group}
    {\slshape test}))}\par When the system is tested, all the named
  tests should be run.
\end{itemize}
The next three arguments to an \texttt{nst-testable} system are
mutually exclusive, and moreover exclude any of the above group or
\texttt{:nst-systems}\,:
\begin{itemize}
\item\texttt{:nst-package {\slshape package}}\par When the system is
  tested, all groups and tests in the named package should be run.
\item\texttt{:nst-group ({\slshape package} {\slshape group})}\par
  When the system is tested, all tests in the named group should be
  run.
\item\texttt{:nst-test ({\slshape package} {\slshape group} {\slshape
      test})}\par When the system is tested, the given test should be
  run.
\end{itemize}
Figure~\ref{fig:asdf} gives examples of \texttt{nst-testable}
ASDF system definitions.
\begin{figure}
\begin{verbatim}
;; NST and its ASDF interface must be loaded
;; before we can process the defsystem form.
(asdf:oos 'asdf:load-op :asdf-nst)

(defsystem :mnst
    :class nst-test-holder
    :description "The NST test suite's self-test."
    :serial t
    :nst-systems (:masdfnst)
    :nst-groups ((:mnst-simple . g1)
                 (:mnst-simple . g1a)
                 (:mnst-simple . g1a1)
                 (:mnst-simple . core-checks))
    :depends-on (:nst)
    :in-order-to ((test-op (load-op :mnst)))
    :components ((:module "core"
                  :components ((:file "byhand")
                               (:file "builtin-checks")))))
\end{verbatim}
\caption{Definitions of \texttt{nst-testable} ASDF systems.}
\label{fig:asdf}
\end{figure}

\subsection{An alternate ASDF integration technique}
We plan to deprecate and then remove \texttt{asdf:nst-test-holder} and
\texttt{nst-test-runner} once we have implemented a unified
replacement for them.  To avoid the possibility of a bit-rotted test
scheme, the link between a system and its unit tests can be made
explicit by providing methods for ASDF generic functions which make
calls to the NST API.  Specifically:
\begin{itemize}
\item A method of the ASDF \texttt{asdf:perform} generic function
  specialized to the \texttt{asdf:test-op} operation and the system in
  question will be executed to test a system.  So an appropriate
  method definition would begin:
  \begin{verbatim}
  (defmethod asdf:perform ((op asdf:test-op)
                           (sys (eql (asdf:find-system
                                        :SYSTEM-NAME))))
\end{verbatim}
\item NST API functions for running tests are:
\begin{tightlist}
\item\texttt{nst:run-package}
\item\texttt{nst:run-group}
\item\texttt{nst:run-test}
\end{tightlist}
\item The main NST API function for printing the results of testing is
  \texttt{asdf:report-multiple}\,.  In situations where only a single
  package, group or test is associated with a system, one of the
  following function may be more convenient:
  \begin{tightlist}
  \item\texttt{nst:report-package}
  \item\texttt{nst:report-group}
  \item\texttt{nst:report-test}
  \end{tightlist}
\end{itemize}
When providing an explicit \texttt{asdf:perform} method, it is also
necessary to explicitly list system dependencies to NST and to the
other systems which contain the tested system's unit test definitions.


%%% Local Variables: 
%%% mode: latex
%%% TeX-master: "manual"
%%% TeX-PDF-mode: t
%%% End: 

\section{The runtime system}
\fbox{The system is now much smaller; update to match the REPL but move any neat idea to Section~\ref{temp-removed}.}

The runtime system provides several operations for scheduling and
running tests, and debugging failing and erring tests.  The operations
are accessible from the
\texttt{run-nst-commands}\index{run-nst-commands@\texttt{run-nst-commands}}
function.  Under Allegro, the top-level alias
\texttt{:nst}\index{nst@\texttt{:nst}} provides a shorthand to this
function; for the sake of brevity we use this shorthand below.

\subsection{Output control}
\subsubsection{The \texttt{:help} operation}
\index{help@\texttt{:help}}%
Shows a brief summary of the runtime system operations.
\\ Syntax: \texttt{:nst :help}

\subsubsection{The \texttt{:verbose} operation}
\index{verbose@\texttt{:verbose}}%
Sets whether the runtime system should show verbose output.
\\ Syntax: \texttt{:nst :verbose FORM}

\subsubsection{The \texttt{:debug} operation}
\index{debug@\texttt{:debug}}%
Sets whether NST debugging messages should be generated.
\\ Syntax: \texttt{:nst :debug BOOL}

\subsubsection{The \texttt{:dump} operation}
\index{dump@\texttt{:dump}}%
Prints the state of the test system.
\\ Syntax: \texttt{:nst :dump}

\subsection{Test suite execution}
These commands control execution of tests selected by the commands of
the previous section.

\subsubsection{The \texttt{:run} operation}
\index{run@\texttt{:run}}%
Runs all marked tests.
\\ Syntax: \texttt{:nst :run}

\subsubsection{The \texttt{:continue} operation}
\index{continue@\texttt{:continue}}%
Resumes running tests after an interruption arising from one of the
three switch setting above.
\\ Syntax: \texttt{:nst :continue}

\subsubsection{The \texttt{:retry} operation}
\index{retry@\texttt{:retry}}%
Retries failed or error-raising tests from the last run.
\\ Syntax: \texttt{:nst :retry}

\subsubsection{The \texttt{:blurb} operation}
\index{blurb@\texttt{:blurb}}%
Describes the outcome of the last run of a test.
\\ Syntax: \texttt{:nst :blurb GROUPNAME TESTNAME}

\subsection{Controlling test suite execution behavior}
The following commands control options governing how the runtime
system should behave when a test fails or raises an error.

\subsubsection{The \texttt{:break-on-wrong} operation}
\index{break-on-wrong@\texttt{:break-on-wrong}}%
If \texttt{BOOL} evaluates to non-null, then a failing test will cause
test execution to pause.
\\ Syntax: \texttt{:nst :break-on-wrong BOOL}

\subsubsection{The \texttt{:break-on-error} operation}
\index{break-on-error@\texttt{:break-on-error}}%
If \texttt{BOOL} evaluates to non-null, then any error in a test run
will cause test execution to pause.
\\ Syntax: \texttt{:nst :break-on-error BOOL}

\subsubsection{The \texttt{:debug-on-error} operation}
\index{debug-on-error@\texttt{:debug-on-error}}%
If \texttt{BOOL} evaluates to non-null, then an error in a test run
will pause test execution and enter debug mode.
\\ Syntax: \texttt{:nst :debug-on-error BOOL}

\subsection{One-off execution}
The following commands allow one-time execution of tests without
altering the settings made by the commands of
Section~\ref{nominating-tests}.

\subsubsection{The \texttt{:run-package} operation}
\index{run-package@\texttt{:run-package}}%
Runs all of the tests in a package.
\\ Syntax: \texttt{:nst :run-package PACKAGE}

\subsubsection{The \texttt{:run-group} operation}
\index{run-group@\texttt{:run-group}}%
Runs all of the tests in a single group.
\\ Syntax: \texttt{:nst :run-group GROUP}

\subsubsection{The \texttt{:run-test} operation}
\index{run-test@\texttt{:run-test}}%
Runs a single test.
\\ Syntax: \texttt{:nst :run-test GROUP TEST}

\subsection{Opening fixtures}
The use of fixtures would make debugging tedious without access to
these named values.  Fortunately, it is straightforward to inject
these names into the runtime system.

\subsubsection{The \texttt{:open} operation}
\index{open@\texttt{:open}}%
Fixtures\index{fixtures!debugging} can be \emph{opened} into the
interactive namespace for debugging with the
\texttt{:nst~:open}\index{open@\texttt{:open}} command:
\\ Syntax: \texttt{:nst :open FIXTURE-NAME FIXTURE-NAME ... FIXTURE-NAME}
\\ Example:
\begin{verbatim}
  CL-USER(75): (nst:def-fixtures small-fixture
                  :bindings ((fix-var1 3) (fix-var2 'asdfg)))
  NIL
  CL-USER(76): (boundp 'fix-var1)
  NIL
  CL-USER(77): :nst :open small-fixture
  Opened fixture SMALL-FIXTURE.
  CL-USER(78): fix-var1
  3
  CL-USER(79): 
\end{verbatim}

%%% Local Variables: 
%%% mode: latex
%%% TeX-master: "manual"
%%% End: 


%%% Local Variables: 
%%% mode: latex
%%% TeX-master: "manual"
%%% End: 
\section{Known bugs and enhancement ideas}\index{bugs}\index{enhancements}
\fbox{TOTALLY OUT OF DATE}
\begin{enumerate}
\item We do a poor job of printing the conditions that come back from
  erring tests.  Moreover, the
  \texttt{def-check}\index{def-check@\texttt{def-check}} macro should
  be able to provide better feedback about the context of an error
  when failing (the kind of failing that doesn't throw an error), not
  just at the time, but also when
  \texttt{:blurb}ing\index{blurb@\texttt{:blurb}} tests
  after-the-fact.
\item There are still, sometimes, warnings from the use of fixture
  names in other fixtures.
\item We are not yet catching errors arising from
  \texttt{:cleanup}\index{cleanup@\texttt{:cleanup}} blocks.
\item Clobbering of test/group names should be caught by the macros,
  not by the lower-level expansions arising from those methods.
\item In the runtime system, the implementations of
  \texttt{:blurb}\index{blurb@\texttt{:blurb}},
  \texttt{:open-used}\index{open-used@\texttt{:open-used}} and
  \texttt{:reopen}\index{reopen@\texttt{:reopen}} are questionable.
\item Possibly,
  \texttt{run-nst-commands}\index{run-nst-commands@\texttt{run-nst-commands}}
  should be turned into a macro.
\item The def-check form \texttt{:perf} might be extended to provide
  (implementation-dependent) checks on memory limits as well as time
  limits.
\item Some sort of timeout mechanism (again, possibly
  implementation-dependent) could be helpful for measuring correctness
  via termination in more complicated algorithms.
\end{enumerate}

\subsection{Removed features}
\label{temp-removed}
Some features of the NST 0.9 have not yet been reimplemented in NST
1.0.  Tell John if one is urgent for you.

\subsection{Alternate fixture definitions}
\fbox{UPDATE}

The
\texttt{def-capture/restore-fixtures}\index{def-capture/restore-fixtures@\texttt{def-capture/restore-fixtures}}
declaration binds \texttt{nil} to a collection of variables in the
extent of associated test groups.  This form is useful when hiding
some developers' state from tests intended to simulate a
non-development environment.
\begin{center}
\textbf{def-capture/restore-fixtures}
 \textit{name} \textit{variables}
\texttt{\&key} \textit{documentation}
\end{center}

\subsection{Anonymous fixture sets}
\fbox{UPDATE}

It is not necessary to name fixture sets; they may be given
anonymously in any situation where a fixture set name is allowed.  The
syntax of anonymous fixture sets
is:\index{fixtures@\texttt{:fixtures}}\index{fixtures!anonymous}
\begin{center}
\texttt{(:fixtures }\textit{(name form)}$^+$\texttt{ )}
\end{center}
The examples of Section~\ref{tests} show the use of anonymous fixture
sets.

%
\appendix

\printindex
\end{document}
