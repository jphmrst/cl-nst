
\section{Verifying invariants against sampled data}
\label{quickcheck}
The \texttt{:sample}\indexKeyword{sample} criterion provides random
generation of data for validating program properties.  Our approach is
based on Claessen and Hughes's Quickcheck\footnote{Koen Claessen and
  John Hughes, ``QuickCheck: a lightweight tool for random testing of
  Haskell programs,'' from \emph{Proceedings of the International
    Conference on Functional Programming}, 2000.  QuickCheck papers,
  code and other resources are available at
  \textsl{www.cs.chalmers.se/\~{}rjmh/QuickCheck}~.}.

This style of testing is somewhat more complicated than specific tests
on single, bespoke forms.  There are two distinct efforts, which we
address in the next two sections: describing how the sample data is to
be generated, and specifying the test itself.

\subsection{Generating sample data}
\label{sec:quickcheck-data}
Data generation is centered around the generic function
\texttt{arbitrary}.\indexLisp{arbitrary} %
\input{../gen/arbitrary_function_nst-item-style}

NST provides method of \texttt{arbitrary} for many standard Lisp
types, listed in Table~\ref{table:built-in-arbitrary-types}.  Types in
the first column --- the standard numeric types plus the common
supertype \texttt{t} --- are not associated with additional keyword
arguments.\indexLisp{t}\indexLisp{complex}\indexLisp{integer}\indexLisp{ratio}\indexLisp{single-float}
\begin{verbatim}
  (nst:arbitrary t)
  (nst:arbitrary 'complex)
  (nst:arbitrary 'integer)
  (nst:arbitrary 'ratio)
  (nst:arbitrary 'single-float)
\end{verbatim}
Keyword arguments for other NST-provided type specifiers
are as follows:
\begin{table}
\begin{center}
  \texttt{
    \begin{tabular}{lll|ll||l}
      \multicolumn{5}{c}{\textrm{\textbf{Standard Lisp types}}}
          & \multicolumn{1}{c}{\textrm{\textbf{Other types}}}
      \\ \hline
         number  & character & symbol & cons   & hash-table & scalar
      \\ real & string    &        & list   & &
      \\ rational &       &        & vector & &
      \\ integer  &       &        & array  & &
      \\ float &&&&t&
      \\ fixnum &&&& &
      \\ bignum &&&& &
      \\ ratio &&&& &
      \\ \multicolumn{2}{l}{short-float\footnotemark} &&& &
      \\ \multicolumn{2}{l}{single-float} &&& &
      \\ \multicolumn{2}{l}{double-float\addtocounter{footnote}{-1}\footnotemark} &&& &
      \\ \multicolumn{2}{l}{long-float} &&& &
      \\ complex &&&& &
      \\ \cline{1-3} \multicolumn{3}{c}{\textrm{Considered \texttt{scalar}}}
    \end{tabular}}
\end{center}
\caption{NST provides methods of generic function \texttt{arbitrary} generating values of the types in this table.}
\label{table:built-in-arbitrary-types}
\end{table}
\footnotetext{Not available on Allegro Lisp.}

\begin{itemize}
\item Types \texttt{character} and \texttt{string}:\indexLisp{character}\indexLisp{string}
  \begin{itemize}
  \item Argument \texttt{noncontrol}.  Excludes the control characters
    associated with ASCII code 0 through 31.\indexLisp{noncontrol}
  \item Argument \texttt{range}.  Allows the range of characters to be
    restricted to a particular subset:\indexLisp{range}\indexKeyword{standard}
    \begin{center}
      \begin{tabular}{>{\ttfamily:}c|l}
        \multicolumn{1}{c|}{Value} & \multicolumn{1}{c}{Meaning}
        \\ \hline standard & Codes up to 96
        \\ ascii & Codes through 127
        \\ ascii-ext & Codes through 255
      \end{tabular}
    \end{center}
    Omitted or with any other value, characters with any code up to
    \texttt{char-code-limit} can result.  Examples:
\begin{verbatim}
  (nst:arbitrary 'character)
  (nst:arbitrary '(character :noncontrol t
                             :range :standard))
\end{verbatim}
  \end{itemize}
  
\item Type \texttt{symbol}:\indexLisp{symbol}\indexKeyword{existing}\indexKeyword{exported}\indexKeyword{package}\indexKeyword{nonnull}\indexKeyword{gensym}
  \begin{itemize}
  \item Argument \texttt{existing}.  If non-nil, requires that the
    result be a previously-interned symbol.
  \item Argument \texttt{exported}.  Requires that the result be not
    only a previously-interned symbol, but also one exported by its
    package.  Ignored if \texttt{:existing} is explicitly set to nil.
  \item Argument \texttt{package}.  Specifies the package from which
    the symbol will be generated.  If omitted, a package is selected
    at random from the existing ones.
  \item Argument \texttt{nonnull}.  If non-nil, allows
    \texttt{arbitrary} to ignore other restriction to guarantee
    returning a non-nil symbol.  When null, \texttt{arbitrary} may
    return nil.
  \item Argument \texttt{gensym}.  If non-nil, and if
    \texttt{:existing} is explicitly set to nil, returns a new
    uninterned symbol.
  \end{itemize}
  
\item Type \texttt{cons}:\indexLisp{cons}\indexKeyword{car}\indexKeyword{cdr}
  \begin{itemize}
  \item Arguments \texttt{car} and \texttt{cdr} should be additional
    type specifications, used direct the generation of respectively
    the left and right elements of the result.  Each defaults to
    \texttt{t}~.
  \end{itemize}
  
\item Types \texttt{list} and \texttt{vector}:\indexLisp{list}\indexLisp{vector}\indexKeyword{length}\indexKeyword{elem}
  \begin{itemize}
  \item Argument \texttt{length} specifies the length of the structure.
    If omitted, will be randomly generated.
  \item Argument \texttt{elem} directs the generation of the
    container's elements.  For both, the default element type is
    \texttt{t}~.
  \end{itemize}

\item Type \texttt{array}:
  \begin{itemize}
  \item Argument \texttt{elem}.  As for \texttt{list} and
    \texttt{vector}.
  \item Argument \texttt{dimens}.  Should be a list of nonnegative
    integers specifying the length of each dimension of the array.  If
    omitted, will be randomly generated.
  \item Argument \texttt{rank}.  Specifies the number of dimensions.
    If omitted but \texttt{:dimens} is given, will be set to the
    length of \texttt{:dimens}.  If both \texttt{:rank} and
    \texttt{:dimens} are omitted, then both are randomly generated.
  \end{itemize}
  
\item Type \texttt{hash-table}:\indexLisp{hash-table}\indexKeyword{size}\indexKeyword{test}\indexKeyword{key}\indexKeyword{val}
  \begin{itemize}
  \item Argument \texttt{size}.  Specifies the number of entries in
    the table.  If omitted, will be randomly generated.
  \item Argument \texttt{test}.  Specifies the hash table's test
    function.  If omitted, will be randomly selected from \texttt{eq},
    \texttt{eql}, \texttt{equal} and
    \texttt{equalp}.\indexLisp{eq}\indexLisp{eql}\indexLisp{equal}\indexLisp{equalp}
  \item Arguments \texttt{key} and \texttt{val} direct the generation
    of the table's keys and values, respectively.  For the keys, the
    default element type is \texttt{t} when the test function is
    \texttt{eq} or \texttt{eql}, and \texttt{scalar} otherwise.  For
    the values, the default element type is \texttt{t}~.\indexLisp{scalar}
  \end{itemize}
\end{itemize}

Beyond those standard Lisp types, NST provides the type
\texttt{scalar} as a supertype of the numeric types plus
\texttt{character}, \texttt{string} and \texttt{symbol}.  Users may
extend this definition to include additional type specifications, as
we discuss below.  Types are not associated with \texttt{scalar} are
referred to as \emph{compound} (although there is no corresponding
type specification).  To avoid generating structures too large to hold
in memory, NST provides the global variable
\texttt{*max-compound-structure-depth*}\label{max-compound-structure-depth}\indexLispGlobal{max-compound-structure-depth}
and the macro \texttt{compund-structure}.
\input{../gen/*max-compound-structure-depth*_variable_nst-item-style}
\input{../gen/compound-structure_compiler-macro_nst-item-style}
\par
\input{../gen/def-arbitrary-instance-type_compiler-macro_nst-item-style}

\subsection{Invariants as tests}
\label{sec:quickcheck-tests}
\input{../gen/sample_criterion_nst-item-style}

%%% Local Variables: 
%%% mode: latex
%%% TeX-master: "manual"
%%% TeX-PDF-mode: t
%%% End: 
