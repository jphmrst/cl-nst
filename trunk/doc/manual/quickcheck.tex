
\section{Verifying invariants against sampled data}
\label{quickcheck}
The \texttt{:sample}\indexKeyword{sample} criterion provides random
generation of data for validating program properties.  Our approach is
based on Claessen and Hughes's Quickcheck\footnote{Koen Claessen and
  John Hughes, ``QuickCheck: a lightweight tool for random testing of
  Haskell programs,'' from \emph{Proceedings of the International
    Conference on Functional Programming}, 2000.  QuickCheck papers,
  code and other resources are available at
  \textsl{www.cs.chalmers.se/\~{}rjmh/QuickCheck}~.}.

This style of testing is somewhat more complicated than specific tests
on single, bespoke forms.  There are two distinct efforts, which we
address in the next two sections: describing how the sample data is to
be generated, and specifying the test itself.

\subsection{Generating sample data}
\label{sec:quickcheck-data}
Data generation is centered around the generic function
\texttt{arbitrary}.\indexLisp{arbitrary} %
\input{../gen/arbitrary_function_nst-item-style}

\input{./arbitrary-builtin}

\subsection{Invariants as tests}
\label{sec:quickcheck-tests}
\input{../gen/sample_criterion_nst-item-style}

%%% Local Variables: 
%%% mode: latex
%%% TeX-master: "manual"
%%% TeX-PDF-mode: t
%%% End: 
