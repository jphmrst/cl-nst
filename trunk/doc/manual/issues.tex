
%%% Local Variables: 
%%% mode: latex
%%% TeX-master: "manual"
%%% End: 
\section{Known bugs and enhancement ideas}\index{bugs}\index{enhancements}
\fbox{TOTALLY OUT OF DATE}
\begin{enumerate}
\item We do a poor job of printing the conditions that come back from
  erring tests.  Moreover, the
  \texttt{def-check}\index{def-check@\texttt{def-check}} macro should
  be able to provide better feedback about the context of an error
  when failing (the kind of failing that doesn't throw an error), not
  just at the time, but also when
  \texttt{:blurb}ing\index{blurb@\texttt{:blurb}} tests
  after-the-fact.
\item There are still, sometimes, warnings from the use of fixture
  names in other fixtures.
\item We are not yet catching errors arising from
  \texttt{:cleanup}\index{cleanup@\texttt{:cleanup}} blocks.
\item Clobbering of test/group names should be caught by the macros,
  not by the lower-level expansions arising from those methods.
\item In the runtime system, the implementations of
  \texttt{:blurb}\index{blurb@\texttt{:blurb}},
  \texttt{:open-used}\index{open-used@\texttt{:open-used}} and
  \texttt{:reopen}\index{reopen@\texttt{:reopen}} are questionable.
\item Possibly,
  \texttt{run-nst-commands}\index{run-nst-commands@\texttt{run-nst-commands}}
  should be turned into a macro.
\item The def-check form \texttt{:perf} might be extended to provide
  (implementation-dependent) checks on memory limits as well as time
  limits.
\item Some sort of timeout mechanism (again, possibly
  implementation-dependent) could be helpful for measuring correctness
  via termination in more complicated algorithms.
\end{enumerate}

\subsection{Removed features}
\label{temp-removed}
Some features of the NST 0.9 have not yet been reimplemented in NST
1.0.  Tell John if one is urgent for you.

\subsection{Alternate fixture definitions}
\fbox{UPDATE}

The
\texttt{def-capture/restore-fixtures}\index{def-capture/restore-fixtures@\texttt{def-capture/restore-fixtures}}
declaration binds \texttt{nil} to a collection of variables in the
extent of associated test groups.  This form is useful when hiding
some developers' state from tests intended to simulate a
non-development environment.
\begin{center}
\textbf{def-capture/restore-fixtures}
 \textit{name} \textit{variables}
\texttt{\&key} \textit{documentation}
\end{center}

\subsection{Anonymous fixture sets}
\fbox{UPDATE}

It is not necessary to name fixture sets; they may be given
anonymously in any situation where a fixture set name is allowed.  The
syntax of anonymous fixture sets
is:\index{fixtures@\texttt{:fixtures}}\index{fixtures!anonymous}
\begin{center}
\texttt{(:fixtures }\textit{(name form)}$^+$\texttt{ )}
\end{center}
The examples of Section~\ref{tests} show the use of anonymous fixture
sets.
