%% File getting-started.tex
%%
%% NST by John Maraist, based on RRT by Robert Goldman.
%%
%% NST is Copyright (c) 2006-2009 Smart Information Flow Technologies.
%% RRT is Copyright (c) 2005 Robert Goldman, released under the LGPL,
%% and the Lisp-specific preamble to that license.

\documentclass{article}
\usepackage{times}
\usepackage{array}
\usepackage{listings}
\parskip 0.6em
\parindent 0pt

\newenvironment{criteria}%
{\begin{list}{}
    {\setlength{\labelwidth}{0pt}
     \setlength{\leftmargin}{2em}
     \setlength{\rightmargin}{1em}
     \setlength{\itemindent}{0em}}}%
  {\end{list}}
\def\criterion#1#2#3#4{\item[\bfseries #1] #4\par
  \emph{Syntax:}~~\texttt{#2}\\ \emph{Applicable to:}~~#3}
\begin{document}
\thispagestyle{empty}
\begin{center}
  \LARGE Getting started with NST\footnote{This document was last
    updated for NST version 1.0.2.  Written by John Maraist, Smart
    Information Flow Technologies, 211 N.\ First St.\ Suite 300,
    Minneapolis, MN 55401; \textsl{jmaraist} at \textsl{sift.info}.}
\end{center}

\section*{Overview}
\emph{Fixtures} are data structures and values which may be referred
to by name during testing.  NST provides the ability to use fixtures
across multiple tests and test groups, and to inject fixtures into the
runtime namespace for debugging.  Fixtures are defined using the
\texttt{def-fixture} form.

\emph{Groups} of tests can be associated with fixture sets, stateful
initiatization, and stateful cleanup.  Test groups are defined using
the \texttt{def-test-group} form.

Individual tests in NST are called \emph{checks}.  Checks are declared
within test declarations using the \texttt{def-check} form.

Examples of NST fixtures, groups and checks
are available in \textsl{self-test/core/simple-mnst.lisp}~.  The
figure on the next page shows the changes involved in using NST in a
system definition.

\section*{Forms}
The \texttt{def-fixtures} form associates a sequence of names and
forms.
\begin{verbatim}
  (def-fixtures FIXTURE-NAME
          ([ :uses USES ]
           [ :assumes ASSUMES ]
           [ :outer OUTER ]
           [ :inner INNER ]
           [ :documentation DOCUMENTATION ])
    (NAME FORM)
    (NAME FORM)
    ...
    (NAME FORM))
\end{verbatim}
with keyword arguments:
\begin{itemize}

\item\texttt{uses} gives fixtures assumed to also be used whenever
  this fixture is used.

\item\texttt{assumes} lists names assumed to be bound --- although not
  necessarily from the same source at each use of this fixture --- at
  the point of any use of this fixture.

\item\texttt{documentation} is a documentation string for the fixture
  set.

\item\texttt{inner} is the list of declarations to be made inside the
  let-binding of names of any use of this fixture.  Do not include the
  "declare" keyword here; NST adds these declarations to others,
  including a special declaration of all bound names.

\item\texttt{outer} is the list of declarations to be made outside the
  let-binding of names of any use of this fixture.
\end{itemize}

The \texttt{def-test-group} form 
\begin{verbatim}
  (def-test-group NAME (FIXTURE FIXTURE ... FIXTURE)
    [ (:setup FORM FORM ... FORM) ]
    [ (:cleanup FORM FORM ... FORM) ]
    [ (:each-setup FORM FORM ... FORM) ]
    [ (:each-cleanup FORM FORM ... FORM) ]
    FORM
    FORM
    ...
    FORM)
\end{verbatim}
defines a group of the given name, providing one instantiation of the
bindings of the given fixtures to each check.  Forms given with the
\texttt{setup} (respectively \texttt{cleanup}) option are run once,
before (after) any of the individual tests.  Forms given with the
\texttt{each-setup} (respectively \texttt{each-cleanup}) option are
run before (after) each individual tests.

The \texttt{def-check} form defines a single named test:
\begin{verbatim}
  (def-check NAME
      criterion
    FORM FORM ... FORM)
  (def-check (NAME [ :setup FORM ]
                   [ :cleanup FORM ]
                   [ :fixtures (FIXTURE FIXTURE
                                 ... FIXTURE) ] )
      criterion
    FORM FORM ... FORM)
\end{verbatim}
The \texttt{:setup}, \texttt{:cleanup} and \texttt{:fixture} options
are as above, but apply to only this check (and note that for multiple
forms for the first two must be wrapped in a \texttt{progn}).  NST's
built-in criteria are listed below; see the manual for a discussion of
the forms which define new criteria.

\section*{Criteria}
Criteria forms have the following structure:
\begin{verbatim}
  (CRITERIA-NAME ARG ARG ... ARG)
\end{verbatim}
but a no-argument criterion use \texttt{(NAME)} can be abbreviated as
\texttt{NAME}.  NST's built-in criteria include:
\begin{criteria}
  \criterion{:pass}{:pass}{}{A trivial test, which always passes.}

  \criterion{:fail}{(:fail FORMAT ARG ...\ ARG)}{Any.}{A trivial test,
    which always fails.  The format string and arguments should be
    suitable for the Lisp \texttt{format} function.}

  \criterion{:warn}{(:warn FORMAT ARG ...\ ARG)}{Any.}{Issue a warning.
    The format string and arguments should be suitable for the Lisp
    \texttt{format} function.}

  \criterion{:true}{:true}{A single form.}{The test passes if the
    form evaluates to non-nil.}

  \criterion{:eq}{(:eq FORM)}{A single form.}{Require a form to
    evaluate to a form which is \texttt{eq} to the criterion argument.}

  \criterion{:symbol}{(:symbol NAME)}{A single form.}{Require a form
    to evaluate to a symbol which is \texttt{eq} to the symbol name
    given as the criterion argument.}

  \criterion{:eql}{(:eql FORM)}{A single form.}{Require a form to
    evaluate to an expression which is \texttt{eql} to the criterion
    argument.}

  \criterion{:equal}{(:equal FORM)}{A single form.}{Require a form to
    evaluate to an expression which is \texttt{equal} to the criterion
    argument.}

  \criterion{:equalp}{(:equalp FORM)}{A single form.}{Require a form
    to evaluate to an expression which is \texttt{equalp} to the
    criterion argument.}

  \criterion{:forms-eq}{:forms-eq}{Exactly two forms.}{Passes for
    forms which evaluate to expressions which are \texttt{eq}.}

  \criterion{:forms-eql}{:forms-eql}{Exactly two forms.}{Passes for
    forms which evaluate to expressions which are \texttt{eql}.}

  \criterion{:forms-equal}{:forms-equal}{Exactly two forms.}{Passes
    for forms which evaluate to expressions which are \texttt{equal}.}

  \criterion{:value-list}
  {(:value-list CRITERION)}
  {Arbitrarily many values.}
  {Converts multiple values into a single list value.}

  \criterion{:predicate}{(:predicate FUNCTION-FORM)}
  {Forms matching the input lambda list of the \texttt{FUNCTION-FORM}.}
  {Passes exactly when the given function returns non-nil.}

  \criterion{:dump-forms}{(:dump-forms FORMAT)} {Arbitrarily many
    values, compatible with the given string for the Lisp
    \texttt{format} function.}{For debugging NST criteria: fails after
    writes the current forms to standard output.}

  \criterion{:err}{(:err :type CLASS)}{Any.}{This test passes when the
    the forms raise an exception of the given \texttt{CLASS}, by
    default \texttt{error}.}

  \criterion{:perf}{(:perf [ :ns | :sec | :min ] TIME}{Any.}{Passes
    when the forms evaluate within the given time limit.}

  \criterion{:not}{(:not CRITERION)}{As required by the subordinate
    criterion.}{Passes when testing according to \texttt{CRITERION}
    either fails or errs.}

  \criterion{:all}{(:all CRITERION CRITERION ...\ CRITERION)}{As
    required by the subordinate criteria.}{Passes when all of the
    subordinate criteria pass.}

  \criterion{:any}{(:any CRITERION CRITERION ...\ CRITERION)}{As
    required by the subordinate criteria.}{Passes when any of the
    subordinate criteria pass.}

  \criterion{:apply}{(:apply FUNCTION CRITERION)}{Forms matching the
    input lambda list of the \texttt{FUNCTION}.}{Evaluates its input
    forms, and applies the \texttt{FUNCTION} to them.  Evaluates the
    subordinate \texttt{CRITERION} with the \texttt{FUNCTION}'s
    multiple result values as input forms.}

  \criterion{:check-err}{(:check-err CRITERION)}{As required by the
    subordinate criterion.}{Like \texttt{:err}, but proceeds according
    to the subordinate criterion rather than simply evaluating the
    input forms.}

  \criterion{:progn}{(:progn FORM FORM ...\ FORM CRITERION)}{As
    required by the subordinate criterion.}{Evaluates the
    \texttt{FORM}s in order, and then proceeds with evaluation of
    check forms according to the subordinate criterion.}

  \criterion{:proj}{(:proj (INDEX INDEX ...\ INDEX) CRITERION)}{At
    least as many as to be accessible to the largest
    index.}{Rearranges the input forms by selecting a new list
    according to the index numbers into the old list.  Checking of the
    reorganized forms continues according ot the subordinate
    criterion.}

  \criterion{:values}{(:values CRITERION CRITERION ...\
    CRITERION)}{Exactly as many forms as subordinate criteria.}{Checks
    each of the input forms according to the respective subordinate
    criterion.}

  \criterion{:each}{(:each CRITERION)}{A single form which evaluates
    to a list.}{Checks each argument of the list according to the
    subordinate \texttt{CRITERION}, and passes when all of these
    checks pass.}

  \criterion{:seq}{(:values CRITERION CRITERION ...\ CRITERION)}{A
    single form which evaluates to a list with the same number of
    elements as there are subordinate criteria to the
    \texttt{:seq}.}{Evaluates its input form, checks each of its
    elements according to the respective subordinate criterion, and
    passes when all of them pass.}

  \criterion{:permute}{(:permute CRITERION)}{A single form evaluating
    to a list.}{Passes when some permutation of the evaluated list
    will satisfy the subordinate criterion.}

  \criterion{:across}{(:across CRITERION CRITERION ...\ CRITERION)}{A
    single form evaluating to a vector.}{Like \texttt{:seq}, but for a
    vector instead of a list.}

  \criterion{:slots}{\small(:slots (NAME CRT) (NAME CRT) ...\ (NAME
    CRT))}{A single form evaluating to a class or struct
    object.}{Evaluates its input form, and passes when the value at
    each given slot satisfies the corresponding subordinate
    constraint.}
\end{criteria}

\section*{Interactive debugging}
NST defines a REPL alias \texttt{:nst} under Allegro CL.  The general
form of commands is:
\begin{verbatim}
  :nst COMMAND ARGUMENTS
\end{verbatim}
Use \mbox{\texttt{:nst :help}} for a list of commands, and
\mbox{\texttt{:nst COMMAND :help}} for details about individual
commands.

\section*{ASDF}
NST tests can be referenced from ASDF systems, allowing easy
invocation to unit tests relevant to a system.  The system
\textsl{self-test/masdfnst.asd} gives an example of its use.  The
snippet below highlights the difference between non-NST and
NST-oriented ASDF system declarations.
\texttt{\small
\begin{tabbing}
\\;; Force loading NST's ASDF utilities before processing
\\;; this file.
\\{\bfseries (asdf:oos 'asdf:load-op :asdf-nst)}
\\
\\(de\=fpackage :masdfnst-asd
\\   \>(:use :common-lisp :asdf {\bfseries :asdf-nst}))
\\(in-package :masdfnst-asd)
\\
\\(def\=system :masdfnst
\\    \>;; Use the NST-oriented ASDF system definition.
\\    \>{\bfseries :class nst-testable}
\\    
\\    \>:in-order-to ((test-op (load-op :masdfnst)))
\\
\\    \>;; Any one of the six blocks below is reasonable.
\\    \>;; Use exactly one of :nst-package, :nst-group, or
\\    \>;; :nst-test; or any combination of the plural
\\    \>;; versions.
\\
\\    \>{\bfseries ;; (1)}
\\    \>{\bfseries ;; :nst-package :asdf-nst-test}
\\
\\    \>{\bfseries ;; (2)}
\\    \>{\bfseries ;; :nst-group (:asdf-nst-test . core-checks)}
\\
\\    \>{\bfseries ;; (3)}
\\    \>{\bfseries ;; :nst-test (:asdf-nst-test core-checks pass-1)}
\\
\\    \>{\bfseries ;; (4)}
\\    \>{\bfseries ;; :nst-packages (:asdf-nst-test :asdf-nst-test2)}
\\
\\    \>{\bfseries ;; (5)}
\\    \>{\bfseries ;; :nst-packages (:asdf-nst-test)}
\\    \>{\bfseries ;; :nst-groups ((:asdf-nst-test2 . :g1a))}
\\    
\\    \>{\bfseries ;; (6)}
\\    \>{\bfseries :nst-groups ((:asdf-nst-test2 . :g1))}
\\    \>{\bfseries :nst-tests (}\={\bfseries (:asdf-nst-test2 :g1a :fix0)}
\\		\> \>{\bfseries (:asdf-nst-test :core-checks :warn-1))}
\\    
\\    \>:components ( ...\ ))
\end{tabbing}}


\end{document}
