
%% \criteriaDoc
%%    1 - criteria name without punctuation
%%    2 - criteria name with punctuation
%%    3 - syntax
%%    4 - applicable to
%%    5 - blurb
%%    6 - additional text --- wrap in \hasExpl, or \noExpl
%%    7 - example type --- \noEx, \singleEx, \multiEx, \tabbingEx
%%    8 - example(s)

%% \kwdCriteriaDoc
%%    1 - criteria name without colon
%%    2 - syntax
%%    3 - applicable to
%%    4 - blurb
%%    5 - additional text --- wrap in \hasExpl, or \noExpl
%%    6 - example type --- \noEx, \singleEx, \multiEx, \tabbingEx
%%    7 - example(s)

\def\kwdCriteriaDoc#1#2#3#4#5#6#7{\criteriaDoc{#1}{:#1}{#2}{#3}{#4}{#5}{#6}{#7}}

\begin{criteriaGroup}{Basic checks}{}

\kwdCriteriaDoc{true}{:true}
{The test passes if the form evaluates to non-nil.}
{A single form.}
{\noExpl}
{\noEx}{}

\kwdCriteriaDoc{eq}{(:eq FORM)}
{Require a form to evaluate to a form which is \texttt{eq} to the
criterion argument.}
{A single form.}
{\noExpl}
{\singleEx}{(def-check eq1 () (:eq 'b) (cadr '(a b c)))}

\kwdCriteriaDoc{symbol}{(:symbol NAME)}
{Require a form to evaluate to a symbol which is \texttt{eq} to the
symbol name given as the criterion argument.}
{A single form.}
{\noExpl}
{\multiEx}
{\passExample{(def-check sym1  () (:symbol a) (car '(a b c)))}
 \failExample{(def-check sym1x () (:symbol a) (cadr '(a b c)))}}

\kwdCriteriaDoc{eql}{(:eql FORM)}
{Require a form to evaluate to an expression which is \texttt{eql} to
the criterion argument.}
{A single form.}
{\noExpl}
{\singleEx}{(def-check eql1 (:eql 2) (cadr '(1 2 3)))}

\kwdCriteriaDoc{equal}{(:equal FORM)}
{Require a form to evaluate to an expression which is \texttt{equal}
to the criterion argument.}
{A single form.}
{\noExpl}
{\noEx}{}

\kwdCriteriaDoc{equalp}{(:equalp FORM)}
{Require a form to evaluate to an expression which is \texttt{equalp}
to the criterion argument.}
{A single form.}
{\noExpl}
{\noEx}{}

\kwdCriteriaDoc{forms-eq}{:forms-eq}
{Passes for forms which evaluate to expressions which are
\texttt{eq}.}
{Exactly two forms.}
{\noExpl}
{\singleEx}{(def-check eqforms1 :forms-eq (cadr '(a b c)) (caddr '(a c b)))}

\kwdCriteriaDoc{forms-eql}{:forms-eql}
{Passes for forms which evaluate to expressions which are
\texttt{eql}.}
{Exactly two forms.}
{\noExpl}
{\singleEx}{(def-check eqlforms1 :forms-eq (cadr '(a 3 c)) (caddr '(a c 3)))}

\kwdCriteriaDoc{forms-equal}{:forms-equal}
{Passes for forms which evaluate to expressions which are
\texttt{equal}.}
{Exactly two forms.}
{\noExpl}
{\noEx}{}

\kwdCriteriaDoc{predicate}{(:predicate FUNCTION-FORM)}
{Applies some predicate function to a form, and verifies that the
result is non-nil.  Note that the predicate is applied to \emph{all}
of the forms following the criterion.}
{Forms matching the input lambda list of the \texttt{FUNCTION-FORM}.}
{\noExpl}
{\multiEx}
{\passExample{(def-check pred1 (:predicate numberp) 3)}
 \passExample{(def-check pred2 (:predicate eql) (+ 1 2) 3)}}

\kwdCriteriaDoc{err}{(:err :type CLASS)}
{This test passes when the the forms raise an exception of the given
\texttt{CLASS}, by default \texttt{error}.}
{Any.}
{\noExpl}
{\singleEx}{(def-check err1 (:err) (error "this should be caught"))}

\kwdCriteriaDoc{perf}{(:perf [ :ns | :sec | :min ] TIME)}
{Passes when the forms evaluate within the given time limit.}
{Any.}
{\noExpl}
{\singleEx}{(def-check perf1 (:perf :min 2) (ack 3 5))}

\end{criteriaGroup}

\begin{criteriaGroup}{Compound checks}{}

\kwdCriteriaDoc{not}{(:not CRITERION)}
{Passes when testing according to \texttt{CRITERION} either fails (but
does not throw an error).}
{As required by the subordinate criterion.}
{\noExpl}
{\singleEx}{(def-check not1 (:not (:symbol b)) 'a)}

\kwdCriteriaDoc{all}{(:all CRITERION CRITERION ...\ CRITERION)}
{This criterion brings several other criteria under one check, and
verifies that they all pass.}
{As required by the subordinate criteria.}
{\noExpl}
{\tabbingEx}{\texttt{(de}\=\texttt{f-}\=\texttt{check not1 ()}
\\ \>\>\texttt{(:all }\=\texttt{(:predicate even-p)}
\\ \>\>\>\texttt{(:predicate prime-p))}
\\ \>\texttt{2)}}

\kwdCriteriaDoc{any}{(:any CRITERION CRITERION ...\ CRITERION)}
{Passes when any of the subordinate criteria pass.}
{As required by the subordinate criteria.}
{\noExpl}
{\tabbingEx}{
\texttt{(de}\=\texttt{f-}\=\texttt{check not1 ()}
\\ \>\>\texttt{(:any }\=\texttt{(:predicate even-p)}
\\ \>\>\>\texttt{(:predicate prime-p))}
\\ \>\texttt{5)}}

\kwdCriteriaDoc{apply}{(:apply FUNCTION CRITERION)}
{Evaluates its input forms, and applies the \texttt{FUNCTION} to them.
Evaluates the subordinate \texttt{CRITERION} with the
\texttt{FUNCTION}'s multiple result values as input forms.}
{Forms matching the input lambda list of the \texttt{FUNCTION}.}
{\noExpl}
{\singleEx}{(def-check applycheck (:apply cadr (:eql 10)) '(0 10 20))}

\kwdCriteriaDoc{check-err}{(:check-err CRITERION)}
{Like \texttt{:err}, but proceeds according to the subordinate
criterion rather than simply evaluating the input forms.}
{As required by the subordinate criterion.}
{\hasExpl{The difference between \texttt{:check-err} and \texttt{:err}
is that the latter deals only with evaluation of a form, whereas
\texttt{:check-err} is more about the unit testing process.  This form
is mostly useful for temporarily disregarding certain checks until
some later fix, when they \emph{won't} throw an error.}}
{\singleEx}
{(def-check check-err1 (:check-err :forms-eq) 'asdfgh (error "this should be caught"))}

\kwdCriteriaDoc{progn}{(:progn FORM FORM ...\ FORM CRITERION)}
{Evaluates the \texttt{FORM}s in order, and then proceeds with
evaluation of check forms according to the subordinate criterion.}
{As required by the subordinate criterion.}
{\noExpl}
{\singleEx}
{(def-check form1 (:progn (setf zz 3) (:eql 3)) zz)}

\kwdCriteriaDoc{proj}{(:proj (INDEX INDEX ...\ INDEX) CRITERION)}
{Rearranges the input forms by selecting a new list according to the
index numbers into the old list.  Checking of the reorganized forms
continues according ot the subordinate criterion.}
{At least as many as to be accessible to the largest
index.}
{\hasExpl{Note that containing criteria may have reordered forms and
value from the original check.}}
{\singleEx}
{(def-check proj-1 (:proj (0 2) :forms-eq) 'a 3 (car '(a b)))}

\end{criteriaGroup}

\begin{criteriaGroup}{Programmatic and debugging checks}{}

\kwdCriteriaDoc{pass}{:pass}
{A trivial test, which always passes.}
{Any.}
{\noExpl}
{\singleEx}{(def-check passing-test :pass 3 4 "sd")}

\kwdCriteriaDoc{fail}{(:fail FORMAT ARG ...\ ARG)}
{A trivial test, which always fails.  The format string and arguments
should be suitable for the Lisp \texttt{format} function.}
{Any.}
{\noExpl}
{\singleEx}{(def-check fails (:fail "Expected a \~{}a" "string") 312)}

\kwdCriteriaDoc{warn}{(:warn FORMAT ARG ...\ ARG)}
{Issue a warning.  The format string and arguments should be suitable for
 the Lisp \texttt{format} function.}
{Any.}
{\noExpl}
{\singleEx}{(:warn "\~{}d is not a perfect square" 5)}

\kwdCriteriaDoc{dump-forms}{(:dump-forms FORMAT)}
{For debugging NST criteria: fails after writes the current forms to
standard output.}
{Arbitrarily many values, compatible with the given string for the
Lisp \texttt{format} function.}
{\noExpl}{\noEx}{}

\end{criteriaGroup}

\begin{criteriaGroup}{Checks on multiple values}{
NST's approach to multiple values is stricter than Common Lisp's view
in the language itself.   }

\kwdCriteriaDoc{value-list}{(:value-list CRITERION)}
{Converts multiple values into a single list value.}
{Arbitrarily many values.}
{\noExpl}{\noEx}{}

\kwdCriteriaDoc{values}{(:values CRITERION CRITERION ...\ CRITERION)}
{Checks each of the input forms according to the respective
subordinate criterion.}
{Exactly as many forms as subordinate criteria.}
{\noExpl}{\noEx}{}

\end{criteriaGroup}

\begin{criteriaGroup}{Checks on lists}{}

\kwdCriteriaDoc{each}{(:each CRITERION)}
{Checks each argument of the list according to the subordinate
\texttt{CRITERION}, and passes when all of these checks pass.}
{A single form which evaluates to a list.}
{\noExpl}
{\singleEx}
{(def-check each1 (:each (:symbol a)) '(a a a a a))}

\kwdCriteriaDoc{seq}{(:values CRITERION CRITERION ...\ CRITERION)}
{Evaluates its input form, checks each of its elements according to
the respective subordinate criterion, and passes when all of them
pass.}
{A single form which evaluates to a list with the same number of
elements as there are subordinate criteria to the
\texttt{:seq}.}
{\hasExpl{Note that \texttt{:seq} expects that the length of the list
will be the same as the number of subordinate criteria, and will fail
otherwise.}}
{\tabbingEx}{
(de\=f-\=check seqcheck
\\ \>\>  (:seq (:predicate symbolp) (:eql 1) (:symbol d))
\\ \>  '(a 1 d))
}

\kwdCriteriaDoc{permute}{(:permute CRITERION)}
{Passes when some permutation of the evaluated list will satisfy the
subordinate criterion.}
{A single form evaluating to a list.}
{\noExpl}{\tabbingEx}{
\textrm{Examples:}\ \=(def-check permute1 (:permute (:each (:eq 'a))) '(a a))
\\ \> (de\=f-\=check permute2
\\ \> \> \> (:permute (:seq \=(:symbol b)
\\ \> \> \>             \>(:predicate symbolp)
\\ \> \> \>             \>(:predicate numberp)))
\\ \> \>'(1 a b))
}

\end{criteriaGroup}

\begin{criteriaGroup}{Checks on vectors}{}

\kwdCriteriaDoc{across}{(:across CRITERION CRITERION ...\ CRITERION)}
{Like \texttt{:seq}, but for a vector instead of a list.}
{A single form evaluating to a vector.}
{\noExpl}{\tabbingEx}{
(de\=f-\=check across1
\\ \>\>  (:across (:predicate symbolp) (:eql 1))
\\ \>  (vector 'a 1))}

\end{criteriaGroup}

\begin{criteriaGroup}{Checks on classes}{}

\kwdCriteriaDoc{slots}{\small(:slots (NAME CRT) (NAME CRT) ...\ (NAME
    CRT))}
{Evaluates its input form, and passes when the value at each given
slot satisfies the corresponding subordinate constraint.}
{A single form evaluating to a class or struct object.}
{\hasExpl{Use of this criterion with structs rather than classes does
work on many platforms, since the CL specification defines
\texttt{with-slots} on classes only.}}
{\tabbingEx}{
(defc\=lass classcheck ()
\\ \> (\=(s1 :initarg :s1 :reader get-s1)
\\ \> \> (s2 :initarg :s2)
\\ \> \> (s3 :initarg :s3)))
\\ (def-check \=slot1
\\ \>  (:slots \=(s1 (:eql 10))
\\ \>  \>(s2 (:symbol zz))
\\ \>  \>(s3 (:seq \=(:symbol q) (:symbol w)
\\ \>  \> \> (:symbol e) (:symbol r))))
\\ \>  (m\=ake-instance 'classcheck
\\ \> \>  :s1 10 :s2 'zz :s3 '(q w e r)))
}

\end{criteriaGroup}
