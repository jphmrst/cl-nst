
\section{Test groups}
\label{sec:groups}
Groups of tests\index{group}\index{test group|see{group}} can be
associated with fixture sets, stateful initiatization, and stateful
cleanup.  The syntax of a test group declaration
is:\index{def-test-group@\texttt{def-test-group}}
\begin{verbatim}
  (def-test-group NAME (FIXTURE FIXTURE ... FIXTURE)
    [ (:setup FORM FORM ... FORM) ]
    [ (:cleanup FORM FORM ... FORM) ]
    [ (:startup FORM FORM ... FORM) ]
    [ (:finish FORM FORM ... FORM) ]
    [ (:each-setup FORM FORM ... FORM) ]
    [ (:each-cleanup FORM FORM ... FORM) ]
    [ (:documentation STRING) ]
    TEST
    TEST
    ...
    TEST)
\end{verbatim}

%%% Local Variables: 
%%% mode: latex
%%% TeX-master: "../manual/manual"
%%% TeX-PDF-mode: t
%%% End: 

%
\texttt{NAME} is the name of this test group.  The \texttt{FIXTURE}s
are to be applied to the tests in this group.

The \texttt{:setup} forms\index{setup@\texttt{:setup}},
\texttt{:cleanup}, \texttt{:startup}\index{startup@\texttt{:startup}}
and \texttt{:finish}\index{finish@\texttt{:finish}}\footnote{These
  latter two argument names were previously
  \texttt{:fixtures-setup}\index{fixtures-setup@\texttt{:fixtures-setup}}
  and
  \texttt{:fixtures-cleanup}\index{fixtures-cleanup@\texttt{:fixtures-cleanup}};
  the old names still work but are deprecated.}  forms are just as for
fixtures, run either just within or just outside of the scope of the
fixtures associated with the group.  Note that any forms associated by
these keyword arguments with applied fixtures would be executed in
between the group's \texttt{:startup} and \texttt{:setup} (or
\texttt{:cleanup} and \texttt{:finish}) forms.

The \texttt{:each-setup}\index{each-cleanup@\texttt{:each-cleanup}}
and \texttt{:each-cleanup}\index{each-cleanup@\texttt{:each-cleanup}}
forms are run before each test, rather than once for the group.

The \texttt{:documentation}\index{documentation@\texttt{:documentation}}
form sets the docstring for the class.

%%% Local Variables: 
%%% mode: latex
%%% TeX-master: "manual"
%%% TeX-PDF-mode: t
%%% End: 
