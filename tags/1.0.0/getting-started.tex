%% File getting-started.tex
%%
%% NST by John Maraist, based on RRT by Robert Goldman.
%%
%% NST is Copyright (c) 2006-2009 Smart Information Flow Technologies.
%% RRT is Copyright (c) 2005 Robert Goldman, released under the LGPL,
%% and the Lisp-specific preamble to that license.

\documentclass{article}
\usepackage{times}
\usepackage{listings}
\parskip 0.6em
\parindent 0pt

\begin{document}
\thispagestyle{empty}
\begin{center}
{\LARGE Getting started with NST\footnote{John Maraist, Smart Information Flow Technologies, 211 N.\ First St.\ Suite 300, Minneapolis, MN 55401; \textsl{jmaraist} at \textsl{sift.info}.}}
\end{center}

\section*{NST by example}
\emph{Fixtures} are data structures and values which may be referred
to by name during testing.  NST provides the ability to use fixtures
across multiple tests and test groups, and to inject fixtures into the
runtime namespace for debugging.  Fixtures are defined using the
\texttt{def-fixture} form.

\emph{Groups} of tests can be associated with fixture sets, stateful
initiatization, and stateful cleanup.  Test groups are defined using
the \texttt{def-test-group} form.

Individual tests in NST are called \emph{checks}.  Checks are declared
within test declarations using the \texttt{def-check} form.

Examples of NST fixtures, groups and checks
are available in \textsl{self-test/core/simple-mnst.lisp}~.  The
figure on the next page shows the changes involved in using NST in a
system definition.

\section*{Interactive debugging}
NST defines a REPL alias \texttt{:nst} under Allegro CL.  The general
form of commands is:
\begin{verbatim}
  :nst COMMAND ARGUMENTS
\end{verbatim}
Use \mbox{\texttt{:nst :help}} for a list of commands, and
\mbox{\texttt{:nst COMMAND :help}} for details about individual
commands.
\clearpage
\section*{ASDF}
NST tests can be referenced from ASDF systems, allowing easy
invocation to unit tests relevant to a system.  The system
\textsl{self-test/masdfnst.asd} gives an example of its use.  The
snippet below highlights the difference between non-NST and
NST-oriented ASDF system declarations.
\texttt{\small
\begin{tabbing}
\\;; Force loading NST's ASDF utilities before processing
\\;; this file.
\\{\bfseries (asdf:oos 'asdf:load-op :asdf-nst)}
\\
\\(de\=fpackage :masdfnst-asd
\\   \>(:use :common-lisp :asdf {\bfseries :asdf-nst}))
\\(in-package :masdfnst-asd)
\\
\\(def\=system :masdfnst
\\    \>;; Use the NST-oriented ASDF system definition.
\\    \>{\bfseries :class nst-testable}
\\    
\\    \>:in-order-to ((test-op (load-op :masdfnst)))
\\
\\    \>;; Any one of the six blocks below is reasonable.
\\    \>;; Use exactly one of :nst-package, :nst-group, or
\\    \>;; :nst-test; or any combination of the plural
\\    \>;; versions.
\\
\\    \>{\bfseries ;; (1)}
\\    \>{\bfseries ;; :nst-package :asdf-nst-test}
\\
\\    \>{\bfseries ;; (2)}
\\    \>{\bfseries ;; :nst-group (:asdf-nst-test . core-checks)}
\\
\\    \>{\bfseries ;; (3)}
\\    \>{\bfseries ;; :nst-test (:asdf-nst-test core-checks pass-1)}
\\
\\    \>{\bfseries ;; (4)}
\\    \>{\bfseries ;; :nst-packages (:asdf-nst-test :asdf-nst-test2)}
\\
\\    \>{\bfseries ;; (5)}
\\    \>{\bfseries ;; :nst-packages (:asdf-nst-test)}
\\    \>{\bfseries ;; :nst-groups ((:asdf-nst-test2 . :g1a))}
\\    
\\    \>{\bfseries ;; (6)}
\\    \>{\bfseries :nst-groups ((:asdf-nst-test2 . :g1))}
\\    \>{\bfseries :nst-tests (}\={\bfseries (:asdf-nst-test2 :g1a :fix0)}
\\		\> \>{\bfseries (:asdf-nst-test :core-checks :warn-1))}
\\    
\\    \>:components ( ... ))
\end{tabbing}}


\end{document}
