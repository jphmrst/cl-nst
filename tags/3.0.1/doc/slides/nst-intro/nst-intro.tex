\documentclass{beamer}

\usepackage{pgf}
% \setbeameroption{show only notes}

\mode<presentation>
{
  \usetheme{Warsaw}
  % or ...

  \setbeamercovered{transparent}
  % or whatever (possibly just delete it)
}


\usepackage[english]{babel}
% or whatever

\usepackage[latin1]{inputenc}
% or whatever

\usepackage{times}
\usepackage[T1]{fontenc}
% Or whatever. Note that the encoding and the font should match. If T1
% does not look nice, try deleting the line with the fontenc.



\title{NST: A Unit Test Framework for Common Lisp}

% \subtitle
% {Include Only If Paper Has a Subtitle}

\author{John Maraist}
\institute[SIFT, LLC]{Smart Information Flow Technologies (SIFT, LLC)}

\date{TC-lispers, June 9, 2009}

\subject{Common Lisp unit testing}
% This is only inserted into the PDF information catalog. Can be left
% out. 



% If you have a file called "university-logo-filename.xxx", where xxx
% is a graphic format that can be processed by latex or pdflatex,
% resp., then you can add a logo as follows:

% \pgfdeclareimage[height=0.5cm]{university-logo}{university-logo-filename}
% \logo{\pgfuseimage{university-logo}}




% If you wish to uncover everything in a step-wise fashion, uncomment
% the following command: 

%\beamerdefaultoverlayspecification{<+->}


\begin{document}

\begin{frame}
  \titlepage
\end{frame}

\begin{frame}
  \frametitle{Outline}
  \tableofcontents
\end{frame}


% Structuring a talk is a difficult task and the following structure
% may not be suitable. Here are some rules that apply for this
% solution: 

% - Exactly two or three sections (other than the summary).
% - At *most* three subsections per section.
% - Talk about 30s to 2min per frame. So there should be between about
%   15 and 30 frames, all told.

% - A conference audience is likely to know very little of what you
%   are going to talk about. So *simplify*!
% - In a 20min talk, getting the main ideas across is hard
%   enough. Leave out details, even if it means being less precise than
%   you think necessary.
% - If you omit details that are vital to the proof/implementation,
%   just say so once. Everybody will be happy with that.

%%%%%%%%%%%%%%%%%%%%%%%%%%%%%%%%%%%%%%%%%%%%%%%%%%%%%%%%%%%%

% \begin{frame}
%   \frametitle{LTML Objectives}
%   \begin{itemize}
%   \item Radically composable
%     \begin{itemize}
%     \item<1-> Every component is \emph{tagged} so that individual learners can
%       tailor arbitrary program elements.
%     \item<2-> LTML representable as RDF --- which has caused many challenges!
%       \begin{itemize}
%       \item<3-> No scoping!
% 
%       \item<4-> No variables!
% 
%       \item<5-> No contexts!
%       \end{itemize}
%     \end{itemize}
%   \item<6-> S-expression syntax(es)
%   \item<7-> Compatible with existing notations
%     \begin{itemize}
%     \item<8-> OWL --- Web ontology language
%     \item<9-> OWL-S --- Semantic markup for web services
%     \item<10-> PDDL --- Planner Domain Description Language
%     \end{itemize}
%   \end{itemize}
% \end{frame}
% 
% \pgfdeclareimage[width=0.48\textwidth]{boxology}{POIROT-boxology}
% 
% \begin{frame}
%   \frametitle{Boxology}
%   \begin{columns}[T]
%     \begin{column}{.48\textwidth}
%       \pgfuseimage{boxology}
%       \emph{Suggest we simplify this boxology, removing the OVM, and make it
%         bigger and clearer...}
%     \end{column}
%     \begin{column}{.48\textwidth}
%       \begin{itemize}
%       \item Blackboard-based architecture.
%       \item The multiple learners codify components of an overall workflow.
% %         \begin{itemize}
% %         \item Methods for performing tasks;
% %         \item Criteria for choosing objects.
% %         \end{itemize}
%       \item Other components assemble a full workflow (program) out of these components.
% %         \begin{itemize}
% %         \item Stitcher
% %         \item SHOP2
% %         \end{itemize}
%       \item SHOPPER runs the workflow to demonstrate successful learning.
%       \end{itemize}
%     \end{column}
%   \end{columns}
% \end{frame}

\section{Unit testing}

\begin{frame}
  \frametitle{What is unit testing?}

  From wikipedia:
  \begin{quotation}\noindent
    Unit testing is a software verification and validation method
    where the programmer gains confidence that individual units of
    source code are fit for use.
  \end{quotation}

  \begin{itemize}
  \item ``Unit'' refers to the basic elements of program design ---
    procedures, functions, classes, etc.
  \item Unit tests should be independant of each other.
  \item Typically written by the programmer.
  \item Also usable as regression tests.
  \end{itemize}

\end{frame}

\section{Using NST}

\begin{frame}
  \frametitle{Basic concepts}
\textbf{Fixtures}
\begin{itemize}
\item Named data values to which we apply tests.
\end{itemize}
\textbf{Groups}
\begin{itemize}
\item Collections of tests.
\end{itemize}
\textbf{Tests}
\begin{itemize}
\item One application of a criterion to values, usually to fixtures.
\end{itemize}
\textbf{Criteria}
\begin{itemize}
\item Named process of verification.
\end{itemize}
\end{frame}

\begin{frame}[fragile]
  \frametitle{A simple NST example}
\only<1>{Define two fixtures.}
\only<2>{Define a group of tests.}
\only<3>{The groups' tests are given the two fixtures.}
\only<4>{Test names.}
\only<5>{Test criteria.}
\only<6>{Forms to be tested.}
\only<7>{This test passes.}
\only<8>{This test fails.}
{\small\begin{semiverbatim}
\alert<1>{(def-fixtures simple-fixture}
\alert<1>{    (:documentation "Define two bindings")}
\alert<1>{  (magic-number 120)}
\alert<1>{  (magic-symbol 'asdfg))}
\alert<2>{(def-test-group simple-test-group}
    (\alert<3>{simple-fixtures})
  \alert<7>{(def-test \alert<4>{has-num}}
      \alert<7>{\alert<5>{(:eql \alert<3>{magic-number})}}
    \alert<7>{\alert<6>{(factorial 5)})}
  \alert<8>{(def-test \alert<4>{has-sym}}
      \alert<8>{\alert<5>{(:eq \alert<3>{magic-symbol})}}
    \alert<8>{\alert<6>{'asdfh})}\alert<2>{)}
\end{semiverbatim}}
\end{frame}

\subsection{Fixtures}

\begin{frame}
  \frametitle{Fixtures}
{\ttfamily
\begin{tabbing}
  (d\=ef\=-fixtures FIXTURE-NAME
\\  \>  \>( \=[ :uses USES ]
\\  \>  \>  \>[ :assumes ASSUMES ]
\\  \>  \>  \>[ :inner INNER ]
\\  \>  \>  \>[ :documentation DOCUMENTATION ] )
\\  \>(NAME FORM)
\\  \>(NAME FORM)
\\  \>~...
\\  \>(NAME FORM))
\end{tabbing}}
\end{frame}

\begin{frame}[fragile]
  \frametitle{Fixtures and free variables}
Keyword argument \texttt{:assumes}
\begin{itemize}
\item Names the variables occurring free in the fixture bodies.
\begin{verbatim}
(def-fixtures derived-fixtures-1
    (:assumes (magic-number))
  (magic-bonus (* 6 magic-number)))
\end{verbatim}
\end{itemize}
Keyword argument \texttt{:uses}
\begin{itemize}
\item Names the other fixture sets whose bound names occur free in
  these fixtures bodies.
\begin{verbatim}
(def-fixtures derived-fixtures-2
    (:uses (simple-fixtures))
  (magic-bonus (* 6 magic-number)))
\end{verbatim}
\end{itemize}
\end{frame}

\begin{frame}[fragile]
  \frametitle{Fixtures and declarations}
Keyword argument \texttt{:inner}
\begin{itemize}
\item Provides additional declarations for fixture bodies.
\begin{verbatim}
(def-fixtures internal-magic
    (:inner ((special magic-internal-state)))
  (state-head (car magic-internal-state))
  (state-snd  (cadr magic-internal-state)))
\end{verbatim}
\item Should be renamed \texttt{:declare}
\end{itemize}
Keyword argument \texttt{:outer}
\begin{itemize}
\item Ignored --- useless hangover from an earlier design.
\end{itemize}
  
\end{frame}

\subsection{Test groups}

\begin{frame}
  \frametitle{Test groups}
  {\ttfamily
    \begin{tabbing}
      (d\=ef-test-group NAME (FIXTURE $\cdots$ FIXTURE)
      \\    \>[ (:setup FORM FORM $\cdots$ FORM) ]
      \\    \>[ (:cleanup FORM FORM $\cdots$ FORM) ]
      \\    \>[ (:each-setup FORM FORM $\cdots$ FORM) ]
      \\    \>[ (:each-cleanup FORM FORM $\cdots$ FORM) ]
      \\    \>TEST
      \\    \>TEST
      \\    \>...
      \\    \>TEST)
    \end{tabbing}}
\end{frame}

\begin{frame}[fragile]
  \frametitle{Setup and cleanup forms}
{\small
  \begin{columns}[T]
    \begin{column}{2.65in}
\begin{verbatim}
(def-test-group show-setup ()
  (:setup   (write " S group"))
  (:cleanup (write " C group"))
  (:each-setup
    (write "  S-each group"))
  (:each-cleanup
    (write "  C-each group"))
  (def-check ts1 :pass
    (write "   ts1"))
  (def-check ts2 :pass
    (write "   ts2")))
\end{verbatim}
    \end{column}
    \begin{column}{2in}
\begin{verbatim}
Running group show-setup
 S group
  S-each group
   ts1
  C-each group
  S-each group
   ts2
  C-each group
 C group
Group show-setup:
        2 of 2 passed
\end{verbatim}
    \end{column}
  \end{columns}
}
\end{frame}

\subsection{Tests}

\begin{frame}[fragile]
  \frametitle{Tests and groups}
\begin{verbatim}
(def-test (NAME [ :setup FORM ]
                [ :cleanup FORM ]
                [ :fixtures (FXTR ... FXTR) ] )
    criterion
  FORM)

(def-test NAME
    criterion
  FORM)
\end{verbatim}
  
\end{frame}

\begin{frame}[fragile]
  \frametitle{Fixtures for individual tests}
\begin{semiverbatim}
(def-fixtures simple-fixture ()
  (magic-number 120)
  (magic-symbol 'asdfg))

(def-test-group some-magic ()
  (def-test no-magic :true
    \alert{(not (boundp 'magic-number))})

  (def-test (with-magic
             \alert{:fixtures (simple-fixture)})
      (:eql 120)
    \alert{magic-number}))
\end{semiverbatim}
\end{frame}

\begin{frame}[fragile]
  \frametitle{Setup, cleanup on individual tests}
{\small
\begin{semiverbatim}
(defparameter for-setup 0)

(def-test-group setup-cleanup ()
  (:setup (setf for-setup 1))
  (:cleanup (setf for-setup 0))
  (def-test a-sc-for-setup-1 (:eql 1) for-setup)
  (def-test (sc-for-setup-2
             :setup (setf for-setup 2)
             :cleanup (setf for-setup 1))
      (:eql 2)
    for-setup)
  (def-test z-sc-for-setup-1 (:eql 1) for-setup))
\end{semiverbatim}}
\end{frame}

\begin{frame}[fragile]
  \frametitle{Setup, cleanup on individual tests}

  Recently noticed bug: \texttt{each-setup} for the group and
  \texttt{setup} for the test are not applied in the order we'd hope.
  {\small
\begin{semiverbatim}
(def-test-group each-setup-cleanup ()
  (:each-setup (setf for-setup 2))
  (:each-cleanup (setf for-setup 0))
  (def-test (sc-for-setup-2
             :setup (setf for-setup 3)
             :cleanup (setf for-setup 2))
      (:info "This is a known bug" (:eql 3))
    for-setup))
\end{semiverbatim}}
\end{frame}

\subsection{Criteria}

\begin{frame}[fragile]
  \frametitle{Equality criteria}
\begin{verbatim}
(def-test eql1 (:eql 2)
  (cadr '(1 2 3)))

(def-test eq1 (:eq 'a)
  (car '(a 3 c)))
\end{verbatim}
And similarly for \texttt{equal} and \texttt{equalp}.
\begin{verbatim}
(def-test sym1 (:symbol a)
  (car '(a b c)))
\end{verbatim}
\end{frame}

\begin{frame}[fragile]
  \frametitle{Predicates and transformations}
Boolean-valued functions can be used as test criteria:
\begin{verbatim}
  (def-test pred1 (:predicate numberp) 3)
\end{verbatim}
Forms can be altered before testing:
\begin{verbatim}
  (def-test applycheck
      (:apply cadr (:eql 10))
    '(0 10 20))
\end{verbatim}
\end{frame}

\begin{frame}[fragile]
  \frametitle{List criteria}

  Expect every element of a list to pass a criterion:
\begin{verbatim}
  (def-test each1 (:each (:predicate evenp))
    '(2 4 8 20 100))
\end{verbatim}
  Apply different criteria to respective elements:
\begin{verbatim}
  (def-check seqcheck
      (:seq (:predicate symbolp)
            (:eql 1)
            (:symbol d))
    '(a 1 d))
\end{verbatim}
\end{frame}

\begin{frame}[fragile]
  \frametitle{Information and warnings}
  
Additional information for results:
\begin{verbatim}
  (def-test known-bug
      (:info "Known bug" (:eql 3))
    4)
\end{verbatim}
Passing with a warning:
\begin{verbatim}
  (def-test known-bug
      (:warn "~d is not a perfect square" 5)
    5)
\end{verbatim}
\end{frame}

\begin{frame}[fragile]
  \frametitle{Compound criteria}
Negating another criterion:
\begin{verbatim}
  (def-test not1 (:not (:symbol b)) 'a)
\end{verbatim}
Passing all of a set of criteria:
\begin{verbatim}
  (def-check not1 ()
      (:all (:predicate even-p)
            (:predicate prime-p))
    2)
\end{verbatim}
And similarly, \texttt{:any} for at least one of a set of criteria.
\end{frame}

\begin{frame}[fragile]
  \frametitle{Let's argue about multiple values!}

  In Lisp:
\begin{itemize}
\item Functions can return multiple values.
\item But paying attention to the ``extras'' is optional.
\item Accessing ``extras'' takes a little extra effort.
\end{itemize}
\uncover<2->{In NST:}
\begin{itemize}
\item<2-> Even if they're ingorable, extra values must be
  \emph{correctly implemented}.
\item<2-> So when validating a function, NST expects all values to be
  accessed (and presumably validated).
\item<2-> Ignoring ``extras'' takes a little extra effort.
\item<2-> A mismatch between the received and expected values is an
  error.
\end{itemize}
\end{frame}

\begin{frame}[fragile]
  \frametitle{Criteria for multiple values}
Applying different criteria to respective values:
{\small\begin{verbatim}
  (def-test values1 (:values (:symbol a) (:eq 'b))
    (values 'a 'b))
\end{verbatim}}
Treating values as a list:
{\small\begin{verbatim}
  (def-test value-list1
      (:value-list (:seq (:symbol a) (:eq 'b)))
    (values 'a 'b))
\end{verbatim}}
Dropping extra values:
{\small\begin{verbatim}
  (def-test no-values1 (:drop-values (:symbol a))
    (values 'a 'b 'c))  
\end{verbatim}}
\end{frame}

\begin{frame}[fragile]
  \frametitle{Multiple-values and multiple-argument predicates}

  Multiple values are handled as additional arguments to the functions
  underlying criteria:
{\small\begin{verbatim}
  (def-test tricky-1 :eql
    (round 5 4))

  (def-test values-drop1
      (:apply (lambda (x y)
                (declare (ignorable y))
                x)
        (:symbol a))
    (values 'a 'b))
\end{verbatim}}
\end{frame}

\begin{frame}[fragile]
  \frametitle{Other criteria}

  For a full list of criteria, see the manual.
  \begin{itemize}
  \item Additional basic checks.
  \item \texttt{:err} --- expecting an error.
  \item \texttt{:permute} --- list permutation.
  \item Simple checks for vectors, slots.
  \item \texttt{:perf} --- timing form evaluation.
  \end{itemize}
\end{frame}

\subsection{Customizing criteria}

\begin{frame}
  \frametitle{Defining new criteria}

  NST provides three macro-style mechanisms for defining a new
  criterion:
  \begin{itemize}
  \item By describing how it translates to another criterion.
  \item By describing how it maps values to a results expression.
  \item By describing how it maps a form (which would evaluate to a
    list of values) to a results expression.
  \end{itemize}
  In fact, all of the built-in criteria use one of these three
  mechanisms.
\end{frame}

\begin{frame}[fragile]
  \frametitle{Defining new criteria}
  \begin{itemize}
  \item By translation:
{\small
\begin{verbatim}
  (def-criterion-alias (:symbol name)
    `(:eq ',name))
  (def-criterion-alias (:drop-values criterion)
    `(:apply (lambda (x &rest others)
                (declare (ignorable others))
                x)
       ,criterion))
\end{verbatim}}
  \item By a map from values to a results expression: \texttt{:true},
    \texttt{:eql}, \texttt{:predicate}, \texttt{:info}.
  \item By a map from a form to a results expression: \texttt{:not},
    \texttt{:perf}, \texttt{:err}.
  \end{itemize}
\end{frame}

\subsection*{Running NST}

\begin{frame}[fragile]
  \frametitle{Running NST}
  
  \begin{itemize}
  \item<1->\textbf{From the REPL}.  NST provides a command-line
    interface for running tests and inspecting results, with top-level
    aliases for some supported Lisps.
\begin{verbatim}
  :nst :help
\end{verbatim}
  \item<2->\textbf{From ASDF}.  NST provides an ASDF system class,
    providing automatic \texttt{test-op} methods.

  \item<3->\textbf{Output via JUnit}.  NST can generate
    JUnit-compatible XML for GUI browsing of results.

  \end{itemize}
\end{frame}

\section{Inside NST}

\subsection{The basic idea}

\begin{frame}
  \frametitle{A little bit of background}

  A number of Lisp language features greatly simplified writing NST.
  \begin{itemize}
  \item Macros.
  \item Compile-time code execution.
  \item Multiple inheritance.
  \item Sophisticated method dispatch.
  \end{itemize}  
\end{frame}

\begin{frame}
  \frametitle{Macros}

  The top-level NST forms all defined by macros:
  \begin{itemize}
  \item Components not evaluated, but rewritten into other forms which
    (may be) evaluated.
  \item The typical Lisp way of writing language extensions.
  \item Allows a direct specification of the test.
  \end{itemize}
\end{frame}

\begin{frame}
  \frametitle{Compile-time code execution}

  Part of the macro-expansion of tests involves translating criteria
  into Lisp code.
  \begin{itemize}
  \item Criteria definitions expand to method definitions.
  \item So macro-expanding tests involves running Lisp at
    compile-time.
  \item Moreover, the required routines may be found in the same code
    file as the tests they help to expend.
  \end{itemize}
\end{frame}

\begin{frame}
  \frametitle{Multiple inheritance}
  
  Lisp classes may have more than one superclass, and moreover may
  inherit method implementations from different superclass
  hierarchies.
  \begin{itemize}
  \item Unlike e.g.\ Java, where all but one superclass hierarchies
    may define only abstract methods.
  \end{itemize}
  NST uses multiple inheritance for fixture application.
  \begin{itemize}
  \item Fixtures, groups and tests all translate to classes.
  \item Fixtures make their bindings via methods of their class.
  \item Groups' and tests' classes are subclasses of the classes
    corresponding to the fixtures they apply.
  \item So they inherit the fixture-binding methods.
  \end{itemize}
\end{frame}

\begin{frame}[fragile]
  \frametitle{Dynamic variable scope}
  
  Lisp offers both \emph{dynamic} and \emph{static} scoping for local
  variables.
  \begin{itemize}
  \item Determine the rules as to how we know what value-binding a
    program variable should have.
  \item If this idea is new to you, then you're probably used to
    static scope rules.
  \item Simple dynamic scoping example:
{\small \begin{verbatim}
  (defun print-x ()
    (declare (special x))
    (write x))
  (defun x-as-3 ()
    (let ((x 3))
      (declare (special x))
      (print-x)))
\end{verbatim}}
  \end{itemize}
\end{frame}

\begin{frame}
  \frametitle{Method dispatch}
  
  Lisp has a very rich notion of method dispatch.
  \begin{itemize}
  \item Typically, dispatch chooses exactly one method for a
    particular object.
  \item In Lisp, there are many available \emph{method combinations}
    to map a method invocation to some assemblage of the corresponding
    method definitions.
    \begin{itemize}
    \item ``Standard'' principle method override.
    \item ``Around'' methods wrap other methods.
    \item ``Before'' and ``after'' methods.
    \item Sequential execution of several methods, and various ways of
      combining their results.
    \end{itemize}  
  \end{itemize}
\end{frame}

\begin{frame}[fragile]
  \frametitle{Around-methods}

  Used with dynamic scoping rules for fixture definitions:
  {\small
\begin{verbatim}
  (def-fixtures simple-fixture ()
    (magic-number 120) (magic-symbol 'asdfg))
\end{verbatim}}
  becomes something like:
{\small
\begin{verbatim}
  (defmethod run-group :around ((gr simple-fixtures))
    (let ((magic-number 120) (magic-symbol 'asdfg))
      (declare (special magic-number magic-symbol))
      (call-next-method)))
\end{verbatim}}
\end{frame}

\begin{frame}[fragile]
  \frametitle{Before- and after-methods}

  Used for \texttt{:setup} and \texttt{:cleanup} arguments:
{\small
\begin{verbatim}
  (def-test (sc :setup (setf for-setup 2)
                :cleanup (setf for-setup 1))
        (:eql 2)
      for-setup)
\end{verbatim}}
becomes, in part, something like:
  {\small
\begin{verbatim}
  (defmethod run-test :before ((test sc))
    (setf for-setup 2))
  (defmethod run-test :after ((test sc))
    (setf for-setup 1))
\end{verbatim}}
\end{frame}

\subsection{Early implementations, and other lessons}

\begin{frame}
  \frametitle{Fwrappers}
  
  Fwrappers are a Allegro Common Lisp feature.
  \begin{itemize}
  \item Like an around-method for non-generic functions.
  \item Useful for extending third-party code, etc.
  \item But for code we're writing ourselves, no clear advantage over
    around-methods.
  \end{itemize}
\end{frame}

\begin{frame}
  \frametitle{Gensym abuse}

  In one version, NST solved potential name conflicts by naming almost
  all intermediate structures via \texttt{gensym}.
  \begin{itemize}
  \item<2-> Very effective at avoiding accidental name conflicts.
  \item<2-> Very secure against interference with testing internals.
  \item<3-> Very secure against debugging.
  \item<3-> Very secure against examination and understanding.
  \item<3-> Very secure against extension.
  \end{itemize}
\end{frame}

\begin{frame}[fragile]
  \frametitle{Static class data}

  NST must be able to retrieve information associated with fixtures,
  groups and tests.
  \begin{itemize}
  \item We compile fixtures, groups and tests to classes.
  \item We'd prefer this information to be available
    \emph{statically}, with reference to the class but without
    necessarily referencing a particular object.
  \end{itemize}
\end{frame}

\begin{frame}[fragile]
  \frametitle{Static class data}

  With \texttt{:static} allocation for class slots, accessors still
  require an object instance.
  {\small\begin{verbatim}
  (defclass zz () ((aa :allocation :class
                       :accessor zz-aa)))
  (setf z1 (make-instance 'zz)
        z2 (make-instance 'zz))
  (setf (zz-aa z1) 5)
  (pprint (zz-aa z1))  ;; displays 5
  (pprint (zz-aa z2))  ;; displays 5
  (setf (zz-aa z2) 6)
  (pprint (zz-aa z1))  ;; displays 6
\end{verbatim}}
  We'd like a way to store \& retrieve class data \emph{without} an
  object instance.
\end{frame}

\begin{frame}[fragile]
  \frametitle{Class metaobjects for static class data}

  One approach we tried was via the \emph{metaclass}.
  \begin{itemize}
  \item Lisp models a program's structures via standardized Lisp objects.
  \item By default, the definition of a class is stored in an instance
    of \texttt{standard-class}.
  \item But we can extend \texttt{standard-class}, and include static
    data as fields in the metaclass.
  {\small\begin{verbatim}
  (defclass mzz (standard-class)
    ((aaa :accessor zzz-aaa)))
  (defclass zz () () 
    (:metaclass mzz))
  (setf (zzz-aaa (find-class 'zz)) 5)
\end{verbatim}}
  \end{itemize}  
\end{frame}

\begin{frame}[fragile]
  \frametitle{Prefer object prototypes}

  We later discovered the preferred style of using \emph{class
    prototypes}.
  {\small\begin{verbatim}
  (defclass zz () ((aa :allocation :class
                       :accessor zz-aa)))
  (mop:finalize-inheritance (find-class 'zz))
  (defun aa ()
    (zz-aa (class-prototype (find-class 'zz))))
  (defun set-zz-aa (x)
    (setf (zz-aa (mop:class-prototype
                    (find-class 'zz)))
          x)
    x)
\end{verbatim}}
\end{frame}

\begin{frame}[fragile]
  \frametitle{NST internal packages}

  Fixtures correspond to two classes:
  \begin{itemize}
  \item One with methods for applying fixtures to a group once.
  \item One with methods for applying fixtures individually to each
    test within a group.
  \end{itemize}  
  Right now, we create a name for this class.
  {\small\begin{verbatim}
  (def-fixtures pkg::gname |# ... |# 
\end{verbatim}}
  becomes (in part) something like
  {\small\begin{verbatim}
  (defclass nst-fixture-test-class-names::pkg///gname
      () ())
\end{verbatim}}
\end{frame}

\subsection{How it maybe should work}

\begin{frame}
  \frametitle{Package abuse?}

  We would not need special packages for internal class names if the
  classes simply didn't have names.
  \begin{itemize}
  \item By using MOP function calls directly (as opposed to
    \texttt{defclass}) one can create \emph{anonymous} classes.
  \item The internal classes themselves can be made available as
    static information of the reference class.
  \end{itemize}  
  
\end{frame}

\begin{frame}
  \frametitle{A MOP-like model}
  
  Not just macros, but also functions.
  \begin{itemize}
  \item \`A la \texttt{defclass} and \texttt{ensure-class}.
  \item Also suggests: multiple ``front-ends'' (macros) translating to
    the single backend.
  \end{itemize}
\end{frame}

\begin{frame}
  \frametitle{Error reporting}

  Is there actually any programming tool where ``improve error
  messages'' is \emph{not} on the to-do list?
  \begin{itemize}
  \item Errors arising from bad use of criteria (such as treating a
    list with a vector criterion) are not clearly marked as testing
    errors, rather than test failures.
  \item General difficulty in good error analysis.
  \end{itemize}
\end{frame}

\begin{frame}
  \frametitle{QuickCheck}

  QuickCheck originated in Haskell as a framework for generating
  randomized tests of programmer-stated program invariants.
  \begin{itemize}
  \item\texttt{(equal  (reverse (reverse x))  x)}
  \item\texttt{(if (< x y) (eql (max x y) y))}
  \end{itemize}
  Some version of QuickCheck might be useful in NST.
  \begin{itemize}
  \item Side-effects
  \item Cannot rely on types for direction.
  \end{itemize}
\end{frame}

\begin{frame}
  \frametitle{Further integration}

  \begin{itemize}
  \item Better control of NST options through ASDF.
  \item Automated XML/JUnit output.
  \item Other output formats.
  \end{itemize}
\end{frame}

\section{Conclusion}

\begin{frame}
  \frametitle{If you want to really learn Lisp...}

  Writing a unit tester is an effective way to see many interesting
  areas of Lisp!
  \begin{itemize}
  \item Subtle macro authoring.
  \item Evaluation-time issues.
  \item Under the hood of the class system.
  \end{itemize}
\end{frame}

\begin{frame}
  \frametitle{Is this actually unit testing?}
  
  Returning to wikipedia's definition of unit testing:
  \begin{quotation}\noindent\small
    A unit is the smallest testable part of an application. In
    procedural programming a unit may be an individual program,
    function, procedure, etc., while in object-oriented programming,
    the smallest unit is a class, which may belong to a base/super
    class, abstract class or derived/child class.
  \end{quotation}\vspace{-2em}
  \begin{itemize}
  \item Is our unit the expression?
  \item Is there a more natural way to structure tests around the unit
    of a function?
  \item Lisp is \emph{also} object-oriented --- should a Lisp unit
    test package offer a more natural framework for classes and
    methods?
  \end{itemize}
\end{frame}

\begin{frame}
  \frametitle{Proof in the practicality}
  
  Future directions aside, we believe that NST is now, already, a
  useful tool.
  \begin{itemize}
  \item Has been used on several large projects already.
    \begin{itemize}
    \item The SHOP2 planner.
    \item A plan recognizer.
    \item Language interpreter.
    \item Security software.
    \end{itemize}
  \item Reasonable mature and solid.
  \item Actively maintained.
  \item Open-sourced.
  \end{itemize}
\end{frame}

\end{document}



%%% Local Variables: 
%%% mode: latex
%%% TeX-master: t
%%% TeX-PDF-mode: t
%%% End: 
