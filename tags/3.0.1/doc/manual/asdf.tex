\section{Integration with ASDF}
NST's integration with ASDF is a work in progress.  This section
described the current integration, the ways we expect it to change,
and a less-flexible and lower-level, but likely more stable,
alternative integration technique.

\subsection{NST's ASDF systems}
From version~1.2.2, the system \texttt{:asdf-nst} provides two classes
for ASDF system definitions, \texttt{asdf:nst-test-runner} and
\texttt{asdf:nst-test-holder}.

Up to NST~1.2.1 \texttt{:asdf-nst} provided a single class
\texttt{asdf:nst-testable}, and in the future we plan to reunify the
current two classes into a single class again.  However our first
implementation required NST to be loaded even when a system was
\emph{not} being tested, because we had no way to distinguish the
source code associated with testing from production code.  We plan to
solve this problem with a new file type \texttt{nst-file} in a future
version of NST.  This file type would \emph{not} be compiled or loaded
for the \texttt{compile-op} or \texttt{load-op} of the system, only
for its \texttt{test-op}\,.

\subsubsection{Test-running systems}
ASDF systems of the \texttt{asdf:nst-test-runner} class do not
themselves contain NST declarations in their source code, but may
identify other systems which do, and which should be tested as a part
of testing the given system.  These systems also allow local
definitions of NST's configuration for the execution of their tests.

Specify that a system runs NST tests by providing \texttt{:class
  asdf:nst-test-runner} argument to \texttt{asdf:defsystem}.  Use the
\texttt{:nst-systems} argument to name the systems which house the
actual unit tests:
\begin{itemize}
\item\texttt{:nst-systems ({\slshape system} {\slshape system}
    $\cdots$ {\slshape system})}\par Specifies a list of other systems
  which should be tested when testing this system.  These other
  systems do \emph{not} otherwise need to be identified as a
  dependency of this system (nor, for that matter, does \texttt{:nst}
  itself); they will be loaded upon \texttt{test-op} if they are not
  yet present.
\end{itemize}
Another optional argument to an \texttt{nst-test-runner} system
definition is:
\begin{itemize}
\item\texttt{:nst-init ({\slshape arg-list ...\ arg-list})}\par
  Initializing arguments to NST, to be executed after this system is
  loaded.  Each \texttt{arg-list} is passed as the arguments as if to
  a call to the \texttt{nst-cmd} macro.
\item\texttt{:nst-debug-config {\slshape form}}\par NST debugging
  customization for this system.  The \texttt{FORM} Should be an
  expression which, when evaluated, returns a list of keyword
  arguments; note that to give the list itself, it must be explicitly
  quoted, \emph{which is a change of behavior from pre-1.2.2
    versions}.
\item\texttt{:nst-debug-protect ({\slshape symbol ...\ symbol})} \par
  Gives a list of variables whose values should be saved before
  applying any configuration changes from \texttt{:nst-debug-config},
  and restored after testing.
\item\texttt{:nst-push-debug-config {\slshape t-or-nil}}\par If
  non-nil, then when this system is loaded its \texttt{:nst-debug} and
  \texttt{:nst-debug-protect} settings will be used as NST's defaults.
\end{itemize}

\subsubsection{Test-containing systems}
The \texttt{asdf:nst-test-holder} class is a subclass of
\texttt{nst-test-runner} for systems which are not only tested via
NST, but also contains NST tests in their source code.

Specify that a system defines NST tests by providing \texttt{:class
  asdf:nst-test-holder} to \texttt{asdf:defsystem}.  The arguments for
\texttt{asdf:nst-test-runner} may be used for
\texttt{asdf:nst-test-holder}, as well as the following:
\begin{itemize}
\item\texttt{:nst-packages ({\slshape package} {\slshape package}
    $\cdots$ {\slshape package})}\par When the system is tested, all
  groups and tests in the named packages should be run.
\item\texttt{:nst-groups (({\slshape package} {\slshape group})
    $\cdots$ ({\slshape package} {\slshape group}))}\par When the
  system is tested, tests in the named groups should be run.  Naming
  the package separately from the group and test in this argument (and
  in the similar arguments below) allows the group to be named before
  its package is necessarily defined.
\item\texttt{:nst-tests (({\slshape package} {\slshape group}
    {\slshape test}) $\cdots$ ({\slshape package} {\slshape group}
    {\slshape test}))}\par When the system is tested, all the named
  tests should be run.
\end{itemize}
The next three arguments to an \texttt{nst-testable} system are
mutually exclusive, and moreover exclude any of the above group or
\texttt{:nst-systems}\,:
\begin{itemize}
\item\texttt{:nst-package {\slshape package}}\par When the system is
  tested, all groups and tests in the named package should be run.
\item\texttt{:nst-group ({\slshape package} {\slshape group})}\par
  When the system is tested, all tests in the named group should be
  run.
\item\texttt{:nst-test ({\slshape package} {\slshape group} {\slshape
      test})}\par When the system is tested, the given test should be
  run.
\end{itemize}
Figure~\ref{fig:asdf} gives examples of \texttt{nst-testable}
ASDF system definitions.
\begin{figure}
\begin{verbatim}
;; NST and its ASDF interface must be loaded
;; before we can process the defsystem form.
(asdf:oos 'asdf:load-op :asdf-nst)

(defsystem :mnst
    :class nst-test-holder
    :description "The NST test suite's self-test."
    :serial t
    :nst-systems (:masdfnst)
    :nst-groups ((:mnst-simple . g1)
                 (:mnst-simple . g1a)
                 (:mnst-simple . g1a1)
                 (:mnst-simple . core-checks))
    :depends-on (:nst)
    :in-order-to ((test-op (load-op :mnst)))
    :components ((:module "core"
                  :components ((:file "byhand")
                               (:file "builtin-checks")))))
\end{verbatim}
\caption{Definitions of \texttt{nst-testable} ASDF systems.}
\label{fig:asdf}
\end{figure}

\subsection{An alternate ASDF integration technique}
We plan to deprecate and then remove \texttt{asdf:nst-test-holder} and
\texttt{nst-test-runner} once we have implemented a unified
replacement for them.  To avoid the possibility of a bit-rotted test
scheme, the link between a system and its unit tests can be made
explicit by providing methods for ASDF generic functions which make
calls to the NST API.  Specifically:
\begin{itemize}
\item A method of the ASDF \texttt{asdf:perform} generic function
  specialized to the \texttt{asdf:test-op} operation and the system in
  question will be executed to test a system.  So an appropriate
  method definition would begin:
  \begin{verbatim}
  (defmethod asdf:perform ((op asdf:test-op)
                           (sys (eql (asdf:find-system
                                        :SYSTEM-NAME))))
\end{verbatim}
\item NST API functions for running tests are:
\begin{tightlist}
\item\texttt{nst:run-package}
\item\texttt{nst:run-group}
\item\texttt{nst:run-test}
\end{tightlist}
\item The main NST API function for printing the results of testing is
  \texttt{asdf:report-multiple}\,.  In situations where only a single
  package, group or test is associated with a system, one of the
  following function may be more convenient:
  \begin{tightlist}
  \item\texttt{nst:report-package}
  \item\texttt{nst:report-group}
  \item\texttt{nst:report-test}
  \end{tightlist}
\end{itemize}
When providing an explicit \texttt{asdf:perform} method, it is also
necessary to explicitly list system dependencies to NST and to the
other systems which contain the tested system's unit test definitions.


%%% Local Variables: 
%%% mode: latex
%%% TeX-master: "manual"
%%% TeX-PDF-mode: t
%%% End: 
