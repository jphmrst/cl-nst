
\section{Verifying invariants against sampled data}
\label{quickcheck}
The \texttt{:sample} criterion provides random generation of data for
validating program properties.\footnote{This feature appears first in
  NST version 1.1.2.}  Our approach is based on Claessen and Hughes's
Quickcheck\footnote{Koen Claessen and John Hughes, ``QuickCheck: a
  lightweight tool for random testing of Haskell programs,'' from
  \emph{Proceedings of the International Conference on Functional
    Programming}, 2000.  QuickCheck papers, code and other resources
  are available at \textsl{www.cs.chalmers.se/\~{}rjmh/QuickCheck}~.}.

This style of testing is somewhat more complicated to than specific
tests on single, bespoke forms.  There are two distinct efforts, which
we address in the next two sections: describing how the sample data
can be generating, and specifying the test itself.

\subsection{Generating sample data}
\label{sec:quickcheck-data}
Data generation is centered around the generic function
\texttt{arbitrary}.  This function takes a single argument, which
determines the type of the value to be generated.  For simple types,
the name of the type (or the class object, such as returned by
\texttt{find-class}) by itself is a complete specification.  For more
complicated types, \texttt{arbitrary} can also take a list argument,
where the first element gives the type and the remaining elements are
keyword argument providing additional requirements for the generated
value.

NST provides method of \texttt{arbitrary} for many standard Lisp
types, listed in Table~\ref{table:built-in-arbitrary-types}.  Types in
the first column --- the standard numeric types plus the common
supertype \texttt{t} --- are not associated with additional keyword
arguments.
\begin{verbatim}
  (nst:arbitrary t)
  (nst:arbitrary 'complex)
  (nst:arbitrary 'integer)
  (nst:arbitrary 'ratio)
  (nst:arbitrary 'single-float)
\end{verbatim}
Keyword arguments for other NST-provided type specifiers
are as follows:
\begin{table}
\begin{center}
  \texttt{
    \begin{tabular}{lll|ll||l}
      \multicolumn{5}{c}{\textrm{\textbf{Standard Lisp types}}}
          & \multicolumn{1}{c}{\textrm{\textbf{Other types}}}
      \\ \hline
         number  & character & symbol & cons   & hash-table & scalar
      \\ real & string    &        & list   & &
      \\ rational &       &        & vector & &
      \\ integer  &       &        & array  & &
      \\ float &&&&t&
      \\ fixnum &&&& &
      \\ bignum &&&& &
      \\ ratio &&&& &
      \\ \multicolumn{2}{l}{short-float\footnotemark} &&& &
      \\ \multicolumn{2}{l}{single-float} &&& &
      \\ \multicolumn{2}{l}{double-float\addtocounter{footnote}{-1}\footnotemark} &&& &
      \\ \multicolumn{2}{l}{long-float} &&& &
      \\ complex &&&& &
      \\ \cline{1-3} \multicolumn{3}{c}{\textrm{Considered \texttt{scalar}}}
    \end{tabular}}
\end{center}
\caption{NST provides methods of generic function \texttt{arbitrary} generating values of the types in this table.}
\label{table:built-in-arbitrary-types}
\end{table}
\footnotetext{Not available on Allegro Lisp.}

\begin{itemize}
\item Types \texttt{character} and \texttt{string}:
  \begin{itemize}
  \item Argument \texttt{noncontrol}.  Excludes the control characters
    associated with ASCII code 0 through 31.
  \item Argument \texttt{range}.  Allows the range of characters to be
    restricted to a particular subset:
    \begin{center}
      \begin{tabular}{>{\ttfamily:}c|l}
        \multicolumn{1}{c|}{Value} & \multicolumn{1}{c}{Meaning}
        \\ \hline standard & Codes up to 96
        \\ ascii & Codes through 127
        \\ ascii-ext & Codes through 255
      \end{tabular}
    \end{center}
    Omitted or with any other value, characters with any code up to
    \texttt{char-code-limit} can result.  Examples:
\begin{verbatim}
  (nst:arbitrary 'character)
  (nst:arbitrary '(character :noncontrol t
                             :range :standard))
\end{verbatim}
  \end{itemize}
  
\item Type \texttt{symbol}:
  \begin{itemize}
  \item Argument \texttt{existing}.  If non-nil, requires that the
    result be a previously-interned symbol.
  \item Argument \texttt{exported}.  Requires that the result be not
    only a previously-interned symbol, but also one exported by its
    package.  Ignored if \texttt{:existing} is explicitly set to nil.
  \item Argument \texttt{package}.  Specifies the package from which
    the symbol will be generated.  If omitted, a package is selected
    at random from the existing ones.
  \item Argument \texttt{nonnull}.  If non-nil, allows
    \texttt{arbitrary} to ignore other restriction to guarantee
    returning a non-nil symbol.  When null, \texttt{arbitrary} may
    return nil.
  \item Argument \texttt{gensym}.  If non-nil, and if
    \texttt{:existing} is explicitly set to nil, returns a new
    uninterned symbol.
  \end{itemize}
  
\item Type \texttt{cons}:
  \begin{itemize}
  \item Arguments \texttt{car} and \texttt{cdr} should be additional
    type specifications, used direct the generation of respectively
    the left and right elements of the result.  Each defaults to
    \texttt{t}~.
  \end{itemize}
  
\item Types \texttt{list} and \texttt{vector}:
  \begin{itemize}
  \item Argument \texttt{length} specifies the length of the structure.
    If omitted, will be randomly generated.
  \item Argument \texttt{elem} directs the generation of the
    container's elements.  For both, the default element type is
    \texttt{t}~.
  \end{itemize}
  
\item Type \texttt{array}:
  \begin{itemize}
  \item Argument \texttt{elem}.  As for \texttt{list} and
    \texttt{vector}.
  \item Argument \texttt{dimens}.  Should be a list of nonnegative
    integers specifying the length of each dimension of the array.  If
    omitted, will be randomly generated.
  \item Argument \texttt{rank}.  Specifies the number of dimensions.
    If omitted but \texttt{:dimens} is given, will be set to the
    length of \texttt{:dimens}.  If both \texttt{:rank} and
    \texttt{:dimens} are omitted, then both are randomly generated.
  \end{itemize}
  
\item Type \texttt{hash-table}:
  \begin{itemize}
  \item Argument \texttt{size}.  Specifies the number of entries in
    the table.  If omitted, will be randomly generated.
  \item Argument \texttt{test}.  Specifies the hash table's test
    function.  If omitted, will be randomly selected from \texttt{eq},
    \texttt{eql}, \texttt{equal} and \texttt{equalp}.
  \item Arguments \texttt{key} and \texttt{val} direct the generation
    of the table's keys and values, respectively.  For the keys, the
    default element type is \texttt{t} when the test function is
    \texttt{eq} or \texttt{eql}, and \texttt{scalar} otherwise.  For
    the values, the default element type is \texttt{t}~.
  \end{itemize}
\end{itemize}

Beyond those standard Lisp types, NST provides the type
\texttt{scalar} as a supertype of the numeric types,
\texttt{character}, \texttt{string} and \texttt{symbol}.  Users may
extend this definition to include additional type specifications, as
we discuss below.  Types are not associated with \texttt{scalar} are
referred to as \emph{compound} (although there is no corresponding
type specification).  To avoid generating structures too large to hold
in memory, NST provides the global variable
\texttt{*max-compound-structure-depth*}\label{max-compound-structure-depth}
which sets the maximum nesting depth of compound data structures:
beyond that depth, \texttt{scalar} rather than \texttt{t} is the
default element generator.  This restriction does not apply to
explicitly specified element types, only to the use of defaults.

New type specifications are defined with the
\texttt{def-arbitrary-instance-type}
macro.
\label{def-arbitrary-instance-type}
\begin{verbatim}
  (def-arbitrary-instance-type (SPECIFICATION-NAME
                                 [ :key KEYWORD-PARAMS ]
                                 [ :scalar BOOL ] )
    FORM
    ...
    FORM)
\end{verbatim}
When a non-null value is provided for the \texttt{:scalar} argument,
the new specifier is taken to be generable by the \texttt{scalar}
specification.
\begin{verbatim}
  (def-arbitrary-instance-type (ratio :scalar t)
    (/ (arbitrary 'integer)
       (let ((raw (arbitrary (find-class 'integer))))
         (cond
           ((< raw 0) raw)
           (t (+ 1 raw))))))
\end{verbatim}
The \texttt{:key} argument gives a list of keyword arguments which may
accompany the new specification.  For the \texttt{cons} type, keyword
arguments allow specifications for the left and right components:
\begin{verbatim}
  (def-arbitrary-instance-type (cons :key ((car t car-supp-p)
                                           (cdr t cdr-supp-p)))
    (compound-structure
     (when (and (not car-supp-p)
                (>= *current-compound-structure-depth*
                    *max-compound-structure-depth*))
       (setf car 'scalar))
     (when (and (not cdr-supp-p)
                (>= *current-compound-structure-depth*
                    *max-compound-structure-depth*))
       (setf cdr 'scalar))
     (cons (arbitrary car) (arbitrary cdr))))
\end{verbatim}

\subsection{Invariants as tests}
\label{sec:quickcheck-tests}
Invariants to be tested, and the domains over which they range, are
specified in the \texttt{:sample} criterion:
\begin{verbatim}
  (:sample [ :value LAMBDA-LIST ]
           [ :domains ((NAME SPEC) ... (NAME SPEC)) ]
           [ :where FORM ]
           :verify FORM
           [ :sample-size NUMBER ]
           [ :qualifying-sample NUMBER ]
           [ :max-tries NUMBER ] )
\end{verbatim}
The \texttt{verify} argument is the expression to be (repeatedly)
evaluated, and which is expected always to return a non-null value.
This is the sole required argument, although in any particular use it
is unlikely to be the only argument given.  The other arguments are:
\begin{itemize}
\item The \texttt{domains} argument declares the variables in the
  \texttt{verify} expression which are to be given multiple randomized
  values.  The default value is \texttt{nil}, denoting an empty list.
\item The \texttt{value} argument is a lambda list to which the values
  given by the argument form should be applied.  The default value is
  \texttt{nil}, denoting no such arguments.
\item The \texttt{where} argument is a condition which determines the
  validity of the input argument.  For example, the condition would
  assert that a number is positive in an application where a negative
  value would be known to cause a failure.  The default value is
  \texttt{t}, allowing any values.
\item The \texttt{sample-size} argument gives the base specification
  of the number of value sets which will be generated.  Two further
  arguments have some bearing on the number of generation attempts
  when the \texttt{where} argument is non-\texttt{t}.  The
  \texttt{qualifying-sample} argument gives the minimum acceptable
  size of actual tested values, not counting sets rejected via the
  \texttt{where} expression.  The \texttt{max-tries} argument gives
  the maximum number of value sets to be generated.
\end{itemize}

Examples:
\begin{verbatim}
  (:sample :sample-size 10
           :domains ((x (list :elem symbol)))
           :verify (equal x (reverse (reverse x))))
\end{verbatim}

\begin{verbatim}
  (:sample :domains ((x real))
           :where (> x 1)
           :verify (< (sqrt x) x)
           :sample-size 10
           :max-tries 12)
\end{verbatim}

\begin{verbatim}

\end{verbatim}

%%% Local Variables: 
%%% mode: latex
%%% TeX-master: "manual"
%%% TeX-PDF-mode: t
%%% End: 
