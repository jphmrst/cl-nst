%% File manual.tex
%%
%% NST by John Maraist, based on RRT by Robert Goldman.
%%
%% NST is Copyright (c) 2006-2009 Smart Information Flow Technologies.
%% RRT is Copyright (c) 2005 Robert Goldman, released under the LGPL,
%% and the Lisp-specific preamble to that license.

\documentclass{article}
\usepackage{array}
\usepackage{makeidx}
\usepackage{tightlist}
\usepackage[pdftex]{hyperref}
\title{NST 1.2 users' manual}
\setcounter{tocdepth}{2}
\makeindex
\parskip 0.6em
\parindent 0pt

\begin{document}
\maketitle
\thispagestyle{empty}
\parskip -0.5em

\noindent This document is the manual and users' guide to the NST test
framework.  NST is a unit test system for Common Lisp which provides
support for test fixture data, stateful setup and cleanup of tests,
grouping of tests, and (we think!) a useful runtime interface.
Suggestions and comments are welcome; a list of known bugs and
infelicities concludes this document.  The files in the NST
distribution's \texttt{self-test} directory, especially
\texttt{self-test/core/builtin-checks.lisp}, holds the NST tests for
NST and contain many examples (some of which we have adapted for this
manual).

\tableofcontents
\parskip 0.6em
\parindent 0pt

\paragraph{Contributors.}
The primarily author of both NST and this manual is John
Maraist\footnote{Smart Information Flow Technologies, 211 North First
  Street, Suite 300, Minneapolis, MN 55401; \textsl{jmaraist} at
  \textsl{sift.info}.}.  Robert P.\ Goldman provided guidance,
comments and suggestions through the development.  Other contributors:
Mike Pelican, Steve Harp and Michael Atighetchi.
\cleardoublepage
\addtocontents{toc}{\setcounter{tocdepth}{1}}

\section{Fixtures}
\label{fixtures}
Fixtures\index{fixtures} are data structures and values which may be
referred to by name during testing.  NST provides the ability to use
fixtures across multiple tests and test groups, and to inject fixtures
into the runtime namespace for debugging.
A set of fixtures is defined using the \texttt{def-fixtures}
macro:\index{def-fixtures@\texttt{def-fixtures}}
\begin{verbatim}
  (def-fixtures FIXTURE-NAME
          ([ :uses USES ]
           [ :assumes ASSUMES ]
           [ :outer OUTER ]
           [ :inner INNER ]
           [ :documentation DOCUMENTATION ]
           [ :cache FLAG ]
           [ :export-names FLAG ]
           [ :export-fixture-name FLAG ]
           [ :export-bound-names FLAG ])
    ([ ([ :cache FLAG ]) ] NAME FORM)
    ([ ([ :cache FLAG ]) ] NAME FORM)
    ...
    ([ ([ :cache FLAG ]) ] NAME FORM))
\end{verbatim}

%%% Local Variables: 
%%% mode: latex
%%% TeX-master: "../manual/manual"
%%% TeX-PDF-mode: t
%%% End: 

%
\texttt{FIXTURE-NAME} is the name to be associated with this set of
fixtures.  \texttt{USES} is a list of the names of other fixture sets
which this declaration assumes to be available.  This declaration is
optional, but will supress some warnings.  \texttt{OUTER} and
\texttt{INNER} are lists of declarations to be included in a
\texttt{declare} statement respectively outside and inside of the
\texttt{let}-binding in which the \texttt{BINDINGS} are applied.
\texttt{DOCUMENTATION} describes the fixture set.

When a fixture is attached to a test or test group, each \texttt{NAME}
defined in that fixture becomes available in the body of that test or
group as if \texttt{let*}-bound to the corresponding \texttt{FORM}.  A
fixture in one set may refer back to other fixtures in the same set
(again \emph{\`a la} \texttt{let*}) but forward references are not
allowed.

The names of a fixture and the names it binds can be exported from the
package where the fixture is defined using the
\texttt{export-bound-names} and \texttt{export-fixture-name}
arguments.  The default value of both is \texttt{nil} unless a
non-\texttt{nil} value is provided for \texttt{export-names}.

The \texttt{cache} option, if non-nil, directs NST to evaluate a
fixture's form one single time, and re-use the resulting value on
subsequent applications of the fixture.  Note that if this value is
mutated by the test cases, test behavior may become unpredictable!
However this option can considerably improve performance when
constant-valued fixtures are applied repeatedly.  Caching may be set
on or off (the default is off) for the entire fixture set, and the
setting may vary for individual fixtures.

Examples of fixture definitions:
\begin{verbatim}
  (def-fixtures f1 ()
    (c 3)
    (d 'asdfg))
  (def-fixtures f2 (:uses (f1))
    (d 4)
    (e 'asdfg)
    (f c))
  (def-fixtures f3 ()
    ((:cache t)   g (ackermann 1 2))
    ((:cache nil) h (factorial 5)))
\end{verbatim}

%%% Local Variables: 
%%% mode: latex
%%% TeX-master: "manual"
%%% End: 


\section{Test groups}
\label{sec:groups}
Groups of tests\index{group}\index{test group|see{group}} can be
associated with fixture sets, stateful initiatization, and stateful
cleanup.  The syntax of a test group declaration
is:\index{def-test-group@\texttt{def-test-group}}
\begin{verbatim}
  (def-test-group NAME (FIXTURE FIXTURE ... FIXTURE)
    [ (:setup FORM FORM ... FORM) ]
    [ (:cleanup FORM FORM ... FORM) ]
    [ (:startup FORM FORM ... FORM) ]
    [ (:finish FORM FORM ... FORM) ]
    [ (:each-setup FORM FORM ... FORM) ]
    [ (:each-cleanup FORM FORM ... FORM) ]
    [ (:documentation STRING) ]
    TEST
    TEST
    ...
    TEST)
\end{verbatim}

%%% Local Variables: 
%%% mode: latex
%%% TeX-master: "../manual/manual"
%%% TeX-PDF-mode: t
%%% End: 

%
\texttt{NAME} is the name of this test group.  The \texttt{FIXTURE}s
are to be applied to the tests in this group.

The \texttt{:setup} forms\index{setup@\texttt{:setup}} are run after
inclusion of names from fixture sets, but before any tests from the
group.  Individual tests should make no assumptions as to whether the
setup is unique to that test, or whether it is shared among several
tests of that group.  

The \texttt{:cleanup} forms\index{cleanup@\texttt{:cleanup}} are
normally run after the setup completes; however the cleanup form will
not be run if the setup form raises an error.  When the user asks the
runtime system to enter the debugger on an error, the cleanup form
will not run unless the user explicitly enabled a resumption of the
test routine from the debugger.  The cleanup form will be run in other
circumstances, including at a user-requested break in testing at
either failure or error.

The
\texttt{:fixtures-setup}\index{fixtures-setup@\texttt{:fixtures-setup}}
(respectively
\texttt{:fixtures-cleanup})\index{fixtures-cleanup@\texttt{:fixtures-cleanup}}
form is run before fixtures are bound (after their bindings are
released).  These forms are useful, for example, to initialize a
database connection from which the fixture values are drawn.

The \texttt{:each-setup}\index{each-cleanup@\texttt{:each-cleanup}}
and \texttt{:each-cleanup}\index{each-cleanup@\texttt{:each-cleanup}}
forms are run before each test, rather than once for the group.

The \texttt{:documentation}\index{documentation@\texttt{:documentation}}
form sets the docstring for the class.

%%% Local Variables: 
%%% mode: latex
%%% TeX-master: "manual"
%%% TeX-PDF-mode: t
%%% End: 


\section{Tests and test criteria}
\label{sec:deftest}
Individual unit tests are encoded with the \texttt{def-test} form:
\begin{verbatim}
  (def-test (NAME [ :group GROUP-NAME ]
                  [ :setup FORM ]
                  [ :cleanup FORM ]
                  [ :fixtures (FIXTURE FIXTURE ... FIXTURE) ]
                  [ :documentation STRING ] )
      criterion
    FORM FORM ... FORM)

  (def-test NAME
      criterion
    FORM FORM ... FORM)
\end{verbatim}

%
The \texttt{SETUP}, \texttt{CLEANUP} and \texttt{FIXTURES} are just as
for test groups, but apply only to the one test.  The
\texttt{CRITERION} is a list or symbol specifying the properties which
should hold for the \texttt{FORM}s.

When there are no \texttt{SETUP}, \texttt{CLEANUP} or
\texttt{FIXTURES} arguments, the \texttt{NAME} may be given without
parentheses.  Likewise, any criterion consisting of a single symbol,
e.g.\ \texttt{(:pass)}, may be abbreviated as just the symbol without
the parentheses, e.g.\ \texttt{:pass}.

The \texttt{def-check} form is a deprecated synonym for
\texttt{def-test}.

\def\criteriaGroup#1#2{\subsection{#1} #2}
\def\secText#1{#1}
\def\endcriteriaGroup{}
\def\manualOrRef#1#2{#1}
\def\criteriaDoc#1#2#3#4#5#6#7#8{%
\subsubsection{The \texttt{#2} criterion} 
#4\index{#1@\texttt{#2}}
\\ Syntax: \texttt{#3}
#7{#8}#6}
\def\noEx#1{}
\def\singleEx#1{\\Example: \texttt{#1}}
\def\multiEx#1{#1}
\def\passExample#1{\\Passing example: \texttt{#1}}
\def\failExample#1{\\Failing example: \texttt{#1}}
\def\tabbingEx#1{\\Example: \begin{tabbing}#1\end{tabbing}}
\def\noExpl{}
\def\hasExpl#1{\par#1}

%% \criteriaDoc
%%    1 - criteria name without punctuation
%%    2 - criteria name with punctuation
%%    3 - syntax
%%    4 - explanation
%%    5 - applicable to
%%    6 - example type --- \noEx, \singleEx, \multiEx, \tabbingEx
%%    7 - example(s)

%% \kwdCriteriaDoc
%%    1 - criteria name
%%    2 - syntax
%%    3 - explanation
%%    4 - applicable to
%%    5 - example type --- \noEx, \singleEx, \multiEx, \tabbingEx
%%    6 - example(s)

\def\kwdCriteriaDoc#1#2#3#4#5#6{\criteriaDoc{#1}{:#1}{#2}{#3}{#4}{#5}{#6}}

\begin{criteriaGroup}{Basic checks}

\kwdCriteriaDoc{pass}{:pass}
{A trivial test, which always passes.}
{Any.}
{\singleEx}{(def-check passing-test :pass 3 4 "sd")}

\kwdCriteriaDoc{fail}{(:fail FORMAT ARG ...\ ARG)}
{A trivial test, which always fails.  The format string and arguments
should be suitable for the Lisp \texttt{format} function.}
{Any.}
{\singleEx}{(def-check fails (:fail "Expected a \~{}a" "string") 312)}

\end{criteriaGroup}


%%% Local Variables: 
%%% mode: latex
%%% TeX-master: "manual"
%%% End: 


%% \criteriaDoc
%%    1 - criteria name without punctuation
%%    2 - criteria name with punctuation
%%    3 - syntax
%%    4 - explanation
%%    5 - applicable to
%%    6 - example type --- \noEx, \singleEx, \multiEx, \tabbingEx
%%    7 - example(s)

%% \kwdCriteriaDoc
%%    1 - criteria name
%%    2 - syntax
%%    3 - explanation
%%    4 - applicable to
%%    5 - example type --- \noEx, \singleEx, \multiEx, \tabbingEx
%%    6 - example(s)

\def\kwdCriteriaDoc#1#2#3#4#5#6{\criteriaDoc{#1}{:#1}{#2}{#3}{#4}{#5}{#6}}

\begin{criteriaGroup}{Basic checks}

\kwdCriteriaDoc{pass}{:pass}
{A trivial test, which always passes.}
{Any.}
{\singleEx}{(def-check passing-test :pass 3 4 "sd")}

\kwdCriteriaDoc{fail}{(:fail FORMAT ARG ...\ ARG)}
{A trivial test, which always fails.  The format string and arguments
should be suitable for the Lisp \texttt{format} function.}
{Any.}
{\singleEx}{(def-check fails (:fail "Expected a \~{}a" "string") 312)}

\end{criteriaGroup}


\section{Verifying invariants against sampled data}
\label{quickcheck}
The \texttt{:sample} criterion provides random generation of data for
validating program properties.  Our approach is based on Claessen and
Hughes's Quickcheck\footnote{Koen Claessen and John Hughes,
  ``QuickCheck: a lightweight tool for random testing of Haskell
  programs,'' from \emph{Proceedings of the International Conference
    on Functional Programming}, 2000.  QuickCheck papers, code and
  other resources are available at
  \textsl{www.cs.chalmers.se/\~{}rjmh/QuickCheck}~.}.

This style of testing is somewhat more complicated than specific tests
on single, bespoke forms.  There are two distinct efforts, which we
address in the next two sections: describing how the sample data is to
be generated, and specifying the test itself.

\subsection{Generating sample data}
\label{sec:quickcheck-data}
Data generation is centered around the generic function
\texttt{arbitrary}.  This function takes a single argument, which
determines the type of the value to be generated.  For simple types,
the name of the type (or the class object, such as returned by
\texttt{find-class}) by itself is a complete specification.  For more
complicated types, \texttt{arbitrary} can also take a list argument,
where the first element gives the type and the remaining elements are
keyword argument providing additional requirements for the generated
value.

NST provides method of \texttt{arbitrary} for many standard Lisp
types, listed in Table~\ref{table:built-in-arbitrary-types}.  Types in
the first column --- the standard numeric types plus the common
supertype \texttt{t} --- are not associated with additional keyword
arguments.
\begin{verbatim}
  (nst:arbitrary t)
  (nst:arbitrary 'complex)
  (nst:arbitrary 'integer)
  (nst:arbitrary 'ratio)
  (nst:arbitrary 'single-float)
\end{verbatim}
Keyword arguments for other NST-provided type specifiers
are as follows:
\begin{table}
\begin{center}
  \texttt{
    \begin{tabular}{lll|ll||l}
      \multicolumn{5}{c}{\textrm{\textbf{Standard Lisp types}}}
          & \multicolumn{1}{c}{\textrm{\textbf{Other types}}}
      \\ \hline
         number  & character & symbol & cons   & hash-table & scalar
      \\ real & string    &        & list   & &
      \\ rational &       &        & vector & &
      \\ integer  &       &        & array  & &
      \\ float &&&&t&
      \\ fixnum &&&& &
      \\ bignum &&&& &
      \\ ratio &&&& &
      \\ \multicolumn{2}{l}{short-float\footnotemark} &&& &
      \\ \multicolumn{2}{l}{single-float} &&& &
      \\ \multicolumn{2}{l}{double-float\addtocounter{footnote}{-1}\footnotemark} &&& &
      \\ \multicolumn{2}{l}{long-float} &&& &
      \\ complex &&&& &
      \\ \cline{1-3} \multicolumn{3}{c}{\textrm{Considered \texttt{scalar}}}
    \end{tabular}}
\end{center}
\caption{NST provides methods of generic function \texttt{arbitrary} generating values of the types in this table.}
\label{table:built-in-arbitrary-types}
\end{table}
\footnotetext{Not available on Allegro Lisp.}

\begin{itemize}
\item Types \texttt{character} and \texttt{string}:
  \begin{itemize}
  \item Argument \texttt{noncontrol}.  Excludes the control characters
    associated with ASCII code 0 through 31.
  \item Argument \texttt{range}.  Allows the range of characters to be
    restricted to a particular subset:
    \begin{center}
      \begin{tabular}{>{\ttfamily:}c|l}
        \multicolumn{1}{c|}{Value} & \multicolumn{1}{c}{Meaning}
        \\ \hline standard & Codes up to 96
        \\ ascii & Codes through 127
        \\ ascii-ext & Codes through 255
      \end{tabular}
    \end{center}
    Omitted or with any other value, characters with any code up to
    \texttt{char-code-limit} can result.  Examples:
\begin{verbatim}
  (nst:arbitrary 'character)
  (nst:arbitrary '(character :noncontrol t
                             :range :standard))
\end{verbatim}
  \end{itemize}
  
\item Type \texttt{symbol}:
  \begin{itemize}
  \item Argument \texttt{existing}.  If non-nil, requires that the
    result be a previously-interned symbol.
  \item Argument \texttt{exported}.  Requires that the result be not
    only a previously-interned symbol, but also one exported by its
    package.  Ignored if \texttt{:existing} is explicitly set to nil.
  \item Argument \texttt{package}.  Specifies the package from which
    the symbol will be generated.  If omitted, a package is selected
    at random from the existing ones.
  \item Argument \texttt{nonnull}.  If non-nil, allows
    \texttt{arbitrary} to ignore other restriction to guarantee
    returning a non-nil symbol.  When null, \texttt{arbitrary} may
    return nil.
  \item Argument \texttt{gensym}.  If non-nil, and if
    \texttt{:existing} is explicitly set to nil, returns a new
    uninterned symbol.
  \end{itemize}
  
\item Type \texttt{cons}:
  \begin{itemize}
  \item Arguments \texttt{car} and \texttt{cdr} should be additional
    type specifications, used direct the generation of respectively
    the left and right elements of the result.  Each defaults to
    \texttt{t}~.
  \end{itemize}
  
\item Types \texttt{list} and \texttt{vector}:
  \begin{itemize}
  \item Argument \texttt{length} specifies the length of the structure.
    If omitted, will be randomly generated.
  \item Argument \texttt{elem} directs the generation of the
    container's elements.  For both, the default element type is
    \texttt{t}~.
  \end{itemize}
  
\item Type \texttt{array}:
  \begin{itemize}
  \item Argument \texttt{elem}.  As for \texttt{list} and
    \texttt{vector}.
  \item Argument \texttt{dimens}.  Should be a list of nonnegative
    integers specifying the length of each dimension of the array.  If
    omitted, will be randomly generated.
  \item Argument \texttt{rank}.  Specifies the number of dimensions.
    If omitted but \texttt{:dimens} is given, will be set to the
    length of \texttt{:dimens}.  If both \texttt{:rank} and
    \texttt{:dimens} are omitted, then both are randomly generated.
  \end{itemize}
  
\item Type \texttt{hash-table}:
  \begin{itemize}
  \item Argument \texttt{size}.  Specifies the number of entries in
    the table.  If omitted, will be randomly generated.
  \item Argument \texttt{test}.  Specifies the hash table's test
    function.  If omitted, will be randomly selected from \texttt{eq},
    \texttt{eql}, \texttt{equal} and \texttt{equalp}.
  \item Arguments \texttt{key} and \texttt{val} direct the generation
    of the table's keys and values, respectively.  For the keys, the
    default element type is \texttt{t} when the test function is
    \texttt{eq} or \texttt{eql}, and \texttt{scalar} otherwise.  For
    the values, the default element type is \texttt{t}~.
  \end{itemize}
\end{itemize}

Beyond those standard Lisp types, NST provides the type
\texttt{scalar} as a supertype of the numeric types,
\texttt{character}, \texttt{string} and \texttt{symbol}.  Users may
extend this definition to include additional type specifications, as
we discuss below.  Types are not associated with \texttt{scalar} are
referred to as \emph{compound} (although there is no corresponding
type specification).  To avoid generating structures too large to hold
in memory, NST provides the global variable
\texttt{*max-compound-structure-depth*}\label{max-compound-structure-depth}
which sets the maximum nesting depth of compound data structures:
beyond that depth, \texttt{scalar} rather than \texttt{t} is the
default element generator.  This restriction does not apply to
explicitly specified element types, only to the use of defaults.

New type specifications are defined with the
\texttt{def-arbitrary-instance-type}
macro.
\label{def-arbitrary-instance-type}
\begin{verbatim}
  (def-arbitrary-instance-type (SPECIFICATION-NAME
                                 [ :key KEYWORD-PARAMS ]
                                 [ :scalar BOOL ] )
    FORM
    ...
    FORM)
\end{verbatim}
When a non-null value is provided for the \texttt{:scalar} argument,
the new specifier is taken to be generable by the \texttt{scalar}
specification.
\begin{verbatim}
  (def-arbitrary-instance-type (ratio :scalar t)
    (/ (arbitrary 'integer)
       (let ((raw (arbitrary (find-class 'integer))))
         (cond
           ((< raw 0) raw)
           (t (+ 1 raw))))))
\end{verbatim}
The \texttt{:key} argument gives a list of keyword arguments which may
accompany the new specification.  For the \texttt{cons} type, keyword
arguments allow specifications for the left and right components:
\begin{verbatim}
  (def-arbitrary-instance-type (cons :key ((car t car-supp-p)
                                           (cdr t cdr-supp-p)))
    (compound-structure
     (when (and (not car-supp-p)
                (>= *current-compound-structure-depth*
                    *max-compound-structure-depth*))
       (setf car 'scalar))
     (when (and (not cdr-supp-p)
                (>= *current-compound-structure-depth*
                    *max-compound-structure-depth*))
       (setf cdr 'scalar))
     (cons (arbitrary car) (arbitrary cdr))))
\end{verbatim}

\subsection{Invariants as tests}
\label{sec:quickcheck-tests}
Invariants to be tested, and the domains over which they range, are
specified in the \texttt{:sample} criterion:
\begin{verbatim}
  (:sample [ :value LAMBDA-LIST ]
           [ :domains ((NAME SPEC) ... (NAME SPEC)) ]
           [ :where FORM ]
           :verify FORM
           [ :sample-size NUMBER ]
           [ :qualifying-sample NUMBER ]
           [ :max-tries NUMBER ] )
\end{verbatim}
The \texttt{verify} argument is the expression to be (repeatedly)
evaluated, and which is expected always to return a non-null value.
This is the sole required argument, although in any particular use it
is unlikely to be the only argument given.  The other arguments are:
\begin{itemize}
\item The \texttt{domains} argument declares the variables in the
  \texttt{verify} expression which are to be given multiple randomized
  values.  The default value is \texttt{nil}, denoting an empty list.
\item The \texttt{value} argument is a lambda list to which the values
  given by the argument form should be applied.  The default value is
  \texttt{nil}, denoting no such arguments.
\item The \texttt{where} argument is a condition which determines the
  validity of the input argument.  For example, the condition would
  assert that a number is positive in an application where a negative
  value would be known to cause a failure.  The default value is
  \texttt{t}, allowing any values.
\item The \texttt{sample-size} argument gives the base specification
  of the number of value sets which will be generated.  Two further
  arguments have some bearing on the number of generation attempts
  when the \texttt{where} argument is non-\texttt{t}.  The
  \texttt{qualifying-sample} argument gives the minimum acceptable
  size of actual tested values, not counting sets rejected via the
  \texttt{where} expression.  The \texttt{max-tries} argument gives
  the maximum number of value sets to be generated.
\end{itemize}

Examples:
\begin{verbatim}
  (:sample :sample-size 10
           :domains ((x (list :elem symbol)))
           :verify (equal x (reverse (reverse x))))
\end{verbatim}

\begin{verbatim}
  (:sample :domains ((x real))
           :where (> x 1)
           :verify (< (sqrt x) x)
           :sample-size 10
           :max-tries 12)
\end{verbatim}

\begin{verbatim}

\end{verbatim}

%%% Local Variables: 
%%% mode: latex
%%% TeX-master: "manual"
%%% TeX-PDF-mode: t
%%% End: 

\section{The runtime system}
The runtime system provides several operations for scheduling and
running tests, and debugging failing and erring tests.  The operations
are accessible from
the~\texttt{nst-cmd}\index{nst-cmd@\texttt{nst-cmd}} macro.  Under
Allegro, the top-level alias \texttt{:nst}\index{nst@\texttt{:nst}}
provides a shorthand to this function; for the sake of brevity we use
this shorthand below.

The \texttt{:help} command gives a complete inventory of runtime
system commands:\index{help@\texttt{:help}}
\begin{verbatim}
 :nst :help
 (nst-cmd :help)
\end{verbatim}

There are three commands for running tests:
\begin{itemize}
\item\texttt{:nst :run-package {\itshape package-name}}\index{run-package@\texttt{:run-package}}
\par Run all tests defined in groups under the given package.
\item\texttt{:nst :run-group {\itshape group-name}}\index{run-group@\texttt{:run-group}}
\par Run all tests in the given group.
\item\texttt{:nst :run-test {\itshape group-name} {\itshape test-name}}\index{run-test@\texttt{:run-test}}
\par Run the named test.
\end{itemize}

There are two commands for (re)printing the results of tests:
\begin{itemize}
\item\texttt{:nst :report}\index{report@\texttt{:report}}
\\ \texttt{:nst :report {\itshape package-name}}
\\ \texttt{:nst :report {\itshape group-name}}
\\ \texttt{:nst :report {\itshape group-name} {\itshape test-name}}
\item\texttt{:nst :detail}\index{detail@\texttt{:detail}}
\\ \texttt{:nst :detail {\itshape package-name}}
\\ \texttt{:nst :detail {\itshape group-name}}
\\ \texttt{:nst :detail {\itshape group-name} {\itshape test-name}}
\end{itemize}
The \texttt{:report} command summarizes successes, failures and
errors; the \texttt{:detail} command gives more detailed information
about individual tests.

The \texttt{:clear} command erases NST's internal record of test
results.

The \texttt{:set} and \texttt{:unset} commands adjust NST's
configuration.
\begin{itemize}
\item\texttt{:nst :set {\itshape property} {\itshape value}}\index{set@\texttt{:set}}
\item\texttt{:nst :unset {\itshape property} {\itshape value}}\index{unset@\texttt{:unset}}
\end{itemize}
There are currently two properties which can be manipulated by \texttt{:set}\,:
\begin{itemize}
\item\texttt{:verbose}\index{verbose@\texttt{:verbose}}~~ Controls the
  level of output at various points of NST.  Valid settings are:
  \begin{tightlist}
  \item\texttt{:silent} (aka \texttt{nil})
  \item\texttt{:quiet} (aka \texttt{:default})
  \item\texttt{:verbose} (aka \texttt{t})
  \item\texttt{:vverbose}
  \end{tightlist}

  The \texttt{:report} and \texttt{:detail} commands operate by
  setting minimum levels of verbosity.

\item\texttt{:debug-on-error}\index{debug-on-error@\texttt{:debug-on-error}}~~
  When this property has a non-nil value, NST will exit into the
  debugger when it catches an error.

  The \texttt{:debug} command is a short-cut for setting this
  property.\index{debug@\texttt{:debug}}
\end{itemize}

Fixtures\index{fixtures!debugging} can be \emph{opened} into the
interactive namespace for debugging with the
\texttt{:nst~:open}\index{open@\texttt{:open}} command:\index{open@\texttt{:open}}%
\\ Syntax: \texttt{:nst :open FIXTURE-NAME FIXTURE-NAME ... FIXTURE-NAME}
\\ Example:
\begin{verbatim}
  CL-USER(75): (nst:def-fixtures small-fixture ()
                  (fix-var1 3)
                  (fix-var2 'asdfg))
  NIL
  CL-USER(76): (boundp 'fix-var1)
  NIL
  CL-USER(77): :nst :open small-fixture
  Opened fixture SMALL-FIXTURE.
  CL-USER(78): fix-var1
  3
  CL-USER(79): 
\end{verbatim}

Calling \texttt{:nst} or \texttt{(nst-cmd}) without a command argument
repeats the last test-executing command.

%%% Local Variables: 
%%% mode: latex
%%% TeX-master: "manual"
%%% End: 

\section{Integration with ASDF}
NST's integration with ASDF is a work in progress.  This section
described the current integration, the ways we expect it to change,
and a less-flexible and lower-level, but likely more stable,
alternative integration technique.

\subsection{NST's ASDF systems}
From version~1.2.2, the system \texttt{:asdf-nst} provides two classes
for ASDF system definitions, \texttt{asdf:nst-test-runner} and
\texttt{asdf:nst-test-holder}.

Up to NST~1.2.1 \texttt{:asdf-nst} provided a single class
\texttt{asdf:nst-testable}, and in the future we plan to reunify the
current two classes into a single class again.  However our first
implementation required NST to be loaded even when a system was
\emph{not} being tested, because we had no way to distinguish the
source code associated with testing from production code.  We plan to
solve this problem with a new file type \texttt{nst-file} in a future
version of NST.  This file type would \emph{not} be compiled or loaded
for the \texttt{compile-op} or \texttt{load-op} of the system, only
for its \texttt{test-op}\,.

\subsubsection{Test-running systems}
ASDF systems of the \texttt{asdf:nst-test-runner} class do not
themselves contain NST declarations in their source code, but may
identify other systems which do, and which should be tested as a part
of testing the given system.  These systems also allow local
definitions of NST's configuration for the execution of their tests.

Specify that a system runs NST tests by providing \texttt{:class
  asdf:nst-test-runner} argument to \texttt{asdf:defsystem}.  Use the
\texttt{:nst-systems} argument to name the systems which house the
actual unit tests:
\begin{itemize}
\item\texttt{:nst-systems ({\slshape system} {\slshape system}
    $\cdots$ {\slshape system})}\par Specifies a list of other systems
  which should be tested when testing this system.  These other
  systems do \emph{not} otherwise need to be identified as a
  dependency of this system (nor, for that matter, does \texttt{:nst}
  itself); they will be loaded upon \texttt{test-op} if they are not
  yet present.
\end{itemize}
Another optional argument to an \texttt{nst-test-runner} system
definition is:
\begin{itemize}
\item\texttt{:nst-init ({\slshape arg-list ...\ arg-list})}\par
  Initializing arguments to NST, to be executed after this system is
  loaded.  Each \texttt{arg-list} is passed as the arguments as if to
  a call to the \texttt{nst-cmd} macro.
\item\texttt{:nst-debug-config {\slshape form}}\par NST debugging
  customization for this system.  The \texttt{FORM} Should be an
  expression which, when evaluated, returns a list of keyword
  arguments; note that to give the list itself, it must be explicitly
  quoted, \emph{which is a change of behavior from pre-1.2.2
    versions}.
\item\texttt{:nst-debug-protect ({\slshape symbol ...\ symbol})} \par
  Gives a list of variables whose values should be saved before
  applying any configuration changes from \texttt{:nst-debug-config},
  and restored after testing.
\item\texttt{:nst-push-debug-config {\slshape t-or-nil}}\par If
  non-nil, then when this system is loaded its \texttt{:nst-debug} and
  \texttt{:nst-debug-protect} settings will be used as NST's defaults.
\end{itemize}

\subsubsection{Test-containing systems}
The \texttt{asdf:nst-test-holder} class is a subclass of
\texttt{nst-test-runner} for systems which are not only tested via
NST, but also contains NST tests in their source code.

Specify that a system defines NST tests by providing \texttt{:class
  asdf:nst-test-holder} to \texttt{asdf:defsystem}.  The arguments for
\texttt{asdf:nst-test-runner} may be used for
\texttt{asdf:nst-test-holder}, as well as the following:
\begin{itemize}
\item\texttt{:nst-packages ({\slshape package} {\slshape package}
    $\cdots$ {\slshape package})}\par When the system is tested, all
  groups and tests in the named packages should be run.
\item\texttt{:nst-groups (({\slshape package} {\slshape group})
    $\cdots$ ({\slshape package} {\slshape group}))}\par When the
  system is tested, tests in the named groups should be run.  Naming
  the package separately from the group and test in this argument (and
  in the similar arguments below) allows the group to be named before
  its package is necessarily defined.
\item\texttt{:nst-tests (({\slshape package} {\slshape group}
    {\slshape test}) $\cdots$ ({\slshape package} {\slshape group}
    {\slshape test}))}\par When the system is tested, all the named
  tests should be run.
\end{itemize}
The next three arguments to an \texttt{nst-testable} system are
mutually exclusive, and moreover exclude any of the above group or
\texttt{:nst-systems}\,:
\begin{itemize}
\item\texttt{:nst-package {\slshape package}}\par When the system is
  tested, all groups and tests in the named package should be run.
\item\texttt{:nst-group ({\slshape package} {\slshape group})}\par
  When the system is tested, all tests in the named group should be
  run.
\item\texttt{:nst-test ({\slshape package} {\slshape group} {\slshape
      test})}\par When the system is tested, the given test should be
  run.
\end{itemize}
Figure~\ref{fig:asdf} gives examples of \texttt{nst-testable}
ASDF system definitions.
\begin{figure}
\begin{verbatim}
;; NST and its ASDF interface must be loaded
;; before we can process the defsystem form.
(asdf:oos 'asdf:load-op :asdf-nst)

(defsystem :mnst
    :class nst-test-holder
    :description "The NST test suite's self-test."
    :serial t
    :nst-systems (:masdfnst)
    :nst-groups ((:mnst-simple . g1)
                 (:mnst-simple . g1a)
                 (:mnst-simple . g1a1)
                 (:mnst-simple . core-checks))
    :depends-on (:nst)
    :in-order-to ((test-op (load-op :mnst)))
    :components ((:module "core"
                  :components ((:file "byhand")
                               (:file "builtin-checks")))))
\end{verbatim}
\caption{Definitions of \texttt{nst-testable} ASDF systems.}
\label{fig:asdf}
\end{figure}

\subsection{An alternate ASDF integration technique}
We plan to deprecate and then remove \texttt{asdf:nst-test-holder} and
\texttt{nst-test-runner} once we have implemented a unified
replacement for them.  To avoid the possibility of a bit-rotted test
scheme, the link between a system and its unit tests can be made
explicit by providing methods for ASDF generic functions which make
calls to the NST API.  Specifically:
\begin{itemize}
\item A method of the ASDF \texttt{asdf:perform} generic function
  specialized to the \texttt{asdf:test-op} operation and the system in
  question will be executed to test a system.  So an appropriate
  method definition would begin:
  \begin{verbatim}
  (defmethod asdf:perform ((op asdf:test-op)
                           (sys (eql (asdf:find-system
                                        :SYSTEM-NAME))))
\end{verbatim}
\item NST API functions for running tests are:
\begin{itemize}
\item\texttt{nst:run-package}
\item\texttt{nst:run-group}
\item\texttt{nst:run-test}
\end{itemize}
\item The main NST API function for printing the results of testing is
  \texttt{asdf:report-multiple}\,.  In situations where only a single
  package, group or test is associated with a system, one of the
  following function may be more convenient:
  \begin{itemize}
  \item\texttt{nst:report-package}
  \item\texttt{nst:report-group}
  \item\texttt{nst:report-test}
  \end{itemize}
\end{itemize}
When providing an explicit \texttt{asdf:perform} method, it is also
necessary to explicitly list system dependencies to NST and to the
other systems which contain the tested system's unit test definitions.


%%% Local Variables: 
%%% mode: latex
%%% TeX-master: "manual"
%%% TeX-PDF-mode: t
%%% End: 

\appendix
\addtocontents{toc}{\setcounter{tocdepth}{1}}

\section{The NST API}
\def\secpageref#1{ --- \S\ref{#1}, p.\,\pageref{#1}}

\subsection{Primary macros}
\texttt{def-fixtures}\secpageref{fixtures}.

\texttt{def-test-group}\secpageref{sec:groups}.

\texttt{def-test}\secpageref{sec:deftest}.

\texttt{def-check}\secpageref{sec:deftest} --- deprecated.

\texttt{def-criterion-alias}\secpageref{sec:def-criterion-alias}.

\texttt{def-check-alias}\secpageref{sec:def-criterion-alias} ---
deprecated.

\texttt{def-values-criterion}\secpageref{sec:def-values-criterion}.

\texttt{def-value-check}\secpageref{sec:def-values-criterion} ---
deprecated.

\texttt{def-form-criterion}\secpageref{sec:def-form-criterion}.

\texttt{def-control-check}\secpageref{sec:def-form-criterion} ---
deprecated.

\subsection{Functions used in criteria definitions}
% \texttt{continue-check}\secpageref{sec:subcriteria}.

\texttt{make-failure-report}\secpageref{sec:criteria-forms-report}.

\texttt{make-warning-report}\secpageref{sec:criteria-forms-report}.

\texttt{make-success-report}\secpageref{sec:criteria-forms-report}.

\texttt{emit-failure}\secpageref{sec:criteria-forms-report}.

\texttt{emit-warning}\secpageref{sec:criteria-forms-report}.

\texttt{emit-success}\secpageref{sec:criteria-forms-report}.

\texttt{add-failure}\secpageref{sec:criteria-forms-report}.

\texttt{add-error}\secpageref{sec:criteria-forms-report}.

\texttt{add-info}\secpageref{sec:criteria-forms-report}.

\texttt{check-result}\secpageref{sec:criteria-forms-report} ---
deprecated.
             
\subsection{NST control and JUnit XML output}
\texttt{nst-cmd}

\texttt{nst-junit-dump}\secpageref{sec:junit}.

\texttt{junit-results-by-group}\secpageref{sec:junit}.

\subsection{Programmatic control of testing and output}
Note that these functions are exported from the package
\texttt{:nst-control-api} --- they are not intended for regular use in
NST tests.

\texttt{run-package}

\texttt{run-group}

\texttt{run-test}
             
\texttt{report-multiple}

\texttt{report-package}

\texttt{report-group}

\texttt{report-test}

\texttt{protect-nst-config}

\texttt{apply-debug-options}

\subsection{User settings}
\texttt{*nst-output-stream*}

\texttt{*default-report-verbosity*}

\texttt{*debug-on-error*}

\subsection{Testing randomized samples}
\texttt{arbitrary}\secpageref{sec:quickcheck-data}

\texttt{compound-structure}

\texttt{def-arbitrary-instance-type}\secpageref{def-arbitrary-instance-type}.

\texttt{*max-compound-structure-depth*}\secpageref{max-compound-structure-depth}.

\subsection{Other}
\texttt{protect-nst-config}

\texttt{apply-debug-options}

\texttt{with-fixtures}

%%% Local Variables: 
%%% mode: latex
%%% TeX-master: "manual"
%%% TeX-PDF-mode: t
%%% End: 


\section{Output to JUnit}

NST reports can be formatted as XML for use with JUnit, although the
API for this feature is underdeveloped.  The
\texttt{junit-results-by-group} function aligns test groups with Java
classes, and individual tests with \texttt{@Test} methods.
\begin{verbatim}
  (junit-results-by-group :verbose VERBOSE
                          :dir DIR :file FILE
                          :stream STREAM
                          :if-dir-does-not-exist BOOL
                          :if-file-exists BOOL)
\end{verbatim}
Either \texttt{:dir} and \texttt{:file} options, or the
\texttt{:stream} option, but not both, should be used to specify the
target for XML output; if none of the three options are given, the
function will write to \texttt{*standard-output*}.

%%% Local Variables: 
%%% mode: latex
%%% TeX-master: "manual"
%%% End: 


\section{Issues, bugs and enhancements}\index{bugs}\index{enhancements}
\begin{enumerate}
\item There are still, sometimes, warnings from the use of fixture
  names in other fixtures.
\item We are not yet catching errors arising from
  \texttt{:cleanup}\index{cleanup@\texttt{:cleanup}} blocks.
\item Clobbering of test/group names should be caught by the macros,
  not by the lower-level expansions arising from those methods.
\item The def-test form \texttt{:perf} might be extended to provide
  (implementation-dependent) checks on memory limits as well as time
  limits.
\item Some sort of timeout mechanism --- perhaps
  implementation-dependent --- could be helpful for measuring
  correctness via termination in more complicated algorithms.
\item The JUnit implementation should be more tightly integrated with
  ASDF, allowing automatic generation of XML at each test.
\item NST uses Allegro CL-specific features to capture backtraces from
  errors in NST tests.  However, due to a bug in the Allegro for
  Macintosh, this feature is not safe on that platform.  Therefore, by
  default this feature is disabled on MacOS.  Users can re-enable
  backtrace capture by adding \texttt{:nst-unsafe-allegro-backtraces}
  to the \texttt{*features*} global user
  variable.\index{@nst-unsafe-allegro-backtraces\texttt{:nst-unsafe-allegro-backtraces}}\index{features@\texttt{*features*}}

  The bug arises in situations where the \texttt{:err} criterion
  expects to find one type of error, but in fact an error of a
  different type is thrown.  NST will hang and must be manually
  interrupted.  We have reported this issue to
  Franz.\index{err@\texttt{:err}}
\end{enumerate}

\subsection{Removed features}
\label{temp-removed}
Some features of the NST 0.9 have not yet been reimplemented in NST
1.0.  Tell John if one is urgent for you.

\subsubsection{Test group documentation}
The \texttt{def-test-group} form should allow group
documentation:\index{documentation@\texttt{:documentation}}.
\begin{verbatim}
    (:documentation DOC-STRING)
\end{verbatim}

\subsubsection{Alternate fixture definitions}
The
\texttt{def-capture/restore-fixtures}\index{def-capture/restore-fixtures@\texttt{def-capture/restore-fixtures}}
declaration binds \texttt{nil} to a collection of variables in the
extent of associated test groups.  This form is useful when hiding
some developers' state from tests intended to simulate a
non-development environment.
\begin{center}
\textbf{def-capture/restore-fixtures}
 \textit{name} \textit{variables}
\texttt{\&key} \textit{documentation}
\end{center}

\subsubsection{Anonymous fixture sets}
It is not necessary to name fixture sets; they may be given
anonymously in any situation where a fixture set name is allowed.  The
syntax of anonymous fixture sets
is:\index{fixtures@\texttt{:fixtures}}\index{fixtures!anonymous}
\begin{center}
\texttt{(:fixtures }\textit{(name form)}$^+$\texttt{ )}
\end{center}

\subsubsection{The \texttt{:with} check}
This check assumes that the next item is a list, whose contents are
expanded into the methods-and-forms.\index{with@\texttt{:with}} This
check is especially useful with the \texttt{def-test-cri\-ter\-ion}
command below.
\\ Syntax: \texttt{(:with \emph{criterion})}
\\ Example: The following two \texttt{def-test}s are equivalent.
\begin{verbatim}
  (def-test seq1
      (:seq (:predicate symbolp) (:eql 1) (:symbol d))
      '(a 1 d))
  (def-test with-seq1
      (:with (:seq (:predicate symbolp) (:eql 1) (:symbol d)))
      '(a 1 d))
\end{verbatim}

\subsubsection{Compilation deferral control}
In earlier versions the \texttt{:defer-compile} switch on test
definition allowed control over when an expression would be compiled.
\index{defer-compile@\texttt{:defer-compile}}

\subsubsection{In the runtime system}

\subsubsection{The \texttt{:summarize-scheduled} operation}
\index{summarize-scheduled@\texttt{:summarize-scheduled}}%
If \texttt{BOOL} evaluates to non-null, then the runtime system will
print a summary after running scheduled tests with \texttt{:run},
\texttt{:continue}, etc.
\\ Syntax: \texttt{:nst :summarize-scheduled BOOL}

\subsubsection{The \texttt{:summarize-single} operation}
\index{summarize-single@\texttt{:summarize-single}}%
If \texttt{BOOL} evaluates to non-null, then the runtime system will
print a summary after one-time test runs initiated by
\texttt{:run-test}, \texttt{:run-group}, etc.
\\ Syntax: \texttt{:nst :summarize-single BOOL}

\subsection{Marking tests of interest for execution}
\label{nominating-tests}
NST allows tests to be marked for execution by the \texttt{:run}
command.  Tests can be marked by package or group, or as an individual
test.

\subsubsection{The \texttt{:p} operation}
\index{p@\texttt{:p}}%
Indicates that all tests in a particular package should be run.
\\ Syntax: \texttt{:nst :p PACKAGE}

\subsubsection{The \texttt{:g} operation}
\index{g@\texttt{:g}}%
Indicates that all tests in a particular group should be run.
\\ Syntax: \texttt{:nst :g GROUP}

\subsubsection{The \texttt{:t} operation}
\index{t@\texttt{:t}}%
Indicates that the named test should be run.
\\ Syntax: \texttt{:nst :t TEST}

\subsection{Test definition}
\subsubsection{The \texttt{:defer-test-compile} operation}
\index{defer-test-compile@\texttt{:defer-test-compile}}%
Sets whether tests defined subsequently should, by default, defer
compilation of their forms until actually running the test.  This
feature is useful when debugging code involving macros, but changing
this feature in the runtime system can lead to confusion.  It is
surely almost always the right thing to set this flag locally via
def-test-group and def-test.  In fact, this operation may be removed
in a future version of the runtime system.
\\ Syntax: \texttt{:nst :defer-test-compile BOOL}

\subsubsection{The \texttt{:open{\textasteriskcentered}} operation}
\index{open{\textasteriskcentered}@\texttt{:open{\textasteriskcentered}}}%
Multiple fixtures\index{fixtures!debugging} can be \emph{opened} into the
interactive namespace with a single command using the
\texttt{:nst~:open} command:
\\ Syntax: \texttt{:nst :open\* } {\textit{FIXTURE-NAME}{\textasteriskcentered}}
\par
In earlier versions, only one fixture could be given to
\texttt{:open}; this is no longer true, and the \texttt{open*} command
has been removed.

\subsubsection{The \texttt{:open-used} operation}
\index{open-used@\texttt{:open-used}}%
If \texttt{BOOL} evaluates to non-null, then opening a fixture will
always also open the fixtures it uses.  Default is t.
\\ Syntax: \texttt{:nst :open-used BOOL}

\subsubsection{The \texttt{:reopen} operation}
\index{reopen@\texttt{:reopen}}%
If \texttt{BOOL} evaluates to non-null, then fixtures will be
re-opened \emph{e.g.}\ when required multiple times by opening
different fixtures that use them.
\\ Syntax: \texttt{:nst :reopen BOOL}

%%% Local Variables: 
%%% mode: latex
%%% TeX-master: "manual"
%%% End: 

\printindex
\end{document}

%%% Local Variables: 
%%% mode: latex
%%% TeX-master: t
%%% TeX-PDF-mode: t
%%% End: 
