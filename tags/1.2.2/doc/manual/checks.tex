
\section{Tests and test criteria}
\label{sec:deftest}
Individual unit tests are encoded with the \texttt{def-test} form:
\begin{verbatim}
  (def-test (NAME [ :group GROUP-NAME ]
                  [ :setup FORM ]
                  [ :cleanup FORM ]
                  [ :fixtures (FIXTURE FIXTURE ... FIXTURE) ]
                  [ :documentation STRING ] )
      criterion
    FORM FORM ... FORM)

  (def-test NAME
      criterion
    FORM FORM ... FORM)
\end{verbatim}

%
The \texttt{SETUP}, \texttt{CLEANUP} and \texttt{FIXTURES} are just as
for test groups, but apply only to the one test.  The
\texttt{CRITERION} is a list or symbol specifying the properties which
should hold for the \texttt{FORM}s.

When there are no \texttt{SETUP}, \texttt{CLEANUP} or
\texttt{FIXTURES} arguments, the \texttt{NAME} may be given without
parentheses.  Likewise, any criterion consisting of a single symbol,
e.g.\ \texttt{(:pass)}, may be abbreviated as just the symbol without
the parentheses, e.g.\ \texttt{:pass}.

The \texttt{def-check} form is a deprecated synonym for
\texttt{def-test}.

\def\criteriaGroup#1#2{\subsection{#1} #2}
\def\secText#1{#1}
\def\endcriteriaGroup{}
\def\manualOrRef#1#2{#1}
\def\criteriaDoc#1#2#3#4#5#6#7#8{%
\subsubsection{The \texttt{#2} criterion} 
#4\index{#1@\texttt{#2}}
\\ Syntax: \texttt{#3}
#7{#8}#6}
\def\noEx#1{}
\def\singleEx#1{\\Example: \texttt{#1}}
\def\multiEx#1{#1}
\def\passExample#1{\\Passing example: \texttt{#1}}
\def\failExample#1{\\Failing example: \texttt{#1}}
\def\tabbingEx#1{\\Example: \begin{tabbing}#1\end{tabbing}}
\def\noExpl{}
\def\hasExpl#1{\par#1}

%% \criteriaDoc
%%    1 - criteria name without punctuation
%%    2 - criteria name with punctuation
%%    3 - syntax
%%    4 - explanation
%%    5 - applicable to
%%    6 - example type --- \noEx, \singleEx, \multiEx, \tabbingEx
%%    7 - example(s)

%% \kwdCriteriaDoc
%%    1 - criteria name
%%    2 - syntax
%%    3 - explanation
%%    4 - applicable to
%%    5 - example type --- \noEx, \singleEx, \multiEx, \tabbingEx
%%    6 - example(s)

\def\kwdCriteriaDoc#1#2#3#4#5#6{\criteriaDoc{#1}{:#1}{#2}{#3}{#4}{#5}{#6}}

\begin{criteriaGroup}{Basic checks}

\kwdCriteriaDoc{pass}{:pass}
{A trivial test, which always passes.}
{Any.}
{\singleEx}{(def-check passing-test :pass 3 4 "sd")}

\kwdCriteriaDoc{fail}{(:fail FORMAT ARG ...\ ARG)}
{A trivial test, which always fails.  The format string and arguments
should be suitable for the Lisp \texttt{format} function.}
{Any.}
{\singleEx}{(def-check fails (:fail "Expected a \~{}a" "string") 312)}

\end{criteriaGroup}


%%% Local Variables: 
%%% mode: latex
%%% TeX-master: "manual"
%%% End: 
