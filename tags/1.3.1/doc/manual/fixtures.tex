\section{Fixtures}
\label{fixtures}
Fixtures\index{fixtures} are data structures and values which may be
referred to by name during testing.  NST provides the ability to use
fixtures across multiple tests and test groups, and to inject fixtures
into the runtime namespace for debugging.
A set of fixtures is defined using the \texttt{def-fixtures}
macro:\index{def-fixtures@\texttt{def-fixtures}}
\begin{verbatim}
  (def-fixtures FIXTURE-NAME
          ([ :uses USES ]
           [ :assumes ASSUMES ]
           [ :outer OUTER ]
           [ :inner INNER ]
           [ :documentation DOCUMENTATION ]
           [ :cache FLAG ]
           [ :export-names FLAG ]
           [ :export-fixture-name FLAG ]
           [ :export-bound-names FLAG ])
    ([ ([ :cache FLAG ]) ] NAME FORM)
    ([ ([ :cache FLAG ]) ] NAME FORM)
    ...
    ([ ([ :cache FLAG ]) ] NAME FORM))
\end{verbatim}

%%% Local Variables: 
%%% mode: latex
%%% TeX-master: "../manual/manual"
%%% TeX-PDF-mode: t
%%% End: 

%
\texttt{FIXTURE-NAME} is the name to be associated with this set of
fixtures.  \texttt{USES} is a list of the names of other fixture sets
which this declaration assumes to be available.  This declaration is
optional, but will supress some warnings.  \texttt{OUTER} and
\texttt{INNER} are lists of declarations to be included in a
\texttt{declare} statement respectively outside and inside of the
\texttt{let}-binding in which the \texttt{BINDINGS} are applied.
\texttt{DOCUMENTATION} describes the fixture set.

When a fixture is attached to a test or test group, each \texttt{NAME}
defined in that fixture becomes available in the body of that test or
group as if \texttt{let*}-bound to the corresponding \texttt{FORM}.  A
fixture in one set may refer back to other fixtures in the same set
(again \emph{\`a la} \texttt{let*}) but forward references are not
allowed.

The names of a fixture and the names it binds can be exported from the
package where the fixture is defined using the
\texttt{export-bound-names} and \texttt{export-fixture-name}
arguments.  The default value of both is \texttt{nil} unless a
non-\texttt{nil} value is provided for \texttt{export-names}.

The \texttt{cache} option, if non-nil, directs NST to evaluate a
fixture's form one single time, and re-use the resulting value on
subsequent applications of the fixture.  Note that if this value is
mutated by the test cases, test behavior may become unpredictable!
However this option can considerably improve performance when
constant-valued fixtures are applied repeatedly.  Caching may be set
on or off (the default is off) for the entire fixture set, and the
setting may vary for individual fixtures.

Examples of fixture definitions:
\begin{verbatim}
  (def-fixtures f1 ()
    (c 3)
    (d 'asdfg))
  (def-fixtures f2 (:uses (f1))
    (d 4)
    (e 'asdfg)
    (f c))
  (def-fixtures f3 ()
    ((:cache t)   g (ackermann 1 2))
    ((:cache nil) h (factorial 5)))
\end{verbatim}

The \texttt{with-fixtures} macro faciliates debugging and other
non-NST uses of fixtures sets:
\begin{verbatim}
  (with-fixtures (FIXTURE ... FIXTURE)
    FORM
    ...
    FORM)
\end{verbatim}
This macro evaluates the forms in a namespace expanded with the
bindings provided by the fixtures.


%%% Local Variables: 
%%% mode: latex
%%% TeX-master: "manual"
%%% End: 
