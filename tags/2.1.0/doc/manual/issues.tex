
\section{Issues, bugs and enhancements}\index{bugs}\index{enhancements}
Tickets for NST bugs and wishes are on NST's trac:
\begin{center}
  \texttt{https://svn.sift.info:3333/trac/nst/}
\end{center}

\begin{enumerate}
\item The criterion \texttt{:perf} might be extended to provide
  (implementation-dependent) checks on memory limits as well as time
  limits.
\item Some sort of timeout mechanism --- perhaps
  implementation-dependent --- could be helpful for measuring
  correctness via termination in more complicated algorithms.
\end{enumerate}

\subsection{Removed features}
\label{temp-removed}
Some features of the NST 0.9 have not yet been reimplemented.  Tell
John if one is urgent for you.

\subsubsection{Test group documentation}
The \texttt{def-test-group} form should allow group
documentation:\index{documentation@\texttt{:documentation}}.
\begin{verbatim}
    (:documentation DOC-STRING)
\end{verbatim}

\subsubsection{Alternate fixture definitions}
The
\texttt{def-capture/restore-fixtures}\index{def-capture/restore-fixtures@\texttt{def-capture/restore-fixtures}}
declaration binds \texttt{nil} to a collection of variables in the
extent of associated test groups.  This form is useful when hiding
some developers' state from tests intended to simulate a
non-development environment.
\begin{center}
\textbf{def-capture/restore-fixtures}
 \textit{name} \textit{variables}
\texttt{\&key} \textit{documentation}
\end{center}

\subsubsection{Anonymous fixture sets}
It is not necessary to name fixture sets; they may be given
anonymously in any situation where a fixture set name is allowed.  The
syntax of anonymous fixture sets
is:\index{fixtures@\texttt{:fixtures}}\index{fixtures!anonymous}
\begin{center}
\texttt{(:fixtures }\textit{(name form)}$^+$\texttt{ )}
\end{center}

\subsubsection{The \texttt{:with} check}
This check assumes that the next item is a list, whose contents are
expanded into the methods-and-forms.\index{with@\texttt{:with}} This
check is especially useful with the \texttt{def-test-cri\-ter\-ion}
command below.
\\ Syntax: \texttt{(:with \emph{criterion})}
\\ Example: The following two \texttt{def-test}s are equivalent.
\begin{verbatim}
  (def-test seq1
      (:seq (:predicate symbolp) (:eql 1) (:symbol d))
      '(a 1 d))
  (def-test with-seq1
      (:with (:seq (:predicate symbolp) (:eql 1) (:symbol d)))
      '(a 1 d))
\end{verbatim}

\subsubsection{Compilation deferral control}
In earlier versions the \texttt{:defer-compile} switch on test
definition allowed control over when an expression would be compiled.
\index{defer-compile@\texttt{:defer-compile}}

\subsubsection{In the runtime system}

\subsubsection{The \texttt{:summarize-scheduled} operation}
\index{summarize-scheduled@\texttt{:summarize-scheduled}}%
If \texttt{BOOL} evaluates to non-null, then the runtime system will
print a summary after running scheduled tests with \texttt{:run},
\texttt{:continue}, etc.
\\ Syntax: \texttt{:nst :summarize-scheduled BOOL}

\subsubsection{The \texttt{:summarize-single} operation}
\index{summarize-single@\texttt{:summarize-single}}%
If \texttt{BOOL} evaluates to non-null, then the runtime system will
print a summary after one-time test runs initiated by
\texttt{:run-test}, \texttt{:run-group}, etc.
\\ Syntax: \texttt{:nst :summarize-single BOOL}

\subsection{Marking tests of interest for execution}
\label{nominating-tests}
NST allows tests to be marked for execution by the \texttt{:run}
command.  Tests can be marked by package or group, or as an individual
test.

\subsubsection{The \texttt{:p} operation}
\index{p@\texttt{:p}}%
Indicates that all tests in a particular package should be run.
\\ Syntax: \texttt{:nst :p PACKAGE}

\subsubsection{The \texttt{:g} operation}
\index{g@\texttt{:g}}%
Indicates that all tests in a particular group should be run.
\\ Syntax: \texttt{:nst :g GROUP}

\subsubsection{The \texttt{:t} operation}
\index{t@\texttt{:t}}%
Indicates that the named test should be run.
\\ Syntax: \texttt{:nst :t TEST}

\subsection{Test definition}
\subsubsection{The \texttt{:defer-test-compile} operation}
\index{defer-test-compile@\texttt{:defer-test-compile}}%
Sets whether tests defined subsequently should, by default, defer
compilation of their forms until actually running the test.  This
feature is useful when debugging code involving macros, but changing
this feature in the runtime system can lead to confusion.  It is
surely almost always the right thing to set this flag locally via
def-test-group and def-test.  In fact, this operation may be removed
in a future version of the runtime system.
\\ Syntax: \texttt{:nst :defer-test-compile BOOL}

\subsubsection{The \texttt{:open{\textasteriskcentered}} operation}
\index{open{\textasteriskcentered}@\texttt{:open{\textasteriskcentered}}}%
Multiple fixtures\index{fixtures!debugging} can be \emph{opened} into the
interactive namespace with a single command using the
\texttt{:nst~:open} command:
\\ Syntax: \texttt{:nst :open\* } {\textit{FIXTURE-NAME}{\textasteriskcentered}}
\par
In earlier versions, only one fixture could be given to
\texttt{:open}; this is no longer true, and the \texttt{open*} command
has been removed.

\subsubsection{The \texttt{:open-used} operation}
\index{open-used@\texttt{:open-used}}%
If \texttt{BOOL} evaluates to non-null, then opening a fixture will
always also open the fixtures it uses.  Default is t.
\\ Syntax: \texttt{:nst :open-used BOOL}

\subsubsection{The \texttt{:reopen} operation}
\index{reopen@\texttt{:reopen}}%
If \texttt{BOOL} evaluates to non-null, then fixtures will be
re-opened \emph{e.g.}\ when required multiple times by opening
different fixtures that use them.
\\ Syntax: \texttt{:nst :reopen BOOL}

%%% Local Variables: 
%%% mode: latex
%%% TeX-master: "manual"
%%% TeX-PDF-mode: t
%%% End: 
