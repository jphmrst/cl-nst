%% File quickref.tex
%%
%% NST by John Maraist, based on RRT by Robert Goldman.
%%
%% NST is Copyright (c) 2006-2010 Smart Information Flow Technologies.
%% RRT is Copyright (c) 2005 Robert Goldman, released under the LGPL,
%% and the Lisp-specific preamble to that license.

\documentclass{article}
\usepackage{times}
\usepackage{array}
\usepackage{listings}
\parskip 0.6em
\parindent 0pt

\newenvironment{criteria}%
{\begin{list}{}
    {\setlength{\labelwidth}{0pt}
     \setlength{\leftmargin}{2em}
     \setlength{\rightmargin}{1em}
     \setlength{\itemindent}{0em}}}%
  {\end{list}}
\def\criterion#1#2#3#4{\item[\bfseries #1] #4\par
  \emph{Syntax:}~~\texttt{#2}\\ \emph{Applicable to:}~~#3}
\begin{document}
\thispagestyle{empty}
\begin{center}
  \LARGE Getting started with NST\footnote{This document was last
    updated for NST version 3.0.1.  Written by John Maraist, Smart
    Information Flow Technologies, 211 N.\ First St.\ Suite 300,
    Minneapolis, MN 55401; \textsl{jmaraist} at \textsl{sift.info}.}
\end{center}

The key elements to getting starting with NST are:
%
\\\hspace*{0.5em}--- The syntactic \textbf{forms} for specifying tests
(below).
%
\\\hspace*{0.5em}--- The \textbf{criteria} used in tests
(p.\,\pageref{sec:criteria}).
%
\\\hspace*{0.5em}--- Using NST for \textbf{interactive debugging}
(p.\,\pageref{sec:interact}).
%
\\\hspace*{0.5em}--- Combining NST with \textbf{ASDF} to help automate
testing (p.\,\pageref{sec:asdf}).

\section*{Nomenclature}
\emph{Fixtures} are data structures and values which may be referred
to by name during testing.  NST provides the ability to use fixtures
across multiple tests and test groups, and to inject fixtures into the
runtime namespace for debugging.  Fixtures are defined using the
\texttt{def-fixture} form.

\emph{Groups} of tests can be associated with fixture sets, stateful
initiatization, and stateful cleanup.  Test groups are defined using
the \texttt{def-test-group} form.

Individual tests are declared within test groups using the
\texttt{def-test} form.

Examples of NST fixtures, groups and tests are available in
\textsl{self-test/core/simple-mnst.lisp}~.  The figure on the next
page shows the changes involved in using NST in a system definition.

\section*{Forms}
\input{../gen/def-fixtures_compiler-macro_nst-quickref}
%
\input{../gen/def-test-group_compiler-macro_nst-quickref}
%
\input{../gen/def-test_compiler-macro_nst-quickref} 
The \texttt{startup}, \texttt{:setup}, \texttt{:cleanup},
\texttt{:finish} and \texttt{:fixture} options are as above, but apply
to only this test (and note that for multiple forms for the first two
must be wrapped in a \texttt{progn}).  NST's built-in criteria are
listed below; see the manual for a discussion of the forms which
define new criteria.

\section*{Criteria}
\label{sec:criteria}
Criteria forms have the following structure:
\begin{verbatim}
  (CRITERIA-NAME ARG ARG ... ARG)
\end{verbatim}
but a no-argument criterion use \texttt{(NAME)} can be abbreviated as
\texttt{NAME}.  NST's built-in criteria include:
\def\criteriaGroup#1#2{\par\textbf{#1.}\begin{criteria}}
\def\secText#1{}
\def\endcriteriaGroup{\end{criteria}}
\def\criteriaDoc#1#2#3#4#5#6#7#8{\criterion{#2}{#3}{#5}{#4}}
\def\noExpl{}
\def\hasExpl#1{}
\def\manualOrRef#1#2{#2}

%% \criteriaDoc
%%    1 - criteria name without punctuation
%%    2 - criteria name with punctuation
%%    3 - syntax
%%    4 - explanation
%%    5 - applicable to
%%    6 - example type --- \noEx, \singleEx, \multiEx, \tabbingEx
%%    7 - example(s)

%% \kwdCriteriaDoc
%%    1 - criteria name
%%    2 - syntax
%%    3 - explanation
%%    4 - applicable to
%%    5 - example type --- \noEx, \singleEx, \multiEx, \tabbingEx
%%    6 - example(s)

\def\kwdCriteriaDoc#1#2#3#4#5#6{\criteriaDoc{#1}{:#1}{#2}{#3}{#4}{#5}{#6}}

\begin{criteriaGroup}{Basic checks}

\kwdCriteriaDoc{pass}{:pass}
{A trivial test, which always passes.}
{Any.}
{\singleEx}{(def-check passing-test :pass 3 4 "sd")}

\kwdCriteriaDoc{fail}{(:fail FORMAT ARG ...\ ARG)}
{A trivial test, which always fails.  The format string and arguments
should be suitable for the Lisp \texttt{format} function.}
{Any.}
{\singleEx}{(def-check fails (:fail "Expected a \~{}a" "string") 312)}

\end{criteriaGroup}


\section*{Interactive debugging}
\label{sec:interact}
NST defines a REPL alias \texttt{:nst} under Allegro CL.  The general
form of commands is:
\begin{verbatim}
  :nst COMMAND ARGUMENTS
\end{verbatim}
Use \mbox{\texttt{:nst :help}} for a list of commands, and
\mbox{\texttt{:nst COMMAND :help}} for details about individual
commands.

\section*{ASDF}
\label{sec:asdf}
NST tests can be referenced from ASDF systems, allowing easy
invocation to unit tests relevant to a system.  The system
\textsl{self-test/masdfnst.asd} gives an example of its use.  The
snippet below highlights the difference between non-NST and
NST-oriented ASDF system declarations.
\texttt{\small
\begin{tabbing}
\\;; Force loading NST's ASDF utilities before processing
\\;; this file.
\\{\bfseries (asdf:oos 'asdf:load-op :asdf-nst)}
\\
\\(de\=fpackage :masdfnst-asd
\\   \>(:use :common-lisp :asdf {\bfseries :asdf-nst}))
\\(in-package :masdfnst-asd)
\\
\\(def\=system :masdfnst
\\    \>;; Use the NST-oriented ASDF system definition.
\\    \>{\bfseries :class nst-testable}
\\    
\\    \>:in-order-to ((test-op (load-op :masdfnst)))
\\
\\    \>;; Any one of the six blocks below is reasonable.
\\    \>;; Use exactly one of :nst-package, :nst-group, or
\\    \>;; :nst-test; or any combination of the plural
\\    \>;; versions.
\\
\\    \>{\bfseries ;; (1)}
\\    \>{\bfseries ;; :nst-package :asdf-nst-test}
\\
\\    \>{\bfseries ;; (2)}
\\    \>{\bfseries ;; :nst-group (:asdf-nst-test . core-checks)}
\\
\\    \>{\bfseries ;; (3)}
\\    \>{\bfseries ;; :nst-test (:asdf-nst-test core-checks pass-1)}
\\
\\    \>{\bfseries ;; (4)}
\\    \>{\bfseries ;; :nst-packages (:asdf-nst-test :asdf-nst-test2)}
\\
\\    \>{\bfseries ;; (5)}
\\    \>{\bfseries ;; :nst-packages (:asdf-nst-test)}
\\    \>{\bfseries ;; :nst-groups ((:asdf-nst-test2 . :g1a))}
\\    
\\    \>{\bfseries ;; (6)}
\\    \>{\bfseries :nst-groups ((:asdf-nst-test2 . :g1))}
\\    \>{\bfseries :nst-tests (}\={\bfseries (:asdf-nst-test2 :g1a :fix0)}
\\		\> \>{\bfseries (:asdf-nst-test :core-checks :warn-1))}
\\    
\\    \>:components ( ...\ ))
\end{tabbing}}


\end{document}

%%% Local Variables: 
%%% mode: latex
%%% TeX-master: t
%%% TeX-PDF-mode: t
%%% End: 
