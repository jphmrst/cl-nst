\section{The runtime system}
The runtime system provides several operations for scheduling and
running tests, and debugging failing and erring tests.  The operations
are accessible from
the~\texttt{nst-cmd}\index{nst-cmd@\texttt{nst-cmd}} macro.  Under
Allegro, the top-level alias \texttt{:nst}\index{nst@\texttt{:nst}}
provides a shorthand to this function; for the sake of brevity we use
this shorthand below.

The \texttt{:help} command gives a complete inventory of runtime
system commands:\index{help@\texttt{:help}}
\begin{verbatim}
 :nst :help
 (nst-cmd :help)
\end{verbatim}

There are three commands for running tests:
\begin{itemize}
\item\texttt{:nst :run-package {\itshape package-name}}\index{run-package@\texttt{:run-package}}
\par Run all tests defined in groups under the given package.
\item\texttt{:nst :run-group {\itshape group-name}}\index{run-group@\texttt{:run-group}}
\par Run all tests in the given group.
\item\texttt{:nst :run-test {\itshape group-name} {\itshape test-name}}\index{run-test@\texttt{:run-test}}
\par Run the named test.
\end{itemize}

There are two commands for (re)printing the results of tests:
\begin{itemize}
\item\texttt{:nst :report}\index{report@\texttt{:report}}
\\ \texttt{:nst :report {\itshape package-name}}
\\ \texttt{:nst :report {\itshape group-name}}
\\ \texttt{:nst :report {\itshape group-name} {\itshape test-name}}
\item\texttt{:nst :detail}\index{detail@\texttt{:detail}}
\\ \texttt{:nst :detail {\itshape package-name}}
\\ \texttt{:nst :detail {\itshape group-name}}
\\ \texttt{:nst :detail {\itshape group-name} {\itshape test-name}}
\end{itemize}
The \texttt{:report} command summarizes successes, failures and
errors; the \texttt{:detail} command gives more detailed information
about individual tests.

The \texttt{:clear} command erases NST's internal record of test
results.

The \texttt{:set} and \texttt{:unset} commands adjust NST's
configuration.
\begin{itemize}
\item\texttt{:nst :set {\itshape property} {\itshape value}}\index{set@\texttt{:set}}
\item\texttt{:nst :unset {\itshape property} {\itshape value}}\index{unset@\texttt{:unset}}
\end{itemize}
There are currently two properties which can be manipulated by \texttt{:set}\,:
\begin{itemize}
\item\texttt{:verbose}\index{verbose@\texttt{:verbose}}~~ Controls the
  level of output at various points of NST.  Valid settings are:
  \begin{tightlist}
  \item\texttt{:silent} (aka \texttt{nil})
  \item\texttt{:quiet} (aka \texttt{:default})
  \item\texttt{:verbose} (aka \texttt{t})
  \item\texttt{:vverbose}
  \end{tightlist}

  The \texttt{:report} and \texttt{:detail} commands operate by
  setting minimum levels of verbosity.

\item\texttt{:debug-on-error}\index{debug-on-error@\texttt{:debug-on-error}}~~
  When this property has a non-nil value, NST will exit into the
  debugger when it catches an error.

  The \texttt{:debug} command is a short-cut for setting this
  property.\index{debug@\texttt{:debug}}
\end{itemize}

Fixtures\index{fixtures!debugging} can be \emph{opened} into the
interactive namespace for debugging with the
\texttt{:nst~:open}\index{open@\texttt{:open}} command:\index{open@\texttt{:open}}%
\\ Syntax: \texttt{:nst :open FIXTURE-NAME FIXTURE-NAME ... FIXTURE-NAME}
\\ Example:
\begin{verbatim}
  CL-USER(75): (nst:def-fixtures small-fixture ()
                  (fix-var1 3)
                  (fix-var2 'asdfg))
  NIL
  CL-USER(76): (boundp 'fix-var1)
  NIL
  CL-USER(77): :nst :open small-fixture
  Opened fixture SMALL-FIXTURE.
  CL-USER(78): fix-var1
  3
  CL-USER(79): 
\end{verbatim}
Fixtures can be opened into a different package than where they were
first defined, but these bindings are in addition to the bindings in
the original package, and are made by a symbol import to the
additional package.

Calling \texttt{:nst} or \texttt{(nst-cmd}) without a command argument
repeats the last test-executing command.

%%% Local Variables: 
%%% mode: latex
%%% TeX-master: "manual"
%%% End: 
